



\begin{theorem}
    The structure of the category $\Syn_{E,F}$ with respect to the parameters $\lambda, \tau$ admits the following identifications:
  
    \begin{enumerate}
      \item The category $\lambda^{-1}\Syn_{E,F}$ of $\lambda$-local objects is equivalent to the category $\Syn_E$. In addition, the map $\bS\to \bS/\lambda$ is $\bE_\infty$ and $\Mod(\bS/\lambda)$ embeds fully faithfully into the category $\Stable(\nu_E F_{**}\nu_E F)$.
      \item The category $\tau^{-1}\Syn_{E,F}$ of $\tau$-local objects is equivalent to the category $\Syn_E$. In addition, the map $\bS\to \bS/\tau$ is $\bE_\infty$ and $\Mod(\bS/\tau)$ embeds fully faithfully into the category $\Stable(\nu_F E_{**}\nu_F E)$.
      \item The category $\tau^{-1}\lambda^{-1}\Syn_{E,F}$ is equivalent to the category $\Sp$ of spectra.
      \item The category $\Mod(\bS/(\tau,\lambda))$ emebds fully faithfully into the category $\Stable(\cP_{**}\cP)$ where $\cP$ is the object $E_*F$ and $\cP_{**}\cP$ is defined to be the homology of $\cP\otimes \cP$ in the category $\Stable(E_*E)$ or $\Stable(F_*F)$.
    \end{enumerate}
    All of the equivalences and functors described are symmetric monoidal.
    
  \end{theorem}
  
  \begin{proof}
    Claim (1) is \ref{todo}, claim (2) is \ref{todo}, and claim (4) is \ref{todo}. Claim (3) follows from either (1) or (2) as we already know the effect of inverting the deformation parameter in $\Syn_E$ by \cite{todo}.
  \end{proof}
  
  In order to prove these results, we will need the following sites whose categories of spherical sheaves will provide new models for some of the above.
  
  \begin{definition}[The $\tau$-local site]
    Let $\Syn_E^{\fp, \tau=1}$ denote the subcategory of $\tau^{-1}\Syn_E$ consisting of those $\tau$-local synthetic spectra which are compact in $\tau^{-1}\Syn_E$ and whose $\tau^{-1}(\nu_E F)$-homology is finitely generated and projective over $\pi_{*,*}\tau^{-1}(\nu_E F)$. We say that a map in this category is a cover if it is a surjection on $\tau^{-1}(\nu_E F)$-homology.
  \end{definition}
  
  \begin{definition}[The mod-$\tau$ site]
    Let $\Syn_E^{\fp, \tau=0}$ denote the subcategory of $\Mod(\Syn_E, \bS/\tau)^\omega$ consisting of objects whose $(\nu_E F/\tau)$-homology is finitely generated and projective over $\pi_{*,*}(\nu_E F/ \tau)$. A map in this category is a cover if it induces a surjection on $\nu_E F/ \tau$-homology.
  \end{definition}
  
  \begin{lemma}
    The sites $\Syn_E^{\fp, \tau=1}$ and $\Syn_E^{\fp, \tau=0}$ are both excellent $\infty$-sites with the coverages described above.
  \end{lemma}
  
  \begin{proposition}\label{prop:spfpissyntauone}
    There is an equivalence of excellent $\infty$-sites $\Sp^\fp_F\simeq \Syn_E^{\fp, \tau=1}$ induced by realization $\Re:\Syn_E\to \Sp$. As a result, there is a symmetric monoidal equivalence $\Sh_\Sigma^\Sp(\Syn_E^{\fp, \tau=1})\simeq \Syn_F$ which identifies the sheaf t-structures on both sides.
  \end{proposition}
  
  \begin{proof}
    To avoid ambiguity, let us write $\Re$ for the $\tau$-inversion functor when viewed as having codomain $\Sp$ and $\tau^{-1}$ for the mapping telescope internal to $\Syn_E$. With this notation, note that $\Re$ factors through $\tau^{-1}$.
    
    
    It then suffices to show that there is a symmetric monoidal equivalence $\Sh_{\Sigma}^\Sp(\Syn_E^{\fp, \tau=1})\simeq \Syn_F$. We claim that $\Re$ restricts to a functor $\Syn_E^{\fp, \tau=1}\to \Sp^\fp_F$ and that this is an equivalence of sites. 
  
    First we show that the functor restricts as described. If $X\in \Syn_E^{\fp, \tau=1}$ this amounts to proving that $\Re(X)$ is compact and that it has finitely generated and projective $F$-homology. Note first that the two notions of compactness coincide, due to the equivalence $\tau^{-1}\Syn_E\simeq \Sp$. Then we can directly compare the two homologies as we have
    \begin{align*}
      \pi_{**}(\tau^{-1}\nu_E F\otimes X)\cong F_*\Re(X)[\tau^{\pm}]
    \end{align*}
    and aftering inverting $\tau$, the synthetic weight becomes superfluous as $F_*\Re(X)$ has the same information. As a result the finite generation and projectivity criterion coincide. Finally, the same argument shows that the two notions of coverage coincide.
  \end{proof}
  
  \begin{proposition}
    There is a symmetric monoidal equivalence  of categories $\tau^{-1}\Syn_{E,F}\simeq \Sh_{\Sigma}^\Sp(\Syn_E^{\fp, \tau=1})$.
  \end{proposition}
  
  \begin{proof}
    We claim that $(\Syn_E^{\fp, \tau=1},\tau^{-1}\Syn_{E,F})$ is a recovery pair (see Definition \ref{todo}). It is clear that it contains enough compact generators as it contains the bigraded shifts of the $\tau$-inverted bisynthetic spheres. It remains to check that for $X\in \Syn_E^{\fp, \tau=1}$ the object $Y(X)$ is connective in the sheaf t-structure on $\Sh_\Sigma^\Sp(\Syn_E^{\fp, \tau=1})$. However, this is already known to be the case by Proposition \ref{prop:spfpissyntauone}.
  \end{proof}
  
  \begin{proposition}
    There is a symmetric monoidal equivalence  of categories $\Mod(\Syn_{E,F}, \bS/\tau)\simeq \Sh_{\Sigma}^\Sp(\Syn_E^{\fp, \tau=0})$.
  \end{proposition}
  
  \begin{theorem}
    There is a symmetric monoidal equivalence $\Mod(\Syn_{E,F})\simeq \Syn_F$.
  \end{theorem}
  
  
  \subsection{$t$-structures}
  
  In the category $\Syn_E$ of $E$-synthetic spectra developed by \cite{Pst22}, there is a natural $t$-structure which plays an important role in the structure of the category. This $t$-structure is a specialization of a general $t$-structure on spherical sheaves, whose heart can also be identified:
  
  \begin{definition}[\cite{Pst22}]
  \label{con_cocon_defin}
  Suppose $\cC$ is an additive $\infty$-category and let $Sh_{\Sigma}^{\Sp}(\cC)$ denote the category of spectra-valued spherical sheaves on $\cC$. An object $X\in Sh_{\Sigma}^{\Sp}(\cC)$ is \textit{connective} if the sheafification of the presheaf $\pi_nX$ defined by
  $$
  c\in\cC\mapsto \pi_nX(c)
  $$
  satisfies $\pi_nX=0$ for $n<0$. An object $X$ is \textit{coconnective} if $\Omega^{\infty}X$ is a discrete sheaf of spaces.
  \end{definition}
  
  \begin{proposition}[\cite{Pst22}]
  \label{general_sheaf_tstruct_prop}
  The pair $(\Sh_{\Sigma}^{\Sp}(\cC)_{\geq 0},\Sh_{\Sigma}^{\Sp}(\cC)_{\leq 0})$ of full subcategories of connective and coconnective objects determines a right
  complete $t$-structure on $\Sh_{\Sigma}^{\Sp}(\cC)$ compatible with filtered colimits. Moreover, there is a canonical equivalence $\Sh_{\Sigma}^{\Sp}(\cC)^\heartsuit\simeq \Sh_{\Sigma}^{\mathrm{Set}}(\cC)$ between the heart of this $t$-structure and the category of
  spherical sheaves of sets.
  \end{proposition}
  
  When specializing to $\cC=\Sp_E^{fp}$ for an Adams-type spectrum $E$, \cite{Pst22} shows that the functor of additive $\infty$-sites $E_*(-):\Sp_E^{fp}\to\Comod_{E_*E}^{fp}$ induces an equivalence on spherical sheaves of sets. Together with work of Goerss-Hopkins, this gives a nice identification of the heart $\Syn_E^\heartsuit$ of the $t$-structure on $\Syn_E$ in terms of $E_*E$-comodules:
  
  \begin{theorem}[\cite{GH05},\cite{Pst22}]
  If $E$ is an Adams-type spectrum, then the functor of additive $\infty$-sites $E_*(-):\Sp_E^{fp}\to\Comod_{E_*E}^{fp}$ induces an equivalence $Sh_{\Sigma}^{\mathrm{Set}}(\Sp_E^{fp})\simeq Sh_{\Sigma}^{\mathrm{Set}}(\Comod_{E_*E}^{fp}).$
  In particular, there are equivalences
  $$
  \Syn_E^{\heartsuit}\simeq Sh_{\Sigma}^{\mathrm{Set}}(\Comod_{E_*E}^{fp})\simeq \Comod_{E_*E}.
  $$
  \end{theorem}
  
  For $X\in\Syn_E$, \cite{Pst22} also identifies an explicit formula for the homotopy objects $\pi_k^\heartsuit X$ in terms of synthetic $E$-homology:
  
  \begin{theorem}[\cite{Pst22}]
  \label{SynE_homology_tstruct}
      At the level of graded abelian groups, there's an isomorphism
  $$
  (\pi_k^\heartsuit X)_l\cong \nu E_{k+l,l}X
  $$
  In particular, $X$ is connective if and only if $\nu E_{k,w}X$ is concentrated in non-negative Chow degree $k-w\geq 0$.
  \end{theorem}
  
  If $X=\nu Y$ is the synthetic analog of a spectrum $Y$, the calculation
  $$
  \nu E_{*,*}\nu Y\cong \nu (E\otimes Y)_{*,*}\cong E_*Y[\tau],
  $$
  where $E_*Y$ is concentrated in bidegree $(k,k)$, shows that $\nu Y$ is always connective in this $t$-structure. In particular, this implies that $\nu Y\otimes C\tau$ lies in $\Syn_E^\heartsuit$. This fact is key in relating $\Mod_{C\tau}(\Syn_E)$ to $\Stable_{E_*E}$ and the $E$-Adams spectral sequence for $Y$ to the $\tau$-Bockstein spectral sequence for $\nu Y$ in $\Syn_E$.
  
  \bigskip
  
  Just as there are two deformation parameters, the category $\Syn_{E,F}$ will have two t-structures corresponding to bisynthetic $E$ and $F$ homology. We first study the t-structure related to $F$ in section \ref{F_tstruct_subsec}. This $t$-structure comes about in the exact same way that the $t$-structure in $\Syn_E$ appears. We also prove several results about this $t$-structure, analogous to results in \cite{Pst22}, which will be useful later for identifying $\Syn_{E,F}[\lambda^{-1}]$ and $\Mod_{C\lambda}(\Syn_{E,F})$ in terms of more familiar categories.
  
  \bigskip
  
  We then study a $t$-structure related to $E$ in \ref{E_tstruct_subsec}. The connective objects of this $t$-structure are controlled by $\nu^2(E)$-homology, analogous to Theorem~\ref{SynE_homology_tstruct}. We also prove several results about this $t$-structure, analogous to results in \cite{Pst22}, which will be useful later for identifying $\Syn_{E,F}[\tau^{-1}]$ and $\Mod_{C\tau}(\Syn_{E,F})$ in terms of more familiar categories.
  
  \subsection{$t$-structure for $F$}
  \label{F_tstruct_subsec}
  
  Since $\Syn_{E,F}$ is the category of spherical sheaves on an additive $\infty$-structure, we immediately get a $t$-structure on $\Syn_{E,F}$ via Definition~\ref{con_cocon_defin} and Proposition~\ref{general_sheaf_tstruct_prop}:
  
  \begin{proposition}
  \label{F_bisyn_tstruct_prop}
  The pair $((\Syn_{E,F})_{\geq 0}^F,(\Syn_{E,F})_{\leq 0}^F)$ of full subcategories of connective and coconnective objects determines a right
  complete $t$-structure on $\Syn_{E,F}$ compatible with filtered colimits. Moreover, there is a canonical equivalence $\Syn_{E,F}^{F,\heartsuit}\simeq Sh_{\Sigma}^{\mathrm{Set}}((\Syn_E)_{\nu F}^{fp})$ between the heart of this $t$-structure and the category of
  spherical sheaves of sets.  
  \end{proposition}
  
  \begin{remark}
  We use the superscript $F$ to emphasize that this $t$-structure is related to $F$ and $\lambda$. This will become clearer later in the subsection when we relate the $t$-structure to $\nu^2F$-homology. We will also use the notation $\tau_{\geq n}^F, \tau_{\leq n}^F$ for the associated truncation functors.
  \end{remark}
  
  We can identify the heart in a similar manner to \cite{Pst22}:
  
  \begin{theorem}
      The heart $\Syn_{E,F}^{F,\heartsuit}$ is equivalent to $\Comod_{\nu_EF_{*,*}\nu_EF}$. (monoidal conditions should be added too)
  \end{theorem}
  
  \begin{proof}
      By Proposition~\ref{F_bisyn_tstruct_prop}, the heart is equivalent to $Sh_{\Sigma}^{\mathrm{Set}}((\Syn_E)_{\nu F}^{fp})$. By (ref. to lemma in Section 2), the morphism of $\infty$-sites $$\nu_EF_{*,*}(-):(\Syn_E)_{\nu F}^{fp}\to\Comod_{\nu_EF_{*,*}\nu_EF}^{fp}$$ is one which reflects coverings and admits a common envelope. By \cite[Rem. 2.50]{Pst22}, this induces an adjoint equivalence $$Sh_{\Sigma}^{\mathrm{Set}}((\Syn_E)_{\nu F}^{fp})\rightleftarrows Sh_{\Sigma}^{\mathrm{Set}}(\Comod_{\nu_EF_{*,*}\nu_EF}^{fp})\,.$$
  The bigraded Hopf algebroid $(\nu_EF_{*,*},\nu_EF_{*,*}\nu_EF)$ is Adams, in the sense of \cite[Def. 3.1]{Pst22}, by (Lemma in Section 2 which proves that it's Adams). By a bigraded version of \cite[2.1.12]{GH05}, \cite[Thm. 3.2]{Pst22} there is an equivalence
  $$
  \Comod_{\nu_EF_{*,*}\nu_EF}\simeq Sh_{\Sigma}^{\mathrm{Set}}(\Comod_{\nu_EF_{*,*}\nu_EF}^{fp}),
  $$
  and the result follows.
  \end{proof}
  
  Now we work towards identifying the homotopy objects $\pi_k^{F,\heartsuit}X$ in terms of $\nu^2F$-homology.
  
  \begin{lemma}
  \label{F_dual_tstruct_lemma}
      For $X\in\Syn_{E,F}$, the graded components of the $\nu_EF_{*,*}\nu_EF$-comodule $\pi_k^{F,\heartsuit}X$ are described by
      $$
  (\pi_k^{F,\heartsuit}X)_{l,m} \cong \colim_\alpha \pi_kX(\Sigma^{l,m}D\nu_E F_\alpha),
      $$
      where $F\simeq \colim_\alpha F_\alpha$ is a presentation of $F$ as a filtered colimit of $F$-finite projective spectra.
  \end{lemma}
  
  \begin{proof}
      This is essentially a bigraded version of \cite[Lemma 4.17]{Pst22} and the proof is similar to the proof of that lemma. By \cite[Thm. 2.58]{Pst22}, the sheaf $\pi_k^{F,\heartsuit}X\in Sh_{\Sigma}^{\mathrm{Set}}((\Syn_E)_{\nu F}^{fp})$ is representable by some comodule $N$; i.e. $$(\pi_k^{F,\heartsuit}X)(-)\simeq \Hom_{\nu_EF_{*,*}\nu_EF}(\nu_EF_{*,*}(-),N).$$
      Now notice that $\nu_EF_{*,*}\nu_EF\simeq \colim_\alpha \nu_EF_{*,*}\nu_EF_\alpha$, since $\nu_E$ commutes with filtered colimits, and $E_*(D\nu_EF_\alpha)\cong\Hom_{\nu_EF_{*,*}\nu_EF}(\nu_EF_{*,*}\nu_EF_\alpha,\nu_EF_{*,*})$. Then by \cite[Lemma 3.3]{Pst22}, as a bigraded abelian group
      $$
  N_{l,m}\cong \colim_\alpha \pi_k^{F,\heartsuit}X(\Sigma^{l,m}D\nu_EF_\alpha).
      $$
      By a bigraded version of \cite[Lemma 3.25]{Pst22},
      $$
  \colim_{\alpha}\pi_k^{F,\heartsuit}X(\Sigma^{l,m}D\nu_EF_\alpha)\cong\colim_{\alpha} \pi_kX(\Sigma^{l,m}D\nu_EF_\alpha),
      $$
      which completes the proof.
  \end{proof}
  
  \begin{theorem}
  \label{F_homol_tstruct_theorem}
      For $X\in\Syn_{E,F}$, there is an isomorphism
      $$
  (\pi_k^{F,\heartsuit}X)_{l,m}\cong\nu^2F_{k+l,m,l}X,
      $$
      where $\nu^2F_{*,*,*}(-)$ denotes bisynthetic $F$-homology.
  \end{theorem}
  
  \begin{proof}
      Again, this is a similar proof to \cite[Thm. 4.18]{Pst22}. We have that
      \begin{equation*}
       \begin{aligned}
        \nu^2F_{k+l,m,l}X&\cong [\bS^{k+l,m,l},\nu^2F\otimes X] \\
        &\cong \colim_\alpha[\Sigma^k\mu_F(\bS^{l,m}_E),\nu^2F_\alpha\otimes X] \\
        &\cong \colim_\alpha [\Sigma^k\mu_F(\Sigma^{l,m}D\nu_EF_\alpha),X] \\
        &\cong \colim_\alpha \pi_kX(\Sigma^{l,m}D\nu_EF_\alpha) \\
        &\cong (\pi_k^{F,\heartsuit}X)_{l,m}.
      \end{aligned}   
      \end{equation*}
   The first isomorphism is by definition, the second isomorphism follows from (definition from Section 2 about trigraded spheres) and equivalence $\nu^2F\simeq\colim_\alpha \nu^2F_\alpha$, the fourth isomorphism follows from (lemma from Section 2 which shows that $map(\mu_FP,X)\simeq \Omega^\infty(X(P))$ for $P\in(\Syn_E)_{\nu F}^{fp}$), and the fifth isomorphism follows from Lemma~\ref{F_dual_tstruct_lemma}.   
  \end{proof}
  
  As a corollary, we get the following analog of \cite[Cor. 4.19]{Pst22}:
  
  \begin{corollary}
  \label{F_chow_degree_cor}
  A bisynthetic spectrum $X\in\Syn_{E,F}$ is in $(\Syn_{E,F})_{\geq 0}^F$ if and only if $\nu^2F_{k,w,v}X =0$ for Chow degree $k-v<0$. 
  \end{corollary}
  
  \begin{proof}
      In this $t$-structure, $X\in\Syn_{E,F}$ is in $(\Syn_{E,F})_{\geq 0}^F$ if and only if $\pi_k^{F,\heartsuit}X$ vanishes for $k<0$. By Theorem~\ref{F_homol_tstruct_theorem}, this happens exactly when $k-v<0$.
  \end{proof}
  
  This result is what motivates naming this $t$-structure after $F$. As a consequence, we see that the $\nu F$-synthetic analog of an $E$-synthetic spectrum $Y$ is always connective.
  
  \begin{corollary}
      If $Y\in\Syn_E$, then $\mu_FY\in (\Syn_{E,F})_{\geq 0}^F$.
  \end{corollary}
  
  \begin{proof}
      Consider the homology calculation
  \begin{equation*}
      \begin{aligned}
          \nu^2F_{t,w,v}\mu_FY &\cong \mu_F(\nu F\otimes Y)_{t,w,v} \\
          &\cong \nu F_{t,w} Y[\lambda]\, ,
      \end{aligned}
  \end{equation*}
  where $\nu F_{k,w} Y$ lives in tridegree $(k,w,k)$. The first isomorphism follows from (lemma in Section 2 about when $\mu_F$ is symmetric monoidal) and the second isomorphism follows (lemma in Section 2 about homotopy of $\nu F$-module). The result then follows from Corollary~\ref{F_chow_degree_cor}.
  \end{proof}
  
  This means that for the $\nu F$-synthetic analog of an $E$-synthetic spectrum $Y$, the tensor product $\mu_FY\otimes C\lambda$ lives in the heart $\Syn_{E,F}^{F,\heartsuit}$.
  
  \begin{corollary}
  If $Y\in\Syn_E$, then $\Sigma^{0,0,-1}\mu_FY\simeq \tau_{\geq 1}^F(\mu_FY)$ and $\mu_F Y\otimes C\lambda\simeq \tau_{\leq 0}^F(\mu_FY)$. In particular, $\mu_FY\otimes C\lambda\in \Syn_{E,F}^{F,\heartsuit}$.  
  \end{corollary}
  
  \begin{proof}
     Again, the proof is similar to the proof of \cite[Lemma 4.29]{Pst22}. Consider the cofiber sequence
     $$
  \Sigma^{0,0,-1}\mu_FY\xrightarrow{\lambda}\mu_FY\to \mu_FY\otimes C\lambda
     $$
     By Corollary~\ref{F_chow_degree_cor}, it's clear that $\Sigma^{0,0,-1}\mu_FY$ is 1-connective. By using the definition of $\mu_F$ and the colimit-comparison definition of $\lambda$, it follows that $\mu_FY\otimes C\lambda$ lives in $(\Syn_{E,F})_{\leq 0}^F$. The result then follows.
  \end{proof}
  
  \begin{remark}
  Similar to $\Syn_E$, we see that $\mu_FY\otimes C\lambda$ is lives in an algebraic category; namely the category of $\nu_EF_{*,*}\nu_E F$-comodules. In Section 4, we will show that, in fact, $\mu_FY\otimes C\lambda$ can be identified with the comodule $\nu_{E}F_{*,*}Y$ and there is an embedding $\Mod_{C\lambda}(\Syn_{E,F})\hookrightarrow \Stable_{\nu_EF_{*,*}\nu_EF}$ of $C\lambda$-modules into the stable comodule category associated to the bigraded Hopf algebroid $(\nu_EF_{*,*},\nu_EF_{*,*}\nu_EF)$.    
  \end{remark}
  
  \subsection{$t$-structure for $E$}
  \label{E_tstruct_subsec}
  
  \section{Specializations by $\tau,\lambda$}
  
  \section{The Categorified Miller Square}