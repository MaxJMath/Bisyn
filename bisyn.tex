\documentclass[10pt]{amsart}
\usepackage[letterpaper,margin=1in,footskip=0.25in]{geometry}


%\usepackage{garamondlibre}
\usepackage{times}
%\usepackage{CormorantGaramond}
%\usepackage{baskervald}
\usepackage{microtype}
\usepackage{eucal}
\usepackage{setspace}
\usepackage{mathrsfs}
\usepackage{tikz-cd}
\usepackage{todonotes}
\usepackage{soul}

\usepackage[
backend=biber,
style=alphabetic,
sorting=nyt, maxbibnames=20, maxalphanames=99
]{biblatex}
%\usepackage{pdfpages}

\addbibresource{refs.bib}


\usepackage{amsmath,amssymb,amsthm}
\usepackage{mathtools}
\usepackage{mathabx}
\makeatletter
  \newcommand{\supsize}{%
    \expandafter\ifx\csname S@\f@size\endcsname\relax
      \calculate@math@sizes
    \fi
    \csname S@\f@size\endcsname
    \fontsize\sf@size\z@\selectfont
  }
  \DeclareRobustCommand{\tsup}[1]{%
    \leavevmode\raise.9ex\hbox{\supsize #1}%
  }
  \DeclareTextSymbolDefault{\textprimechar}{OMS}
  \DeclareTextSymbol{\textprimechar}{OMS}{48}
  \DeclareRobustCommand{\tprime}{\tsup{\textprimechar}}
  \ProvideTextCommandDefault{\cprime}{\tprime}
\makeatother


\usepackage{enumitem}
\setlist{noitemsep}

\usepackage[pdfusetitle,colorlinks]{hyperref}
\hypersetup{allcolors=blue}
\usepackage[capitalise,noabbrev]{cleveref}
\crefformat{equation}{\ensuremath{(#2#1#3)}}
\crefmultiformat{equation}{\ensuremath{(#2#1#3)}}{ and~\ensuremath{(#2#1#3)}}{, \ensuremath{(#2#1#3)}}{, and~\ensuremath{(#2#1#3)}}

\theoremstyle{definition}

\numberwithin{figure}{section}
\numberwithin{equation}{section}

\newtheorem{theorem}[figure]{Theorem}
\newtheorem{lemma}[figure]{Lemma}
\newtheorem{construction}[figure]{Construction}

\newtheorem{corollary}[figure]{Corollary}
\newtheorem{proposition}[figure]{Proposition}
\newtheorem{definition}[figure]{Definition}
\newtheorem{notation}[figure]{Notation}
\newtheorem{exercise}[figure]{Exercise}
\newtheorem{remark}[figure]{Remark}
\newtheorem{example}[figure]{Example}
\newtheorem{conjecture}[figure]{Conjecture}
\newtheorem{convention}[figure]{Convention}

\newtheorem{thm}[figure]{Theorem}
\newtheorem{var}[figure]{Variant}
\newtheorem{lem}[figure]{Lemma}
\newtheorem{cons}[figure]{Construction}

\newtheorem{cor}[figure]{Corollary}
\newtheorem{prop}[figure]{Proposition}
\newtheorem{defn}[figure]{Definition}
\newtheorem{notn}[figure]{Notation}
\newtheorem{rem}[figure]{Remark}

\newtheorem{manualtheoreminner}{Theorem}
\newenvironment{manualtheorem}[1]{%
    \renewcommand\themanualtheoreminner{#1}%
  \manualtheoreminner
}{\endmanualtheoreminner}

\newcommand{\op}{\mathrm{op}}

\newcommand{\poly}[1]{{#1}[x_1,\ldots,x_n]}
\newcommand{\cA}{\mathcal{A}}
\newcommand{\cB}{\mathcal{B}}
\newcommand{\cC}{\mathcal{C}}
\newcommand{\cD}{\mathcal{D}}
\newcommand{\cE}{\mathcal{E}}
\newcommand{\cF}{\mathcal{F}}
\newcommand{\cG}{\mathcal{G}}
\newcommand{\cH}{\mathcal{H}}
\newcommand{\cI}{\mathcal{I}}
\newcommand{\cJ}{\mathcal{J}}
\newcommand{\cK}{\mathcal{K}}
\newcommand{\cL}{\mathcal{L}}
\newcommand{\cM}{\mathcal{M}}
\newcommand{\cN}{\mathcal{N}}
\newcommand{\cO}{\mathcal{O}}
\newcommand{\cP}{\mathcal{P}}
\newcommand{\cQ}{\mathcal{Q}}
\newcommand{\cR}{\mathcal{R}}
\newcommand{\cS}{\mathcal{S}}
\newcommand{\cT}{\mathcal{T}}
\newcommand{\cU}{\mathcal{U}}
\newcommand{\cV}{\mathcal{V}}
\newcommand{\cW}{\mathcal{W}}
\newcommand{\cX}{\mathcal{X}}
\newcommand{\cY}{\mathcal{Y}}
\newcommand{\cZ}{\mathcal{Z}}
\newcommand{\pp}{\mathbf{p}}
\newcommand{\mm}{\mathbf{m}}
\newcommand{\mbfa}{\mathbf{A}}
\newcommand{\sX}{\mathscr{X}}
\newcommand{\sY}{\mathscr{Y}}
\newcommand{\sch}{\operatorname{Sch}}
\newcommand{\GG}{\mathbf{G}}
\newcommand{\BB}{\mathbf{B}}
\newcommand{\pic}{\operatorname{Pic}}

\newcommand{\MU}{\mathrm{MU}}
\newcommand{\BP}{\mathrm{BP}}
\newcommand{\SU}{\mathrm{SU}}
\newcommand{\BU}{\mathrm{BU}}


\newcommand{\gr}{\mathrm{gr}}
\newcommand{\fil}{\mathrm{fil}}

\newcommand{\BGL}{\mathrm{BGL}}
\newcommand{\Thick}{\mathrm{Thick}}
\newcommand{\Th}{\mathrm{Th}}
\newcommand{\Ext}{\operatorname{Ext}}

\newcommand{\spaces}{\mathcal{S}}
\newcommand{\one}{\mathrm{1}}
\newcommand{\ord}{\mathrm{ord}}
\newcommand{\wt}{\mathrm{wt}}
\newcommand{\unfinished}{\textcolor{red}{INCOMPLETE }}
\newcommand{\done}{\textcolor{green}{DONE }}

\newcommand{\anss}{{}^{\mathrm{an}}\mathrm{E}}
\newcommand{\genanss}[1]{{}^{\mathrm{#1}}\mathrm{E}}
\newcommand{\fgenanss}{{}^{\mathrm{F}}\mathrm{E}}

\newcommand{\mayss}{{}^{\mathrm{may}}\mathrm{E}}
\newcommand{\vlines}{\mathrm{VL}}
\newcommand{\page}{\mathrm{page}}
\newcommand{\intercept}{\mathrm{incpt}}

\newcommand{\kos}[1]{\mathbf{K}_\bullet(\mathbf{#1})}



\newcommand{\ass}{\operatorname{Ass}}
\newcommand{\spec}{\operatorname{Spec}}
\newtheoremstyle{cited}{.5\baselineskip\@plus.2\baselineskip\@minus.2\baselineskip}{.5\baselineskip\@plus.2\baselineskip\@minus.2\baselineskip}{\itshape}{}{\bfseries}{\bfseries .}{5pt plus 1pt minus 1pt}{\thmname{#1}\thmnumber{ #2}\thmnote{ \normalfont #3}}
\theoremstyle{cited}
\newtheorem{citedthm}[figure]{Theorem}
\newtheorem{citedprop}[figure]{Proposition}
\newtheorem{citedcor}[figure]{Corollary}

%Operators
\DeclareMathOperator{\Aut}{Aut}
\newcommand{\isom}{\operatorname{Isom}}
\newcommand{\sets}{\operatorname{Sets}}
\newcommand{\htensor}{\hat{\otimes}}

%Objects
\newcommand{\tensor}{\otimes}
\newcommand{\into}{\hookrightarrow}
\newcommand{\aff}{\mathbb{A}}
\newcommand{\mf}[1]{\mathbf{#1}}
\newcommand{\ess}{\operatorname{Es}/S}

\newcommand{\bZ}{\mathbb{Z}}
\newcommand{\bN}{\mathbb{N}}
\newcommand{\bS}{\mathbb{S}}
\newcommand{\bD}{\mathbb{D}}
\newcommand{\bE}{\mathbb{E}}
\newcommand{\bF}{\mathbb{F}}

%Maps
\newcommand{\qcoh}[1]{\operatorname{Qcoh}(#1)}
\newcommand{\id}{\mathrm{id}}
\newcommand{\der}[3]{\textrm{Der}_{#1}(#2,#3)}

\newcommand{\affs}{\operatorname{Aff}/S}
\newcommand{\ev}{\operatorname{ev}}
\newcommand{\alg}{\operatorname{Alg}}
\newcommand{\cech}{\operatorname{Cech}}
\newcommand{\tot}{\operatorname{Tot}}
\newcommand{\Fun}{\operatorname{Fun}}
\newcommand{\fun}{\Fun}
\newcommand{\Alg}{\operatorname{Alg}}
\newcommand{\CAlg}{\operatorname{CAlg}}
\newcommand{\PSigma}{\operatorname{P}_{\Sigma}}
\newcommand{\Psh}{\operatorname{P}}

\newcommand{\sCAlg}{\operatorname{sCAlg}}
\renewcommand{\poly}{\mathrm{poly}}
\newcommand{\colim}{\operatorname{colim}}
\newcommand{\cof}{\operatorname{cof}}

\newcommand{\HH}{\operatorname{HH}}
\newcommand{\THH}{\operatorname{THH}}

\newcommand{\Fil}{\mathrm{fil}}
%Arrows
\newcommand{\longisoto}{\overset{\sim}{\longrightarrow}}
\newcommand{\spm}{\operatorname{Spm}}

\newcommand{\B}{\mathbf{B}}
\newcommand{\Nm}{\operatorname{Nm}}

\newcommand{\PrL}{\mathrm{Pr}^{\mathrm{L}}}
\newcommand{\Ab}{\mathrm{Ab}}
\newcommand{\Sp}{{\mathcal{S}\mathrm{p}}}
\newcommand{\Mod}{\mathrm{Mod}}
\newcommand{\Set}{{\mathcal{S}\mathrm{et}}}


\newcommand{\HE}{\mathcal{H}\mathrm{Env}}
\newcommand{\HESt}{\HE^{\Sp}}
\newcommand{\Map}{\mathrm{Map}}


\newcommand{\Cond}{\mathrm{Cond}}
\newcommand{\Condbar}{\mathbf{Cond}}
\newcommand{\Cat}{\mathrm{Cat}_\infty}
\newcommand{\proet}{\text{pro-et}}
\newcommand{\pep}{\ast_{\proet}}
\newcommand{\Sh}{\mathrm{Sh}}
\newcommand{\KCond}{\mathrm{K}^{\text{c}}}
\newcommand{\LCA}{\mathrm{LCA}}
\newcommand{\bA}{\mathbb{A}}
\newcommand{\Perf}{\mathrm{Perf}}
\newcommand{\PerfCat}{\mathrm{PerfCat}}
\newcommand{\lc}{\mathrm{lc}}
\newcommand{\LCAbar}{\mathbf{LCA}}

\newcommand{\Condf}[1]{\mathbf{#1}}

\newcommand{\bfC}{\mathbf{C}}
\newcommand{\bfD}{\mathbf{D}}
\newcommand{\CondFun}{\Condf{Fun}}
\newcommand{\CondCat}{\Condf{Cat}}
\newcommand{\cond}{\text{c}}
\newcommand{\CondD}{\Condf{D}}
\newcommand{\CondPerf}{\Condf{Perf}}
\newcommand{\CondLCA}{\Condf{LCA}}
\newcommand{\CondCone}{\Condf{cone}}
\newcommand{\CondK}{\Condf{K}}
\newcommand{\Condpi}{\Condf{\varpi}}

\newcommand{\Gal}{\mathrm{Gal}}
\newcommand{\dKSel}{\mathrm{dK}^{\mathrm{Sel}}}


\newcommand{\maxnote}[1]{\textcolor{blue}{#1}}
\newcommand{\peternote}[1]{\textcolor{orange}{#1}}

\newcommand{\cb}{\mathrm{cb}}

\newcommand{\fp}{\mathrm{fp}}
\newcommand{\Hom}{\mathrm{Hom}}
\newcommand{\Syn}{\mathcal{S}\mathrm{yn}}
\newcommand{\Bisyn}{\mathcal{B}\mathrm{isyn}}
\newcommand{\Stable}{\mathcal{S}\mathrm{table}}
\newcommand{\loc}{\mathrm{loc}}

\renewcommand{\H}{\mathrm{H}}
\newcommand{\G}{\mathrm{G}}

\newcommand{\Comod}{\mathcal{C}\mathrm{omod}}

\newcommand{\Tor}{\mathrm{Tor}}

\renewcommand{\Re}{\mathrm{Re}}

\newcommand{\Coop}{\mathrm{Coop}}

\newcommand{\af}{\mathrm{af}}

\begin{document}

\title{Bisynthetic Spectra and the Miller Square}
\author{Maxwell Johnson and Peter Marek}

\maketitle

%\section*{To do}
%\begin{itemize}
%    \item Lemma that $\nu_EF_{*,*}(-):(\Syn_E)_{\nu F}^{fp}\to\Comod_{\nu_EF_{*,*}\nu_EF}^{fp}$ is morphism of $\infty$-sites which reflects coverings and admits a common envelope
%    \item Other lemmas about other maps of $\infty$-sites? Or wait to do that later
%    \item Lemma that $\nu_EF_{*,*}\nu_EF$ is Adams Hopf algebroid a la \cite[Def. 3.1]{Pst22}
%    \item Lemma about when $\mu_F$ is symmetric monoidal
%    \item Lemma about homotopy of $\mu_F$-modules being lambda free
%\end{itemize}

\tableofcontents

\section{Introduction}

Few problems are as central to the field of algebraic topology as computing the stable homotopy groups of spheres. Although the problem remains--and likely will remain--infinitely far from a complete solution. Despite this, we still understand these groups today far better than when they first introduced; key to this progress has been the Adams Spectral Sequence. 

The Adams spectral sequence is a tool for computing homotopy groups starting from the homological algebra of a generalized homology theory. Explicitly, to a suitable ring spectrum $E$ the Adams spectral sequence is an associated spectral sequence of signature
\[
  \Ext_{E_*E}(E_*,E_*) \Rightarrow \pi_*(\bS^\wedge_E)
\]
where the spectrum on the right is a completion of the sphere with respect to the theory $E$. Not all choices are created equal, however, and by far the Adams spectral sequences we understand best are those associated to the choice $E=\bF_p$, known as the classical Adams spectral sequence, and the choice $E=\MU$, the Adams-Novikov spectral sequence. 

The two theories are intimiatley connected by the Thom reduction map $\MU\to \bF_p$. This map is one of rings, and as a result induces a map of filtrations and therefore spectral sequences

% https://q.uiver.app/#q=WzAsNCxbMCwwLCJcXGdlbmFuc3N7XFxNVX1fciJdLFswLDEsIlxccGlfKihcXGJTKSJdLFsxLDEsIlxccGlfKihcXGJTXlxcd2VkZ2VfcCkiXSxbMSwwLCJcXGdlbmFuc3N7XFxiRl9wfV9yIl0sWzAsMSwiIiwwLHsibGV2ZWwiOjJ9XSxbMywyLCIiLDAseyJsZXZlbCI6Mn1dLFswLDNdLFsxLDJdXQ==
\[\begin{tikzcd}
	{\genanss{\MU}_r} & {\genanss{\bF_p}_r} \\
	{\pi_*(\bS)} & {\pi_*(\bS^\wedge_p)}
	\arrow[from=1-1, to=1-2]
	\arrow[Rightarrow, from=1-1, to=2-1]
	\arrow[Rightarrow, from=1-2, to=2-2]
	\arrow[from=2-1, to=2-2]
\end{tikzcd}\].

However, this comparison map is of limited use. While it is compatible with the filtration, classes can "jump" filtration: a nonzero class $\{x\}\in \genanss{\MU}$ which detects an essential $x\in \pi_*(\bS)$ might have trivial image in $\genanss{\bF_p}$ and $x$ itself may not be detected until some class $\{x'\}$ shows up in higher filtraiton. This happens often in practice; we would like a more robust approach to transfering information between the two tools.

\subsection{The Miller Square}

In studying the homotopy groups of spheres and their connection to the theory $\MU$, algebraic topologists discovered that the classes could be organized by a so-called "chromatic" filtration. Recalling that after localizing at a prime $p$ complex cobordism $\MU$ breaks up as theories $\BP$ with $\BP_*=\bZ_{(p)}[v_1,v_2,...]$, chromatic homotopy theory may be understood as studying the sphere locally against the rising ideals $I_n=(p,v_1,v_2,...,v_{n-1})$. It turns out there are two a priori natural ways to do this. The first is to fix a finite complex $V$ supporting a nontrivial $v_n$-self map and localize with respect to the theory $T(n)=v_n^{-1}V$. The second is to localize against the Morava $K$-theories $K(n)$ which satisfy $\pi_*K(n)=\bF_p[v_n]$. There are natural comparison maps
\[
  X\to L_{T(n)}X\to L_{K(n)}X
\]
and the telescope conjecture asks when this second map is an equivalence for finite spectra $X$.

We now know that in general this map fails to be an equivalence, c.f. \cite{todo}. However, when $n=1$ it turns out to always hold. This was shown first for $p=2$ by \cite{todo} and later for $p>2$ by Miller \cite{todo}. The latter's solution resulted in and dependended on the titular square.

Recall from above that the classical Adams and the Adams-Novikov spectral sequences have $E_2$-pages given by $\Ext_{\MU_*\MU}(\MU_*,\MU_*)$ and $\Ext_{\cA}(\bF_p,\bF_p)$ respectively. One approach to studying these $\Ext$ groups is to produce an auxiliarry filtration resulting in a spectral sequencing converging to the approriate Adams $E_2$ page. We will be particularly interested in two such spectral sequences, one for $E=\bF_p$ and one for $E=\MU$. To describe them, first recall that the dual Steenrod algebra takes the form
\[
\cA_* =\begin{cases}
    \bF_p[\xi_1,\xi_2,...] & p=2\\
    \bF_p[\xi_1,\xi_2,...] \otimes \bF_p\langle \tau_0,\tau_1,...\rangle & p >2
\end{cases}
\]
Let $\cP_*\subset \cA_*$ denote the subalgebra $\bF_p[\xi_1^2,\xi_2^2,...]$ at $p=2$ and $\bF_p[\xi_1,\xi_2,...]$ when $p>2$. The inclusion $\cP_*\hookrightarrow \cA_*$ induces an extension of Hopf algebras
\[
  \cP_* \to \cA_* \to \cE_*
\]
where $\cE_*$ is exterior on countable many generators as an algebra. The extension induces a Cartan-Eilenberg \cite{todo} type spectral sequence of signature
\[
  \Ext_{\BP_*\BP}(\bF_p, \Ext_{\cE}(\bF_p,\bF_p))\Rightarrow \Ext_{\cA_*}(\bF_p,\bF_p) =\genanss{\bF_p}_2.
\]
Because $\cE$ is exterior we in addition have $\Ext_{\cE}(\bF_p,\bF_p)=\bF_p[q_0,q_1,...]$. Similarly recall that 
\[
  \BP_*\BP=\BP_*[t_1,t_2,...]
\]
and that the map $\BP_*\BP\to \cA_*$ induced by the Thom reduction takes the generators $t_i$ to the polynomial generators of $\cP_*$ of the same index. Let us abuse notation to denote the generators of $\cP_*$ by $t_i$ in either case. This morphism is clearly surjective onto its image and has kernel the ideal $I=(p,v_1,v_2,...)\subset \BP_*$.


When $E=\bF_p$ we use the Cartan-Eilenberg spectral sequence. The $I=(p,v_1,v_2,...)$-adic filtration on the Hopf algebroid $\BP_*\BP$ then induces a spectral sequence of signature
\[
  \bigoplus_k \Ext_{\BP_*\BP/I}(\BP_*/I,I^k/I^{k+1})\Rightarrow \Ext_{\BP_*\BP}(\BP_*,\BP_*) = \genanss{\MU}_2.
\]
Taking the sum across all $k$, we have $\oplus_k I^k/I^{k+1}=\bF_p[q_0,q_1,...]$ and as a result we have an isomorphism between the $E_2$ pages of the Cartan-Eilenberg and the algebraic Novikov. The Miller square is then the square of spectral sequences
% https://q.uiver.app/#q=WzAsNCxbMSwwLCJcXEV4dF97XFxjUF8qfShcXGJGX3AsIFxcYkZfcFtxXzAscV8xLC4uLl0pIl0sWzIsMSwiXFxFeHRfe1xcY0F9KFxcYkZfcCxcXGJGX3ApIl0sWzEsMiwiXFxwaV8qKFxcYlNeXFx3ZWRnZV9wKSJdLFswLDEsIlxcRXh0X3tcXEJQXypcXEJQfShcXEJQXyosXFxCUF8qKSJdLFsxLDIsIlxcdGV4dHtjbGFzc2ljYWwgQWRhbXN9IiwwLHsibGV2ZWwiOjJ9XSxbMCwxLCJcXHRleHR7Q2FydGFuLUVpbGVuYmVyZ30iLDAseyJsZXZlbCI6Mn1dLFswLDMsIlxcdGV4dHthbGcuIE5vdn0iLDIseyJsZXZlbCI6Mn1dLFszLDIsIlxcdGV4dHtBZGFtcy1Ob3Zpa292fSIsMix7ImxldmVsIjoyfV1d
\[\begin{tikzcd}
	& {\Ext_{\cP_*}(\bF_p, \bF_p[q_0,q_1,...])} \\
	{\Ext_{\BP_*\BP}(\BP_*,\BP_*)} && {\Ext_{\cA}(\bF_p,\bF_p)} \\
	& {\pi_*(\bS^\wedge_p)}
	\arrow["{\text{alg. Nov}}"', Rightarrow, from=1-2, to=2-1]
	\arrow["{\text{Cartan-Eilenberg}}", Rightarrow, from=1-2, to=2-3]
	\arrow["{\text{Adams-Novikov}}"', Rightarrow, from=2-1, to=3-2]
	\arrow["{\text{classical Adams}}", Rightarrow, from=2-3, to=3-2]
\end{tikzcd}\]
which provides a new comparison between our spectral sequences.

\begin{remark}
  One common heuristic in computational stable homotopy theory is that the classical Adams is best used at $p=2$ and the Adams-Novikov at $p>2$. This intuition can be made precise by the Miller square: when $p>2$ the Cartan-Eilenberg spectral sequence collapses for degree reasons and as a result, the classical Adams differentials have to remove all of the error between $\Ext$ over the dual Steenrod algebra and $\pi_*(\bS^\wedge_p)$. The algebraic Novikov spectral sequence however does not collapse at $p>2$ and therefore if one can begin directly with $\Ext$ over $\BP_*\BP$ then a good portion of the work has already been done for them by the algebraic Novikov.
\end{remark}

The above remark is key to Miller's resolution of the height one telescope conjecture at odd primes. His key technical input which takes advantage of the square allows us to transfer differentials from the top left edge to the bottom right.

\begin{theorem}[Miller]
  Every $z\in \Ext_{\cA}(\bF_p,\bF_p)$ has a Cartan-Eilenberg representative $\{z\}$ whose algebraic Novikov target $d_2^{\text{aN}}(\{z\})$ detects $-d^{\bF_p}_2(z)$ in the Cartan-Eilenberg spectral sequence.
\end{theorem}

When the Cartan-Eilenberg collapses, then, the work of choosing Cartan-Eilenberg representatives is done for us and we may compute Adams $d_2$-differentials by computing the (relatively easier) algebraic Novikov differentials on their Cartan-Eilenberg representatives. We prove a strengthening of this theorem:

\begin{manualtheorem}{A}
  \label{thm:A}
  Every $z\in \Ext_{\cA}(\bF_p,\bF_p)$ has a Cartan-Eilenberg representative $\{z\}$ whose algebraic Novikov target $d_r^{\text{aN}}(\{z\})$ detects $-d^{\bF_p}_r(z)$ in the Cartan-Eilenberg spectral sequence.
\end{manualtheorem}

\textcolor{orange}{This is probably not true as stated just a placeholder while I write the intro. Also TODO: Figure out what BX adds to this story concretely.}

\subsection{Synthetic Spectra and Categorification}

We prove Theorem \ref{thm:A} by way of categorification. In \cite{Pst22} Pstragowski associated to suitable ring spectra $E$ a category $\Syn_E$ of $E$-synthetic spectra. Each $E$-synthetic spectrum encodes both the data of a classical spectrum and its homotopy, as well as a spectral sequence computing those homotopy groups. In particular, each classical spectrum $X$ has a functorially associated synthetic analog $\nu_EX$ which encodes both the homotopy groups of $X$ and the $E$-Adams spectral sequence for $X$.

We generalize this construction and produce a category $\Syn_{E,F}$ associated to a pair of rings $E,F$ assumed to have certain compatibility. As a result, we form a square of categories:
% https://q.uiver.app/#q=WzAsNCxbMSwwLCJcXFN5bl97XFxNVSxcXEZfcH0iXSxbMiwxLCJcXFN5bl97XFxiRl9wfSJdLFswLDEsIlxcU3luX3tcXE1VfSJdLFsxLDIsIlxcU3AiXSxbMiwzXSxbMSwzXSxbMCwyXSxbMCwxXSxbMiwwLCIiLDEseyJvZmZzZXQiOi0zfV0sWzEsMCwiIiwxLHsib2Zmc2V0IjozfV0sWzMsMSwiIiwyLHsib2Zmc2V0IjozfV0sWzMsMiwiIiwyLHsib2Zmc2V0IjotM31dXQ==
\[\begin{tikzcd}
	& {\Syn_{\MU,\bF_p}} \\
	{\Syn_{\MU}} && {\Syn_{\bF_p}} \\
	& \Sp
	\arrow[from=1-2, to=2-1]
	\arrow[from=1-2, to=2-3]
	\arrow[shift left=3, from=2-1, to=1-2]
	\arrow[from=2-1, to=3-2]
	\arrow[shift right=3, from=2-3, to=1-2]
	\arrow[from=2-3, to=3-2]
	\arrow[shift left=3, from=3-2, to=2-1]
	\arrow[shift right=3, from=3-2, to=2-3]
\end{tikzcd}\]
in which we go up by taking synthetic analogs and go down by passing to underlying spectra or synthetic spectra. The original categories $\Syn_E$ can be viewed as deformations of spectra. They contain a parameter $\tau$ that when inverted recovers the category of spectra and when killed is closed related to derived stable comodules over the Hopf algebroid $E_*E$; we write
\[
\Syn_E[\tau^{-1}] \simeq \Sp\;\;\text{ and }\;\;\Syn_E/\tau \hookrightarrow \Stable(E_*E).
\]
Ignoring the difference between the latter two categories momentarily, we may summarize this information in a table in which the top row specifies the value of $\tau$.

\begin{table}[!ht]
  \centering
  \begin{tabular}{|l|l|l|}
  \hline
      $\tau$ & $1$ & $0$ \\ \hline
      $\Syn_E$ & $\Sp$ & $\Stable(E_*E)$ \\ \hline
  \end{tabular}
\end{table}

We show that our category of Bisynthetic Spectra is suitably similar, then, to the category of Synthetic Spectra:

\begin{manualtheorem}{B}
  The category $\Syn_{E,F}$ is presentably symmetric monoidal and stable. It contains parameters $\tau$ and $\lambda$, associated to $E$ and $F$ respectively, resulting in specializations where the algebraic $= 0$ specializations are in general only fully faithful inclusions.
  \begin{table}[!ht]
    \centering
    \begin{tabular}{|l|l|l|l|}
    \hline
        ~ & $\tau$ & $1$ & $0$ \\ \hline
        $\lambda$ & $\Syn_{E,F}$ & $\Syn_F$ & $\Stable(E_*E)$ \\ \hline
        $1$ & $\Syn_E$ & $\Sp$ & $\Stable(\nu_FE_{**}\nu_FE)$ \\ \hline
        $0$ & $\Stable(\nu_EF_{**}\nu_EF)$ & $\Stable(F_*F)$ & $\Stable(E_*F)$ \\ \hline
    \end{tabular}
  \end{table}
\end{manualtheorem}

\textcolor{orange}{Bottom right is obviously speculative but I just wrote it to see how it feels. Also I cannot get this table to show up where I want it.}

\subsection{The Diagonal Spectral Sequence}

The parameters $\tau,\lambda$ above arise as certain (graded) self-maps of the $\otimes$-unit $\bS^{0,0,0}$ in the category $\Syn_{E,F}$. In \cite{Pst22} it is shown that the $\tau$-Bockstein spectral sequence
\[
\pi_{*,*}(\bS^{0,0}/\tau)[\tau] \cong \Ext_{E_*E}(E_*,_E*)[\tau] \Rightarrow \pi_{*,*}(\bS)
\]
is a \textit{rigid} deformation of the $E$-Adams spectral sequence. In the category of bisynthetic spectra, then, we may consider the Cartesian square
% https://q.uiver.app/#q=WzAsNCxbMCwwLCJcXGJTLyhcXHRhdVxcbGFtYmRhKSJdLFsxLDAsIlxcYlMvXFx0YXUiXSxbMCwxLCJcXGJTL1xcbGFtYmRhIl0sWzEsMSwiXFxiU157MCwwLDB9L1xcdGF1XFxvdGltZXMgXFxiU157MCwwLDB9L1xcbGFtYmRhIl0sWzAsMV0sWzAsMl0sWzIsM10sWzEsM11d
\[\begin{tikzcd}
	{\bS/(\tau\lambda)} & {\bS/\tau} \\
	{\bS/\lambda} & {\bS^{0,0,0}/\tau\otimes \bS^{0,0,0}/\lambda}
	\arrow[from=1-1, to=1-2]
	\arrow[from=1-1, to=2-1]
	\arrow[from=1-2, to=2-2]
	\arrow[from=2-1, to=2-2]
\end{tikzcd}\]
as well as the $\tau\lambda$ Bockstein spectral sequence
\[
\pi_{***}(\bS^{0,0,0}/(\tau\lambda))[\tau\lambda]\Rightarrow \pi_{***}(\bS^{0,0,0})
\]
\begin{manualtheorem}{C}
  Some collection of statements about this Spectral sequence lol
\end{manualtheorem}



\section{Generalized Synthetic Spectra}

Throughout this section we fix a presentably symmetric monoidal stable $\infty$-category $\cC$. Our goal in this section is to generalize the original construction of synthetic spectra due to \cite{Pst22} for a broader class of homology theories on stable $\infty$-categories. We note that in some sense this has been accomplished by \cite{PP23}, however, the categories produced therein are subject to technical limitations. For example, we will need the existence of a nice symmetric monoidal structure on our synthetic categories which is not constructed in loc. cit.

\subsection{Homological Contexts} In order to set up a theory of synthetic spectra, we must first specify what the necessary inputs are. The original theory depends only on choosing a (nice) ring spectrum $E$, but makes substantial use of the underlying structure present in $\Sp$ and the concomitant properties of the induced homology theory $E_*$. We spell out below what we believe is a suitably general theory of homological contexts, i.e., theories which act sufficiently like functors $X\mapsto \pi_*(X\otimes E)$.


\begin{definition}
  A local grading on a category $\cD$ is an auto-equivalence $-[1]:\cD\to \cD$. A category is said to be locally graded if it has a chosen local grading.
\end{definition}

\begin{example}
  All stable $\infty$-categories are locally graded by by the formal suspension $-[1]:=\Sigma$. 
\end{example}

\begin{example}
  A graded category $\Fun(\bZ,\cD)$ where $\bZ$ is the discrete category on the integers is locally graded by the shift functor induced by $n\mapsto n\pm 1$ on $\bZ$.
\end{example}

Both examples above frequently arise as special cases of the following:

\begin{example}
  If $\cD$ is monoidal and $X$ is a Picard-object, i.e., it is $\otimes$-invertible, then the functor $-\otimes X$ forms a local grading on $\cD$.
\end{example}

\begin{definition}
  Let $\cD$ be a presentably symmetric monoidal stable $\infty$-category and let $\cA$ be an abelian 1-category equipped with a local grading. A functor $\pi_\star:\cD\to \cA$ is said to be a homotopy groups functor if it commutes with filtered colimits, is lax monoidal and additive, and if in addition it
  \begin{enumerate}
    \item sends cofibers in $\cD$ to exact sequences in $\cA$,
    \item and intertwines the local gradings $\H(\Sigma X)=H(X)[1]$ naturally.
  \end{enumerate}
\end{definition}

\begin{remark}
  An important consequence of the above definition is that any such $\H$ will send a cofiber sequence $X\to Y\to Z$ in $\cD$ to a long exact sequence in $\cA$ as rotating the cofiber in $\cD$ results in a local-grading-shift in $\cA$.
\end{remark}

The above definition is engineered not just to capture the classical examples of homotopy groups of spectra, but also the categories of (genuine) equivariant spectra, motivic spectra, and synthetic spectra. In general, we will want the extra flexibility of considering homotopy groups (and later homology theories) which are multigraded and have long exact sequences with respect to the formal suspension.

\begin{definition}
  \label{def:homcontext}
  A \textit{homological context} is the data of a presentably symmetric monoidal  stable $\infty$-category $\cC$, a presentably symmetric monoidal left adjoint $\H:\cC\to \cD$, and a homotopy groups functor $\pi_\star:\cC\to \cA$. If $\G$ is the right adjoint to $\H$ we write $\H_\star$ for the composite $\pi_\star\circ \G\circ \H$.
\end{definition}

\begin{example}
  All examples we study in this paper will be given by the following data. First we fix $\cC$ as above and define $\pi_\star$ to be some (usually graded) abelian enrichment of mapping out of the unit. We then let $R\in \Alg(\cC)$ and put $\H:=-\otimes R$ valued in $\Mod(R)$ so that it is symmetric monoidal and preserves colimits by assumption. Then $\H_\star$ corresponds to taking the homotopy groups on the underlying $\cC$-objects after extending by $-\otimes R$.
\end{example}

\begin{definition}
  Note that by construction the composite $\H_\star$ is lax monoidal, so that the image of the unit $\H_\star(\one_{\cC})$, which we will refer to as the coefficient ring for $\H$ and denote by $\H_\star$, is a commutative ring object in the 1-category $\cA$.
\end{definition}

The essential reason for separating the functors $\H$ and $\pi_\star$ in the definition of a homological context is to be able to make sense of the homological comonad.

\begin{definition}
  The homological comonad for $\H$ is the comonad on $\cD$ (the codomain of $\H$) induced by the adjunction between $\H$ and its right adjoint $\G$, i.e. it is the comonad $\H\G: \cD\to \cD$. 
\end{definition}

\begin{definition}
  The $\cC$-object of $\H$-cooperations is defined to be the object $\Coop(\H):=\H\G\H(\one_{\cC})$. The algebraic cooperations of $\H_\star$ are defined to be $\pi_\star\Coop(\H)$ and we denote them by $\H_\star \H$.
\end{definition}

\begin{example}
  If $\cC=\Sp$ and $E$ is a commutative ring, then the spectral and algebraic cooperations associated to $H=-\otimes E$ are given by the spectrum $E\otimes E$ and $E_*E$ respectively.
\end{example}

Note that the object $\Coop(\H)$ is acquires both a left and right module structure over $\H(\one_{\cC})$ from the structure morphisms arising from the comonad. The lax monoidality of $\pi_\star$ then preserves these structures, so that $\H_\star \H$ acquires a left and right module structure over $\H_\star$.

\subsection{Representable Homology Theories and the Adams-Type Condition}

\begin{definition}
    A homological context $(\cC,\H,\pi_\star)$ is said to be representable if there exists a ring $R\in \cC$ such that $\H$ is equivalent to the functor $-\otimes R: \cC\to \Mod(R)$.
\end{definition}

\begin{definition}
  If $(\cC,-\otimes R,\pi_\star)$ is a representable homological context, we define the functor
  \[
  R^\star:\cC\to \cA
  \]
  by the formula $X\mapsto \pi_\star\Map_{\cC}^{\cC}(X, R)$ and refer to this as the cohomology theory associated to $R$.
\end{definition}

\begin{definition}[Adams-Type]
  Suppose that $(\cC, -\otimes R, \pi_\star)$ is a representable homological context. Then this context is in addition Adams-Type if there exists a filtered diagram $X_\alpha\in \cC^\fp_\H$ such that
  \begin{itemize}
    \item there is an equivalence $R\simeq \colim X_\alpha$ and
    \item the natural maps $R^\star X_\alpha \to \Hom_{R_\star}(R_\star X_\alpha, R_\star)$ are isomorphisms.
  \end{itemize}
\end{definition}

\begin{remark}
  Note that for any Adams-Type homological context, we have that $R_\star R\cong \colim R_\star X_\alpha$ presents the algebraic cooperations as a filtered colimit of projectives so that $R_\star R$ is flat over $R_\star$ as a right module. This is exactly the assumption used in \cite[Section 12]{BB} to construct the Hopf Algebroid of cooperations.
\end{remark}

\begin{proposition}
  If our homological context is Adams-Type, then the pair $(\H_\star, \H_\star\H)$ is a Hopf algebroid in $\cA$ and for any $X\in \cC$ we have that $\H_\star(X)$ acquires a canonical comodule structure over $\H_\star \H$.
\end{proposition}

\begin{proof}
  The argument is identical to that of \cite{BB}, \textcolor{orange}{Slight issue here of generation. The original argument works by proving it for the sphere and then using a generation argument. We obviously don't have that so probably need an extra assumption on $\cC$. Also need to define cohomology above and make sure it makes sense.}
\end{proof}

\subsection{H-finite sites}

In this section we fix a homological context (Definition \ref{def:homcontext})
\[
\cC\xrightarrow{\H} \cD \xrightarrow{\pi_\star} \cA
\]
whose composition is denoted $\H_\star$. To such a context we will assocaited a site $\cC_{\H}^\fp$ which will encode the relevant properties of the $\H$-Adams spectral sequence.

\begin{definition}\label{def:projectivesite}
  With notation as above, the $\H$-projective site of $\cC$, denoted $\cC^{\fp}_{\H}$, is the full subcategory of $\cC^\omega$ consisting of objects $X$ such that $\H_{\star}(X)$ is finitely generated and projective as an $\H_\star$ module. The coverings in $\cC^\fp_{\H}$ are the single maps $f:X\to Y$ such that $\H_\star(f)$ is an epimorphism, which we call $\H$-epimorphisms for short.
\end{definition}

\begin{lemma}
  \label{lem:sitepullbacks}
  Let $Q,R,P\in \cC^\omega_\H$. Suppose $f:Q\to P$ is an $\H$-epimorphism and $g:R\to P$ is arbitrary. Let $X=Q\times_P R$ denote the pullback in $\cC$. Then $X$ is again in $\cC^\omega_\H$ and $X\to R$ is an $\H$-epimorphism.
\end{lemma}

\begin{proof}
  First we note that the pullback may equivalently be described via the fiber sequence
  \[
  X\to Q\oplus R \xrightarrow{f-g} P
  \]
  and as a result is compact as the fiber of a map between compact objects. Moreover, because $\H(Q)\to \H(P)$ is an $\H$-epimorphism and because $\H(Q)$ is projective, we get a lift $\tilde g:\H(R)\to \H(P)$ which then splits the long exact sequence. As a result, the long exact sequence breaks up into short exact sequences, and we use the 2-out-of-3 property to claim that $\H(X)$ is therefore finitely generated and projective. That $\H(X)\to \H(R)$ is an epimorphism follows from the splitting.
\end{proof}

\begin{definition}
  We say that a homological context $\H_\star$ is projectively monoidal if 
  the symmetric monoidal structure on $\cC$ restricts to the site $\cC^\fp_\H$ if in addition the restriction $\cC^\fp_\H\to \cA$ is a monoidal functor.
\end{definition}

\begin{remark}
  The above condition is resonably common. One way it frequently arises is the existence of a spectral sequence 
  \[
  \Tor_{\cA}(\H_\star(X),\H_\star(Y))\Rightarrow \H_\star(X\otimes Y)
  \]
  where the requirement that both terms be projective causes the spectral sequence to collapse immediatley.
\end{remark}

\begin{convention}\label{conv:projectivemonoidality}
  We will assume going forward that the fixed homological context above is in additional projectively monoidal.
\end{convention}

\begin{definition}[\cite{Pst22}]
  A small $\infty$-site is \textit{additive} if the coverings are provided by singletons and the underlying category is additive. 
\end{definition}

\begin{definition}[\cite{Pst22}]
  An additive $\infty$-site is said to be excellent if it is equipped with a symmetric monoidal structure in which every object has a dual and such that the functors $-\otimes P$ preserve coverings for all $P$ in the site.
\end{definition}

\begin{proposition}
  The category $\cC^\fp_\H$ is an excellent site.
\end{proposition}

\begin{proof}
  The symmetric monoidal structure is the one guaranteed by convention \ref{conv:projectivemonoidality}. Because compact objects are dualizable, it is automatic that every object has a dual in $\cC^\fp_\H$. As such, $-\otimes P$ is a right adjoint and preserves all pullbacks for all $P\in \cC^\fp_\H$. It therefore suffices to show that it takes coverings to coverings. But since the functor $H_\star$ is monoidal (again by convention), this is immediate.
\end{proof}

\subsection{H-Synthetic Spectra}

Again we fix a homological context $\cC\to \cD\to \cA$ with notation as in all previous sections. Recall that a presheaf $F:\cC^\op\to \cD$ is said to be spherical if for all $X,Y\in \cC$ the natural map
\[
F(X\amalg Y)\to F(X)\times F(Y)
\]
is an equivalence. A sheaf is said to be spherical if the underlying presheaf is, and the sheafification functor when it exists sends spherical sheaves to spherical presheaves. Spherical presheaves are very well behaved when the category $\cC$ is additive. In particular, in this case we get canonical lifts to grouplike commutative monoids in $\cD$, so long as these make sense. As a result, spherical sheaves of spaces on $\cC$ lift canonically to spherical sheaves of connective spectra.

\begin{definition}\label{def:synsp}
  The category of of Synthetic Spectra with respect to the context above is the category of spherical sheaves of spectra $\Sh^\Sp_\Sigma(\cC^\omega_\H)$ on the $\H$-finite site. We will often drop much of the context data and refer to this as the category $\Syn_\H$ of $\H$-synthetic spectra.
\end{definition}

\begin{lemma}\label{lem:synispsms}
  The category $\Syn_\H$ is presentably symmetric monoidal and stable.
\end{lemma}

\begin{proof}
  Stability and presentability follow from \cite[Corollary 2.13]{Pst22} and presentably symmetric monoidality follows from \cite[Proposition 2.30]{Pst22}.
\end{proof}

Because $\cC_\H^\fp$ is a full subcategory of $\cC$, the yoneda embedding extends to a functor $\cC\to \Sh(\cC_\H^\fp)$ after sheafification, which we will denote by $\hat y$ (reserving the undecorated $y$ for the restriction back to $\cC^\fp_\H$). Note that both $y$ and $\hat y$ automatically land in the subcategory of spherical sheaves by the calculation
\[
  \hat y(c)(d\amalg d')=\Map_{\cC}(d\amalg d', c)\simeq \Map_{\cC}(d, c)\times \Map_{\cC}(d', c).
\]
There is then an adjunction (\cite{Pst22})
\[
\Sigma^\infty_+:\Sh_\Sigma(\cC_\H^\fp) \leftrightarrows \Sh_\Sigma^\Sp(\cC_\H^\fp):\Omega^\infty
\]
which allows us to lift both to functors valued in spherical presheaves.

\begin{definition}[Synthetic Analog Functor]\label{def:synanalog}
  The synthetic analog functor $\nu:\cC\to \Syn_\H$ is defined to be $\Sigma_+^\infty \hat y$, the canonical lift of $\hat y$ to a spherical sheaf of spectra.
\end{definition}

\begin{proposition}\label{prop:analogprops}
   The category $\Syn_H$ and its synthetic analog functor enjoy the following properties:
   \begin{enumerate}
    \item For all $P\in \cC_\H^\fp$ there is an equivalence $\Map(\nu P, X)\simeq \Omega^\infty X(P)$.
    \item The category $\Syn_H$ is generated under colimits by the compact objects $\Sigma^{k}\nu P$ for $P\in \cC_\H^\fp$.\footnote{In almost all cases of interest, the synthetic category has multi-graded suspensions. Here we are only interested in the formal suspension in the stable $\infty$-category $\Syn_\H$.}
    \item The functor $\nu$ preserves filtered colimits and direct sums.
    \item The functor $\nu$ is lax monoidal.
    \item The restriction $\nu:\cC_\H^\fp\to \Syn_\H$ is symmetric monoidal.
   \end{enumerate}  
\end{proposition}

\begin{proof}
  Claim (1) is proven identically to \cite[Lemma 4.11]{Pst22}:
  \[
  \Map(\nu P, X)\simeq \Map(\Sigma_+^\infty y(P), X)\simeq \Map(y(P), \Omega^\infty X)\simeq \Omega^\infty X(P).
  \]
  After noting that for $P\in \cC_\H^\fp$ the object $\nu P$ is compact in $\Syn_\H$ by \cite[Cor. 4.12]{Pst22}, the result follows from (1) as these objects can detect equivalences of spherical presheaves levelwise. Both (3) and (4) are true for the same reason: we can write the synthetic analog as the composite
  \[
  \cC\xrightarrow{\hat y} \Sh_\Sigma(\cC_\H^\fp) \xrightarrow{\Sigma_+^\infty}\Syn_\H
  \]
  wherein the second functor is a symmetric monoidal left adjoint, so it suffices to show that $\hat y$ is lax monoidal, preserves filtered colimits, and direct sums. Lax monoidality follows from recognizing that the functor $\hat y$ admits a left left adjoint which is the unique colimit preserving extension of $\cC^\fp_\H\hookrightarrow \cC$ and that this left adjoint is symmetric monoidal, so that its right adjoint its automatically lax monoidal. This is already enough for direct sums, which are both finite limits and colimits. For filtered colimits, we let $X_\alpha$ be a filtered diagram in $\cC$ and note that for any $Y\in \cC_\H^\fp$ we have
  \begin{align*}
    \colim_\alpha \hat y(X_\alpha)(P) & \simeq \colim_\alpha \Map(P, X_\alpha).
  \end{align*}
  But then $P$ is compact, so this filtered colimit is computed levelwise. But filtered colimits of sheaves are computed levelwise, so we are done. Finally we note that the restriction in (5) is the composite $\Sigma^\infty_+ \circ y$ so that it suffices to show that $y:\cC_\H^\fp\to \Sh(\cC_\H^\fp)$ itself is symmetric monoidal, but this property characterizes the Day convolution product of sheaves.
\end{proof}

\begin{lemma}\label{lem:mapoutofproj}
  Let $P\in \cC_\H^\fp$ be a finite projective and $X\in \Sh_\Sigma^\Sp(\cC_\H^\fp)$ a spherical sheaf. Then there is an equivalence of spaces
  \[
    \Map(\nu P, X)\simeq \Omega^\infty X(P).
  \]
\end{lemma}

\begin{proof}
  The proof is identical to \cite[Lem. 4.11]{Pst22}.
\end{proof}

\subsection{The Sheaf t-Structure} 

Recall that for an arbitrary small $\infty$-site $\cT$ the category of sheaves of spectra on $\cT$ inherits a t-structure from the standard t-structure on spectra. Explicitly, the category of coconnective objects consists of the levelwise coconnective sheaves and the category of connective objects is determined against these. We can take homotopy groups levelwise and sheafify to get functors
\[
  \pi^\heartsuit_n:\Sh^\Sp(\cT)\to \Sh^\Ab(\cT)
\]
and which provide an alternative characterization of the t-structure, the connective objects are those sheaves whose sheaf homotopy groups vanish in negative degrees. 

\begin{proposition}[\cite{Pst22}]\label{lem:gentstructure}
  Let $\cT$ be an additive $\infty$-site. The t-structure on $\Sh^\Sp(\cT)$ described above restricts to a right-complete t-structure on spherical sheaves of spectra which is compatible with filtered colimits. Moreover, the heart of this t-structure is equivalent to the category of spherical sheaves of sets $\Sh_\Sigma^\Set(\cT)$.\footnote{Recall that spherical sheaves acquire abelian group structures levelwise so that the category of sheaves of sets is indeed abelian.}
\end{proposition}

\begin{convention}
  Our synthetic categories of interest will have multiple interesting t-structures. For easy of notation, we will refer to this t-structure as the \textit{sheaf t-structure}.
\end{convention}

\subsection{Thread Structures}

The functor $\nu$ does not preserve (co)fiber sequences in general, although we will prove eventually that it preserves certain $\H$-exact cofibers. In particular, $\nu$ will not commute with formal suspensions. This failure is measured by a canonical comparison map
\[
\tau: \Sigma \circ \nu \to \nu \circ \Sigma
\]
induced by the universal property of the pushout defining $\Sigma$. We will refer to this map as the \textit{deformation parameter} of the deformation $\Syn_\H$ of $\cC$. 

\subsection{Recovering Categories via Spherical Sheaves}

The purpose of this section is to provide critera for when a stable $\infty$-category can be recovered as a category of spherical sheaves on an additive $\infty$-site which is a full subcategory. Let $\cC_0$ be a small full subcategory of a stable $\infty$-category $\cC$ equipped with the structure of an additive $\infty$-site. Then there is a functor $Y:\cC\to \Psh(\cC_0,\Sp)$ given by the spectral Yoneda embedding.

\begin{lemma}
  The functor $Y$ factors through $\Sh_\Sigma^\Sp(\cC)$.
\end{lemma}

\begin{proof}
  It is easy to see that the functor is spherical, it remains to show that it is a sheaf. But because $\cC$ was assumed additive, it is enough to show that for any cover $A\to B$ in $\cC$ with fiber $F$ the sequence
  \[
  Y(B)\to Y(A)\to Y(F)
  \]
  is still a fiber. But because $Y$ is given by mapping spectra, this is immediate. Because $Y$ is exact, it suffices to check that it preserves filtered colimits. 
\end{proof}

\begin{convention}
  From here we will only be interested in the functor $Y:\cC\to \Sh_\Sigma^\Sp(\cC_0)$ rather than the version valued in presheaves.
\end{convention}

\begin{lemma}
  \label{lem:recoveryYcocon}
  If, with notation as above, $\cC_0\subset \cC^\omega$ is a recover pair, then the functor $Y$ is cocontinuous.
\end{lemma}

\begin{proof}
  The functor $Y$ is exact by definition, so it suffices to check that it preserve filtered colimits. But this is immediate as we have assumed that $\cC_0$ consists only of compact objects in $\cC$ so that the relevant filtered colimits are computed levelwise.
\end{proof}

\begin{definition}
  A recovery pair is a pair $(\cC_0,\cC)$ of a presentable stable $\infty$-category $\cC$ and a small full subcategory $\cC_0\subset \cC^\omega$ equipped with the structure of an additive $\infty$-site satisfying additional axioms:
  \begin{enumerate}
    \item The (de)suspensions of the objects in $\cC_0$ generate $\cC$ under colimits.
    \item If $X\in \cC_0$ then $Y(X)$ is connective in $\Sh_\Sigma^\Sp(\cC)$.
  \end{enumerate}
\end{definition}

\begin{proposition}\label{prop:recpairrecovers}
  If $(\cC_0,\cC)$ is a recovery pair then the functor $Y$ is an equivalence $\cC\simeq \Sh_\Sigma^\Sp(\cC)$.
\end{proposition}

\begin{proof}
  We know that the functor $Y$ is cocontinuous from Lemma \ref{lem:recoveryYcocon}. To see that it is fully faithful we can therefore restrict to checking on the generating objects within $\cC_0$. Let $Z\in \cC$ be fixed but arbitrary. Then it suffices to check that for all $X\in \cC_0$ the map
  \[
    \Map(X,Z)\to \Map(Y(X),Y(Z))
  \]
  but this latter space is equivalent to $\Map(\Sigma_+^\infty y(X), Y(Z))$ which is itself equivalent to $\Map(y(X), \Omega^\infty Y(Z))$ which is the same as $\simeq\Map(y(X),y(Z))$ and we conclude by the Yoneda lemma. To see that $Y$ is essentially surjective we recall that it is cocontinunous and the image contains a family of generators.
\end{proof} 


\section{Bisynthetic Spectra}

Having established the necessary categorical preliminaries, the remainder of this article will study the example of bisynthetic spectra. The following convention will be enforced throughout the remainder of the document unless specifically stated otherwise. 

\begin{convention}\label{conv:eandf}
  We will fix two Adams-type ring spectra $E,F$ satisfying the following additional assumptions:
  \begin{enumerate}
      \item The spectrum $E$ is a homotopy ring spectrum and $E\to F$ is a homotopy $E$-algebra.
  \end{enumerate}
\end{convention}

\begin{lemma}
  Under the above assumptions, if $X$ is a spectrum such that $E_*X$ is projective over $E_*$, then there is an isomorphism $F_*X=E_*X\otimes_{E_*}F_*$. As a result, there is an inclusion of sites $\Sp^\fp_E\hookrightarrow \Sp^\fp_F$.
\end{lemma}

\begin{proof}
  There is a universal coefficient spectral sequence
  \[
    \Tor_{E_*}(E_*X, F_*)\Rightarrow F_*X
  \]
  which collapses to its edge homomorphism as we assumed $E_*X$ to be projective over $E_*$. This identification will preserve covers, finite generation, and projectivity.
\end{proof}

\begin{lemma}
  Suppose that $E\to F$ is as above and $F=\colim F_\alpha$ is as required in the definition of Adams-Type.
  \begin{enumerate}
    \item The synthetic spectra $\nu_E(F_\alpha)$ have finitely generated and projective $\nu_E F$-homology.
    \item If ? then the spectra $\nu_E(F_\alpha)$ are compact as objects of $\Syn_E$.
  \end{enumerate}
\end{lemma}

\begin{proof}
  We assume that we may write $F\simeq \colim_\alpha F_\alpha$ for a filtered diagram such that each $F_\alpha \in \Sp^\fp_F$.
\end{proof}

\begin{definition}
  The homological context for $(E,F)$-bisynthetic spectra is given by the pair
  \[
  \Syn_{E} \xrightarrow{-\otimes \nu_E F} \Mod(\nu_E F) \xrightarrow{\pi_{*,*}} \Fun(\bZ^2, \Ab)
  \]
  and the resulting site of Definition \ref{def:projectivesite} is denoted $(\Syn_E)_{\nu_E F}^\fp$. It can be explicitly described as the full subcategory of $\Syn_E$ containing those compact objects whose $\nu_E F$ homology is finitely generated and projective over $\nu_E F_{**}$.
\end{definition}

\begin{definition}
  The category of Bisynthetic spectra $\Syn_{E,F}$ is the category of synthetic objects (Def. \ref{def:synsp}) with respect to the $(E,F)$-bisynthetic homological context, i.e., it is the category of spherical sheaves on the site $(\Syn_E)_{\nu_E F}^\fp$. We will denote the synthetic analog functor (Def. \ref{def:synanalog}) by $\mu_F$.
\end{definition}

\begin{remark}
  We use the notation $\mu_F$ rather than $\nu_F$ or $\nu_{\nu_E F}$ or similar as eventually we will construct a full diagram of functors
\[\begin{tikzcd}
	& {\Syn_{E,F}} \\
	{\Syn_E} && {\Syn_F} \\
	& \Sp
	\arrow["{\mu_F}", from=2-1, to=1-2]
	\arrow["{\mu_E}"', dashed, from=2-3, to=1-2]
	\arrow["{\nu_E}", from=3-2, to=2-1]
	\arrow["{\nu_F}"', from=3-2, to=2-3]
\end{tikzcd}\]
where the dashed functor $\mu_E$ is the only functor lacking definition so far.
\end{remark}

\begin{notation}
  We will write $\nu^2$ for the functor $\mu_F\circ \nu_E$. Note that this functor again is fully faithful, preserves filtered colimits, and is lax monoidal as a composition of such functors.
\end{notation}

\begin{notation}
  We will consider $\Syn_{E,F}$ to be a trigraded category, with the convention
  \[
  \bS^{t,w,v}=\Sigma^{t-v}\mu_F\Sigma^{v-w}\nu_E\bS^w
  \]
\end{notation}

\begin{lemma}
  Each sphere $\bS^{t,w,v}$ is $\otimes$-invertible and there are equivalences $\bS^{t,w,v}\otimes \bS^{t',w',v'}\simeq \bS^{t+t',w+w',v+v'}$. The sphere $\bS^{0,0,0}$ is the $\otimes$-unit.
\end{lemma}

\begin{proof}
  The functor $\mu_F$ is symmetric monoidal when restricted to $(\Syn_{E})^\fp_{\nu_E F}$ and the bigraded $E$-synthetic spheres are finite projective for any choice of $F$.
\end{proof}

\begin{notation}
  The category $\Syn_{E,F}$ can be seen to have two distinct deformation parameters. We will denote by $\lambda$ the parameter which arises formally via the synthetic construction; this parameter will encode the $\nu_E F$-Adams spectral sequence. We will denote by $\tau$ the map $\mu_F(\tau)$; it is the parameter associated to the $E$-Adams. With our conventions above these maps have gradings:
  \begin{itemize}
    \item $\lambda:\bS^{0,0,-1}\to \bS^{0,0,0}$
    \item $\tau:\bS^{0,-1,0}\to \bS^{0,0,0}$
  \end{itemize}
\end{notation}

\begin{remark}
  The reader familiar with synthetic spectra may notice that a priori the element $\mu_F(\tau)$ might be $\lambda$-divisible, which would be unsavory. However, our conventions (Conv. \ref{conv:eandf}) prevent this: the assumed map $E\to F$ guarantees that $\tau$ acts freely on $\nu_E F$ and is in particular detected in $\nu_E F$-Adams filtration $0$.
\end{remark}

\begin{definition}
  We define the trigraded mapping objects between bisynthetic spectra to be
  \[
  [X,Y]_{t,w,v}=[\Sigma^{t,w,v}X, Y]
  \]
  and trigraded homotopy groups $\pi_{t,w,v}X=[\bS, X]_{t,w,v}$. As we see above, this leads to $|\tau|=(0,-1,0)$ and $|\lambda|=(0,0,-1)$ as elements of $\pi_{***}\bS$.
\end{definition}

\begin{definition}
  Given a triple $(t,w,v)$ we will say that it has $E$-Chow degree $t-w$ and $F$-Chow-degree $t-v$.
\end{definition}

\begin{lemma}\label{homotopyfchow}
  For $X\in \Syn_{E,F}$ there is an isomorphism $\pi_{t,w,v} X \cong \pi_{t-v}X(\bS^{v,w})$. If $X=\mu_F Y$ for some $Y\in \Syn_{E}$ then we have $\pi_{t,w,v}\mu_F Y = \pi_{t,w} Y$ whenever $t-v\geq 0$, i.e., in positive $F$-Chow-degree.
\end{lemma}

\begin{proof}
  The proofs are identical to \cite[Lem. 4.11, Cor. 4.12]{Pst22} but we record them in order to take care with indexing. For the first claim we compute
  \begin{align*}
    \pi_{t,w,v}X = \pi_0\Map(\bS^{t,w,v}, X) &= \pi_{0}\Map(\Sigma^{t-v}\mu_F\Sigma^{v-w}\nu_E\bS^w, X)\\
    &= \pi_{t-v}\Map(\mu_F\Sigma^{v-w}\nu_E\bS^w, X)\\
    &= \pi_{t-v}\Map(y(\Sigma^{v-w}\nu_E\bS^w), \Omega^\infty X))\\
    &= \pi_{t-v}\Omega^\infty X(\bS^{v,w})
  \end{align*}
  as desired and for the second we observe that $\pi_{t-v}\Omega^\infty\nu Y(\bS^{v,w})=\pi_{t-v}y(Y)(\bS^{v,w})=\pi_{t,w}Y$.
\end{proof}

\section{A Tale of Two t-Structures}

A key structure in the study of Synthetic spectra is the natural t-structure \cite[Sec. 4.2]{Pst22}. It should be little surprise then that the category of bisynthetic spectra will have two important t-structures, which should be seen as encoding information related to one of each of the theories $E$ and $F$. As with many of our constructions, the t-structure associated to the theory $F$ will arise naturally via synthesis whereas we will have to construct the $E$-t-structure by hand.

\begin{notation}
  We use superscripts $E,F$ to distinguish the t-structures, for example, $\tau_{\geq n}^F$ and $\pi_n^{\heartsuit,F}$ will denote the truncations and homotopy groups of the $F$-t-structure and likewise replacing $F$ with $E$ for the $E$-t-structure.
\end{notation}

\subsection{The F-t-structure}

\begin{definition}
  The $F$-t-structure on $\Syn_{E,F}$ is defined to be the sheaf t-structure of Lemma \ref{lem:gentstructure} naturally associated to the synthetic construction.
\end{definition}

The analogous t-structure on $\Syn_E$ admits an equivalent definition: its connective objects are given by those synthetic spectra whose $\nu_EE$-homology is concentrated in positive chow degree \cite[Cor. 4.19]{Pst22}. We will first attempt to study the $F$-t-structure in detail, as it closely mirrors the arguments and development of \cite{Pst22}.

\begin{theorem}
      The heart $\Syn_{E,F}^{F,\heartsuit}$ is equivalent to $\Comod_{\nu_EF_{*,*}\nu_EF}$. (monoidal conditions should be added too)
\end{theorem}
  
\begin{proof}
      By Proposition~\ref{lem:gentstructure}, the heart is equivalent to $Sh_{\Sigma}^{\mathrm{Set}}((\Syn_E)_{\nu F}^{fp})$. By (ref. to lemma in Section 2), the morphism of $\infty$-sites $$\nu_EF_{*,*}(-):(\Syn_E)_{\nu F}^{fp}\to\Comod_{\nu_EF_{*,*}\nu_EF}^{fp}$$ is one which reflects coverings and admits a common envelope. By \cite[Rem. 2.50]{Pst22}, this induces an adjoint equivalence $$Sh_{\Sigma}^{\mathrm{Set}}((\Syn_E)_{\nu F}^{fp})\rightleftarrows Sh_{\Sigma}^{\mathrm{Set}}(\Comod_{\nu_EF_{*,*}\nu_EF}^{fp})\,.$$
  The bigraded Hopf algebroid $(\nu_EF_{*,*},\nu_EF_{*,*}\nu_EF)$ is Adams, in the sense of \cite[Def. 3.1]{Pst22}, by (Lemma in Section 2 which proves that it's Adams). By a bigraded version of \cite[2.1.12]{GH05}, \cite[Thm. 3.2]{Pst22} there is an equivalence
  $$
  \Comod_{\nu_EF_{*,*}\nu_EF}\simeq Sh_{\Sigma}^{\mathrm{Set}}(\Comod_{\nu_EF_{*,*}\nu_EF}^{fp}),
  $$
  and the result follows.
  \end{proof}
  
  Now we work towards identifying the homotopy objects $\pi_k^{F,\heartsuit}X$ in terms of $\nu^2F$-homology.
  
  \begin{lemma}
  \label{F_dual_tstruct_lemma}
      For $X\in\Syn_{E,F}$, the graded components of the $\nu_EF_{*,*}\nu_EF$-comodule $\pi_k^{F,\heartsuit}X$ are described by
      $$
  (\pi_k^{F,\heartsuit}X)_{l,m} \cong \colim_\alpha \pi_kX(\Sigma^{l,m}D\nu_E F_\alpha),
      $$
      where $F\simeq \colim_\alpha F_\alpha$ is a presentation of $F$ as a filtered colimit of $F$-finite projective spectra.
  \end{lemma}
  
  \begin{proof}
      This is essentially a bigraded version of \cite[Lemma 4.17]{Pst22} and the proof is similar to the proof of that lemma. By \cite[Thm. 2.58]{Pst22}, the sheaf $\pi_k^{F,\heartsuit}X\in Sh_{\Sigma}^{\mathrm{Set}}((\Syn_E)_{\nu F}^{fp})$ is representable by some comodule $N$; i.e. $$(\pi_k^{F,\heartsuit}X)(-)\simeq \Hom_{\nu_EF_{*,*}\nu_EF}(\nu_EF_{*,*}(-),N).$$
      Now notice that $\nu_EF_{*,*}\nu_EF\simeq \colim_\alpha \nu_EF_{*,*}\nu_EF_\alpha$, since $\nu_E$ commutes with filtered colimits, and $E_*(D\nu_EF_\alpha)\cong\Hom_{\nu_EF_{*,*}\nu_EF}(\nu_EF_{*,*}\nu_EF_\alpha,\nu_EF_{*,*})$. Then by \cite[Lemma 3.3]{Pst22}, as a bigraded abelian group
      $$
  N_{l,m}\cong \colim_\alpha \pi_k^{F,\heartsuit}X(\Sigma^{l,m}D\nu_EF_\alpha).
      $$
      By a bigraded version of \cite[Lemma 3.25]{Pst22},
      $$
  \colim_{\alpha}\pi_k^{F,\heartsuit}X(\Sigma^{l,m}D\nu_EF_\alpha)\cong\colim_{\alpha} \pi_kX(\Sigma^{l,m}D\nu_EF_\alpha),
      $$
      which completes the proof.
  \end{proof}
  
  \begin{theorem}
  \label{F_homol_tstruct_theorem}
      For $X\in\Syn_{E,F}$, there is an isomorphism
      \[
          (\pi_k^{F,\heartsuit}X)_{l,m}\cong\nu^2F_{k+l,m,l}X,
      \]
      In effect, $\pi_k^{F,\heartsuit}X$ captures the $F$-Chow-degree $\ell$ part of the $\nu^2F$ homology of $X$.
  \end{theorem}
  
  \begin{proof}
      Again, this is a similar proof to \cite[Thm. 4.18]{Pst22}. We have that
      \begin{equation*}
       \begin{aligned}
        \nu^2F_{k+l,m,l}X&\cong [\bS^{k+l,m,l},\nu^2F\otimes X] \\
        &\cong \colim_\alpha[\Sigma^k\mu_F(\bS^{l,m}_E),\nu^2F_\alpha\otimes X] \\
        &\cong \colim_\alpha [\Sigma^k\mu_F(\Sigma^{l,m}D\nu_EF_\alpha),X] \\
        &\cong \colim_\alpha \pi_kX(\Sigma^{l,m}D\nu_EF_\alpha) \\
        &\cong (\pi_k^{F,\heartsuit}X)_{l,m}.
      \end{aligned}   
      \end{equation*}
   The first isomorphism is by definition, the second isomorphism follows from (definition from Section 2 about trigraded spheres) and equivalence $\nu^2F\simeq\colim_\alpha \nu^2F_\alpha$, the fourth isomorphism follows from (lemma from Section 2 which shows that $map(\mu_FP,X)\simeq \Omega^\infty(X(P))$ for $P\in(\Syn_E)_{\nu F}^{fp}$), and the fifth isomorphism follows from Lemma~\ref{F_dual_tstruct_lemma}.   
  \end{proof}
  
  As a corollary, we get the following analog of \cite[Cor. 4.19]{Pst22}:
  
  \begin{corollary}
  \label{F_chow_degree_cor}
  A bisynthetic spectrum $X\in\Syn_{E,F}$ is in $(\Syn_{E,F})_{\geq 0}^F$ if and only if $\nu^2F_{k,w,v}X =0$ for Chow degree $k-v<0$. 
  \end{corollary}
  
  \begin{proof}
      In this $t$-structure, $X\in\Syn_{E,F}$ is in $(\Syn_{E,F})_{\geq 0}^F$ if and only if $\pi_k^{F,\heartsuit}X$ vanishes for $k<0$. By Theorem~\ref{F_homol_tstruct_theorem}, this happens exactly when $k-v<0$.
  \end{proof}
  
As a consequence, we see that the $\nu F$-synthetic analog of an $E$-synthetic spectrum $Y$ is always connective.
  
  \begin{corollary}
      If $Y\in\Syn_E$, then $\mu_F Y\in (\Syn_{E,F})_{\geq 0}^F$.
  \end{corollary}
  
  \begin{proof}
      Consider the homology calculation
  \begin{equation*}
      \begin{aligned}
          \nu^2F_{t,w,v}\mu_FY &\cong \mu_F(\nu F\otimes Y)_{t,w,v} \\
          &\cong \nu F_{t,w} Y[\lambda]\, ,
      \end{aligned}
  \end{equation*}
  where $\nu F_{t,w} Y$ lives in tridegree $(t,w,t)$. The first isomorphism follows from (lemma in Section 2 about when $\mu_F$ is symmetric monoidal) and the second isomorphism follows (lemma in Section 2 about homotopy of $\nu F$-module). The result then follows from Corollary~\ref{F_chow_degree_cor}.
  \end{proof}
  
  This means that for the $\nu F$-synthetic analog of an $E$-synthetic spectrum $Y$, the tensor product $\mu_FY\otimes C\lambda$ lives in the heart $\Syn_{E,F}^{F,\heartsuit}$.
  
  \begin{corollary}
  If $Y\in\Syn_E$, then $\Sigma^{0,0,-1}\mu_FY\simeq \tau_{\geq 1}^F(\mu_FY)$ and $\mu_F Y\otimes C\lambda\simeq \tau_{\leq 0}^F(\mu_FY)$. In particular, $\mu_FY\otimes C\lambda\in \Syn_{E,F}^{F,\heartsuit}$.  
  \end{corollary}
  
  \begin{proof}
     Again, the proof is similar to the proof of \cite[Lemma 4.29]{Pst22}. Consider the cofiber sequence
     $$
  \Sigma^{0,0,-1}\mu_FY\xrightarrow{\lambda}\mu_FY\to \mu_FY\otimes C\lambda
     $$
     By Corollary~\ref{F_chow_degree_cor}, it's clear that $\Sigma^{0,0,-1}\mu_FY$ is 1-connective. By using the definition of $\mu_F$ and the colimit-comparison definition of $\lambda$, it follows that $\mu_FY\otimes C\lambda$ lives in $(\Syn_{E,F})_{\leq 0}^F$. The result then follows.
  \end{proof}
  
  \begin{remark}
  Similar to $\Syn_E$, we see that $\mu_FY\otimes C\lambda$ is lives in an algebraic category; namely the category of $\nu_EF_{*,*}\nu_E F$-comodules. In Section 4, we will show that, in fact, $\mu_FY\otimes C\lambda$ can be identified with the comodule $\nu_{E}F_{*,*}Y$ and there is an embedding $\Mod_{C\lambda}(\Syn_{E,F})\hookrightarrow \Stable_{\nu_EF_{*,*}\nu_EF}$ of $C\lambda$-modules into the stable comodule category associated to the bigraded Hopf algebroid $(\nu_EF_{*,*},\nu_EF_{*,*}\nu_EF)$.    
  \end{remark}


\subsection{The E-t-structure}

In constructing the $E$-t-structure we reach an immediate impasse: we do not have access now to the formality of a sheaf t-structure. The secondary definition of the $F$-t-structure provides the solution; we will take it as our primary definition for the $E$-t-structure.

\begin{definition}
  A bisynthetic spectrum is said to be $E$-connective if $\nu^2E_{***}X$ is concentrated in positive $E$-chow degree.
\end{definition}

\begin{lemma}
  The $E$-chow-connectives form the connective part of a t-structure on $\Syn_{E,F}$.
\end{lemma}

\begin{proof}
  It suffices by todo to show that these objects are closed under colimits and extensions. They are clearly closed under finite direct sums, and the compactness of the trigraded spheres shows that this extends to infinite coproducts. Finally, long exact sequence arguments show that they are closed under cofibers and extensions as desired.
\end{proof}

  
\section{$\tau$-local Bisynthetic Spectra}

The main result we are to prove in this section is the following:

\begin{theorem}
  \label{thm:taulocal}
  There is a symmetric monoidal equivalence of categories $\Syn_F\simeq \tau^{-1}\Syn_{E,F}$.
\end{theorem}

To prove this, we will show that both are equivalent to an intermediate category of spherical sheaves of spectra on a certain site. Our approach is heavily inspired by the original comparison between cellular motives and $\MU$-synthetic spectra of \cite{Pst22}.

\begin{definition}
  The subcategory $(\tau^{-1}\Syn_{E,F})^\fp$ of $\tau$-local finite projectives is defined to be those bisynthetic spectra $X\in \Syn_{E,F}$ satisfying
  \begin{itemize}
    \item $X$ is compact in $\tau^{-1}\Syn_{E,F}$
    \item and $\pi_{***}(X\otimes \nu^2F)$ is finitely generated by generators in $E$-Chow degree $0$ and projective over $\pi_{***}\nu^2F$. 
  \end{itemize}
\end{definition}

Our proof of Theorem \ref{thm:taulocal} will proceed first by showing that the site described above forms recovery pair $\tau^{-1}\Syn_{E,F}$ and then by comparing the site in this pair to $\Sp_F^\fp$ via the double realization $\Re^2$. We begin by establishing lemmas relating the two sites.

\begin{lemma}
  There is an equivalence of sites $\Sp_F^\fp\simeq (\tau^{-1}\Syn_E)^{\fp}_{\tau^{-1}\nu_E F}$ where the right-hand-side has our usual meaning: $\tau$-local $E$-synthetic spectra which are compact (in $\tau^{-1}\Syn_E$) and have finitely generated and projective $\tau^{-1}\nu_EF$-homology.
\end{lemma}

\begin{proof}
  This is immediate from the symmetric monoidal equivalence $\tau^{-1}\Syn_{E}\simeq \Sp$ as the conditions defining the sites on both sides match up across this equivalence.
\end{proof}

\begin{lemma}
  If $X\in (\tau^{-1}\Syn_{E,F})^\fp$ then $\Re^F(X)\in \Syn_E$ is again $\tau$-local and compact as an object of $\tau^{-1}\Syn_E$.
\end{lemma}

\begin{proof}
  The first claim follows from that fact that $\mu_F$ is a section of $\Re^{F}$. The second follows from TODO.
\end{proof}

\begin{lemma}
  If $X\in \tau^{-1}\Syn_{E,F}$ has finitely generated and projective $\nu^2F$-homology, then $\Re^F(X)$ has finitely generated and projective $\nu F$-homology. If $f:X\to Y$ is a $\nu^2F$-homology epimorphism, then $\Re(f)$ is a $\nu_E F$-homology epimorphism. Moreover, if $X,Y\in \tau^{-1}\Syn_{E,F}$ have $\nu^2F$-homology generated in $F$-Chow degree $0$ then if $f:X\to Y$ is a $\nu_EF$-homology isomorphism after realization, it must have been a $\nu^2F$-homology isomorphism to begin with.
\end{lemma}

\begin{proof}
  Because $\Re^F$ corresponds to inverting $\lambda$, up to regrading this is just the fact that inverting a class preserves projectives, finite generation, and epimorphisms (over the localized ring). The final claim follows from that fact that in $F$-Chow degree $0$ we have that $\nu^2F_{***}X\cong \nu_E F_{**}X$ from todo so that if the map is an epimorphism after realization then it is an epimorphism in the generating degrees of $\nu^2F$-homology.
\end{proof}

As a result of our corollaries we have shown that realization restricts as
\[
\Theta:(\tau^{-1}\Syn_{E,F})^\fp \to (\tau^{-1}\Syn_E)^\fp_{\tau^{-1}\nu F}\simeq \Sp^\fp_F
\]
providing a morphsim of sites into the site defining $\Syn_F$. The final claim of Lemma todo can then be rephrased as stating that this morphism of sites reflects covers. This induces an adjunction:
\[
\Theta^*: \Sh^\Sp_\Sigma((\tau^{-1}\Syn_{E,F})^\fp) \leftrightarrows \Syn_F:\Theta_*.
\]
whose left adjoint is symmetric monoidal and whose right adjoint is lax monoidal, preserves colimits, and is t-exact by \cite[Prop. 2.22, Rem. 2.23]{Pst22}. We will now show that this adjunction is an equivalence. The functor $\Theta_*$ is easy to describe, it is given by precomposition. The left adjoint is characterized as the unique colimit preserving functor such that $\Theta^*(\Sigma^\infty_+y(X))\simeq \Sigma_+^\infty y(\Theta(X))$.

\begin{proposition}
  The pair $(\Theta^*,\Theta_*)$ form an adjoint equivalence which is symmetric monoidal by construction.
\end{proposition}

\begin{proof}
  We first show that $\Theta^*$ is essentially surjective. The category $\Syn_F$ is generated by the objects $\nu_F P$ where $P\in \Sp^\fp_F$. But we know that for any such $P$ we have $\Theta^*\Sigma^\infty_+(y(\nu^2P))\simeq $ TODO: Easy I think if cellular.
\end{proof}

\begin{proposition}
  The site $(\tau^{-1}\Syn_{E,F})^\fp$ recovers $\tau^{-1}\Syn_{E,F}$, i.e. there is a symmetric monoidal equivalence 
  \[
    \Sh_\Sigma^\Sp((\tau^{-1}\Syn_{E,F})^\fp) \simeq \tau^{-1}\Syn_{E,F}
  \]
  induced by the Yoneda embedding.
\end{proposition}

\begin{proof}
  By our general results in todo it suffices to show that the objects of $X\in (\tau^{-1}\Syn_{E,F})^\fp$ generate under colimits and that $\Sigma^\infty_+y(X)$ is connective in the sheaf t-structure for all such $X$.
\end{proof}

\newpage

\begin{lemma}
  Suppose that $X\in \Syn_{E,F}$ is $\tau$-local. Then there is an isomorphism
  \[
    \pi_{t,w,v}X \cong \pi_{t}\Re^2(X)
  \]
  whenever $t-v\geq 0$, i.e., in positive $F$-chow degree.
\end{lemma}

\begin{proof}
  It follows from todo that we have an isomorphism $\pi_{t,w,v}X\cong \pi_{t,w}\Re^E(X)$ in positive $F$-Chow degree.
\end{proof}

\begin{lemma}
  \label{lem:taulocalfptoffp}
  The full realization functor $\Re^2:\Syn_{E,F}\to \Sp$ sends $(\tau^{-1}\Syn_{E,F})^\fp$ onto $\Sp^\fp_F$.
\end{lemma}

\begin{proof}
  Suppose $X\in (\tau^{-1}\Syn_{E,F})^\fp$. First recall that the functor $\Re^2$ will preserve compact objects so it suffices to check that
  \[
  \pi_*\Re^2(X)\otimes F \cong \pi_*\Re^2(X\otimes \nu^2 F)
  \]
  is finitely generated over $\pi_*F$ and projective. But up to regrading $\Re^2$ factors as first inverting $\lambda$ and then inverting $\tau$. But inverting classes will preserve projectivity and finite generation. To see that this restriction is essentially surjective we note that the composition
  \[
  \Sp \xrightarrow{\nu^2} \Syn_{E,F}\xrightarrow{\tau^{-1}}\tau^{-1}\Syn_{E,F}
  \]
  provides a section. %todo: show this has the right codomain
\end{proof}

\begin{lemma}
  The restricted $\Re^2$ functor of Lemma \ref{lem:taulocalfptoffp} is a morphism of sites which reflects covers.
\end{lemma}

\begin{proof}
  A map $\pi_{***}(X\otimes \nu^2 F)\to \pi_{***}(Y\otimes \nu^2 F)$ for $X,Y\in (\tau^{-1}\Syn_{E,F})^\fp$ will be an epimorphism if and only if it is an epimorphism in Chow degree $0$ so that the result follows from Lemma \ref{todo}. 
\end{proof}



\begin{proposition}
  The pair $((\tau^{-1}\Syn_{E,F})^\fp, \tau^{-1}\Syn_{E,F})$ is a recovery pair, i.e., there is a symmetric monoidal equivalence 
  \[
    \Sh_{\Sigma}^\Sp((\tau^{-1}\Syn_{E,F})^\fp)\simeq \tau^{-1}\Syn_{E,F}
  \]
  induced by the Yoneda embedding.
\end{proposition}

\begin{proof}
  By our general results in todo, we see that is stuffices to show that 
\end{proof}

\begin{definition}
  We define the functor $\nu_E:\Syn_{F}\to \Syn_{E,F}$ to be the composite
  \[
  \Syn_{F}\simeq \tau^{-1}\Syn_{E,F} \hookrightarrow \Syn_{E,F}\xrightarrow{\tau^E_{\geq 0}}\Syn_{E,F}
  \]
  and we will refer to this $\nu_E$ is as the $E$-synthetic analog (of an $F$-synthetic spectrum).
\end{definition}

\begin{lemma}
  The functor $\nu_E$ above is lax monoidal and preserves finite direct sums and filtered colimits.
\end{lemma}

\begin{proof}
    The functor is defined as a composite of functors which have the desired properties.
\end{proof}

\end{document}