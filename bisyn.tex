\documentclass[10pt]{amsart}
\usepackage[letterpaper,margin=1in,footskip=0.25in]{geometry}


%\usepackage{garamondlibre}
\usepackage{times}
%\usepackage{CormorantGaramond}
%\usepackage{baskervald}
\usepackage{microtype}
\usepackage{eucal}
\usepackage{setspace}
\usepackage{mathrsfs}
\usepackage{tikz-cd}

\usepackage[
backend=biber,
style=alphabetic,
sorting=nyt, maxbibnames=20, maxalphanames=99
]{biblatex}
%\usepackage{pdfpages}

\addbibresource{refs.bib}


\usepackage{amsmath,amssymb,amsthm}
\usepackage{mathtools}
\usepackage{mathabx}
\makeatletter
  \newcommand{\supsize}{%
    \expandafter\ifx\csname S@\f@size\endcsname\relax
      \calculate@math@sizes
    \fi
    \csname S@\f@size\endcsname
    \fontsize\sf@size\z@\selectfont
  }
  \DeclareRobustCommand{\tsup}[1]{%
    \leavevmode\raise.9ex\hbox{\supsize #1}%
  }
  \DeclareTextSymbolDefault{\textprimechar}{OMS}
  \DeclareTextSymbol{\textprimechar}{OMS}{48}
  \DeclareRobustCommand{\tprime}{\tsup{\textprimechar}}
  \ProvideTextCommandDefault{\cprime}{\tprime}
\makeatother


\usepackage{enumitem}
\setlist{noitemsep}

\usepackage[pdfusetitle,colorlinks]{hyperref}
\hypersetup{allcolors=blue}
\usepackage[capitalise,noabbrev]{cleveref}
\crefformat{equation}{\ensuremath{(#2#1#3)}}
\crefmultiformat{equation}{\ensuremath{(#2#1#3)}}{ and~\ensuremath{(#2#1#3)}}{, \ensuremath{(#2#1#3)}}{, and~\ensuremath{(#2#1#3)}}

\theoremstyle{definition}

\numberwithin{figure}{section}
\numberwithin{equation}{section}

\newtheorem{theorem}[figure]{Theorem}
\newtheorem{lemma}[figure]{Lemma}
\newtheorem{construction}[figure]{Construction}

\newtheorem{corollary}[figure]{Corollary}
\newtheorem{proposition}[figure]{Proposition}
\newtheorem{definition}[figure]{Definition}
\newtheorem{notation}[figure]{Notation}
\newtheorem{exercise}[figure]{Exercise}
\newtheorem{remark}[figure]{Remark}
\newtheorem{example}[figure]{Example}
\newtheorem{conjecture}[figure]{Conjecture}

\newtheorem{thm}[figure]{Theorem}
\newtheorem{var}[figure]{Variant}
\newtheorem{lem}[figure]{Lemma}
\newtheorem{cons}[figure]{Construction}

\newtheorem{cor}[figure]{Corollary}
\newtheorem{prop}[figure]{Proposition}
\newtheorem{defn}[figure]{Definition}
\newtheorem{notn}[figure]{Notation}
\newtheorem{rem}[figure]{Remark}

\newcommand{\op}{\mathrm{op}}

\newcommand{\poly}[1]{{#1}[x_1,\ldots,x_n]}
\newcommand{\cA}{\mathcal{A}}
\newcommand{\cB}{\mathcal{B}}
\newcommand{\cC}{\mathcal{C}}
\newcommand{\cD}{\mathcal{D}}
\newcommand{\cE}{\mathcal{E}}
\newcommand{\cF}{\mathcal{F}}
\newcommand{\cG}{\mathcal{G}}
\newcommand{\cH}{\mathcal{H}}
\newcommand{\cI}{\mathcal{I}}
\newcommand{\cJ}{\mathcal{J}}
\newcommand{\cK}{\mathcal{K}}
\newcommand{\cL}{\mathcal{L}}
\newcommand{\cM}{\mathcal{M}}
\newcommand{\cN}{\mathcal{N}}
\newcommand{\cO}{\mathcal{O}}
\newcommand{\cP}{\mathcal{P}}
\newcommand{\cQ}{\mathcal{Q}}
\newcommand{\cR}{\mathcal{R}}
\newcommand{\cS}{\mathcal{S}}
\newcommand{\cT}{\mathcal{T}}
\newcommand{\cU}{\mathcal{U}}
\newcommand{\cV}{\mathcal{V}}
\newcommand{\cW}{\mathcal{W}}
\newcommand{\cX}{\mathcal{X}}
\newcommand{\cY}{\mathcal{Y}}
\newcommand{\cZ}{\mathcal{Z}}
\newcommand{\pp}{\mathbf{p}}
\newcommand{\mm}{\mathbf{m}}
\newcommand{\mbfa}{\mathbf{A}}
\newcommand{\sX}{\mathscr{X}}
\newcommand{\sY}{\mathscr{Y}}
\newcommand{\sch}{\operatorname{Sch}}
\newcommand{\GG}{\mathbf{G}}
\newcommand{\BB}{\mathbf{B}}
\newcommand{\pic}{\operatorname{Pic}}

\newcommand{\MU}{\mathrm{MU}}
\newcommand{\BP}{\mathrm{BP}}
\newcommand{\SU}{\mathrm{SU}}
\newcommand{\BU}{\mathrm{BU}}


\newcommand{\gr}{\mathrm{gr}}
\newcommand{\fil}{\mathrm{fil}}

\newcommand{\BGL}{\mathrm{BGL}}
\newcommand{\Thick}{\mathrm{Thick}}
\newcommand{\Th}{\mathrm{Th}}
\newcommand{\Ext}{\operatorname{Ext}}

\newcommand{\spaces}{\mathcal{S}}
\newcommand{\one}{\mathrm{1}}
\newcommand{\ord}{\mathrm{ord}}
\newcommand{\wt}{\mathrm{wt}}
\newcommand{\unfinished}{\textcolor{red}{INCOMPLETE }}
\newcommand{\done}{\textcolor{green}{DONE }}

\newcommand{\anss}{{}^{\mathrm{an}}\mathrm{E}}
\newcommand{\genanss}{{}^{\mathrm{F}}\mathrm{E}}

\newcommand{\mayss}{{}^{\mathrm{may}}\mathrm{E}}
\newcommand{\vlines}{\mathrm{VL}}
\newcommand{\page}{\mathrm{page}}
\newcommand{\intercept}{\mathrm{incpt}}

\newcommand{\kos}[1]{\mathbf{K}_\bullet(\mathbf{#1})}



\newcommand{\ass}{\operatorname{Ass}}
\newcommand{\spec}{\operatorname{Spec}}
\newtheoremstyle{cited}{.5\baselineskip\@plus.2\baselineskip\@minus.2\baselineskip}{.5\baselineskip\@plus.2\baselineskip\@minus.2\baselineskip}{\itshape}{}{\bfseries}{\bfseries .}{5pt plus 1pt minus 1pt}{\thmname{#1}\thmnumber{ #2}\thmnote{ \normalfont #3}}
\theoremstyle{cited}
\newtheorem{citedthm}[figure]{Theorem}
\newtheorem{citedprop}[figure]{Proposition}
\newtheorem{citedcor}[figure]{Corollary}

%Operators
\DeclareMathOperator{\Aut}{Aut}
\newcommand{\isom}{\operatorname{Isom}}
\newcommand{\sets}{\operatorname{Sets}}
\newcommand{\htensor}{\hat{\otimes}}

%Objects
\newcommand{\tensor}{\otimes}
\newcommand{\into}{\hookrightarrow}
\newcommand{\aff}{\mathbb{A}}
\newcommand{\mf}[1]{\mathbf{#1}}
\newcommand{\ess}{\operatorname{Es}/S}

\newcommand{\bZ}{\mathbb{Z}}
\newcommand{\bN}{\mathbb{N}}
\newcommand{\bS}{\mathbb{S}}
\newcommand{\bD}{\mathbb{D}}
\newcommand{\bE}{\mathbb{E}}
\newcommand{\bF}{\mathbb{F}}

%Maps
\newcommand{\qcoh}[1]{\operatorname{Qcoh}(#1)}
\newcommand{\id}{\mathrm{id}}
\newcommand{\der}[3]{\textrm{Der}_{#1}(#2,#3)}

\newcommand{\affs}{\operatorname{Aff}/S}
\newcommand{\ev}{\operatorname{ev}}
\newcommand{\alg}{\operatorname{Alg}}
\newcommand{\cech}{\operatorname{Cech}}
\newcommand{\tot}{\operatorname{Tot}}
\newcommand{\Fun}{\operatorname{Fun}}
\newcommand{\fun}{\Fun}

\newcommand{\CAlg}{\operatorname{CAlg}}
\newcommand{\PSigma}{\operatorname{P}_{\Sigma}}

\newcommand{\sCAlg}{\operatorname{sCAlg}}
\renewcommand{\poly}{\mathrm{poly}}
\newcommand{\colim}{\operatorname{colim}}
\newcommand{\cof}{\operatorname{cof}}

\newcommand{\HH}{\operatorname{HH}}
\newcommand{\THH}{\operatorname{THH}}

\newcommand{\Fil}{\mathrm{fil}}
%Arrows
\newcommand{\longisoto}{\overset{\sim}{\longrightarrow}}
\newcommand{\spm}{\operatorname{Spm}}

\newcommand{\B}{\mathbf{B}}
\newcommand{\Nm}{\operatorname{Nm}}

\newcommand{\PrL}{\mathrm{Pr}^{\mathrm{L}}}
\newcommand{\Ab}{\mathrm{Ab}}
\newcommand{\Sp}{{\mathcal{S}\mathrm{p}}}
\newcommand{\Mod}{\mathrm{Mod}}

\newcommand{\HE}{\mathcal{H}\mathrm{Env}}
\newcommand{\HESt}{\HE^{\Sp}}
\newcommand{\Map}{\mathrm{Map}}


\newcommand{\Cond}{\mathrm{Cond}}
\newcommand{\Condbar}{\mathbf{Cond}}
\newcommand{\Cat}{\mathrm{Cat}_\infty}
\newcommand{\proet}{\text{pro-et}}
\newcommand{\pep}{\ast_{\proet}}
\newcommand{\Sh}{\mathrm{Sh}}
\newcommand{\KCond}{\mathrm{K}^{\text{c}}}
\newcommand{\LCA}{\mathrm{LCA}}
\newcommand{\bA}{\mathbb{A}}
\newcommand{\Perf}{\mathrm{Perf}}
\newcommand{\PerfCat}{\mathrm{PerfCat}}
\newcommand{\lc}{\mathrm{lc}}
\newcommand{\K}{\mathrm{K}}
\newcommand{\LCAbar}{\mathbf{LCA}}

\newcommand{\Condf}[1]{\mathbf{#1}}

\newcommand{\bfC}{\mathbf{C}}
\newcommand{\bfD}{\mathbf{D}}
\newcommand{\CondFun}{\Condf{Fun}}
\newcommand{\CondCat}{\Condf{Cat}}
\newcommand{\cond}{\text{c}}
\newcommand{\CondD}{\Condf{D}}
\newcommand{\CondPerf}{\Condf{Perf}}
\newcommand{\CondLCA}{\Condf{LCA}}
\newcommand{\CondCone}{\Condf{cone}}
\newcommand{\CondK}{\Condf{K}}
\newcommand{\Condpi}{\Condf{\varpi}}

\newcommand{\Gal}{\mathrm{Gal}}
\newcommand{\dKSel}{\mathrm{dK}^{\mathrm{Sel}}}


\newcommand{\maxnote}[1]{\textcolor{blue}{#1}}
\newcommand{\peternote}[1]{\textcolor{orange}{#1}}

\newcommand{\cb}{\mathrm{cb}}

\newcommand{\fp}{\mathrm{fp}}
\newcommand{\Hom}{\mathrm{Hom}}
\newcommand{\Syn}{\mathcal{S}\mathrm{yn}}
\newcommand{\Bisyn}{\mathcal{B}\mathrm{isyn}}
\newcommand{\Stable}{\mathcal{S}\mathrm{table}}
\newcommand{\loc}{\mathrm{loc}}

\begin{document}

\title{Bisynthetic Spectra}
\author{Maxwell Johnson and Peter Marek}

\maketitle

\section{Recollections on Synthetic Spectra}

\begin{definition}
  Given $E$ any spectrum and $X$ a finite spectrum, $X$ is said to be $E$-finite projective if $E_*X$ is finitely generated and projective over $E_*$. We denote the full subcategory of spectra spanned by such $\Sp^\fp_E$. A map $X\to Y$ in $\Sp^\fp_E$ is said to be a cover if it is an epimorphism after taking $E$-homology.
\end{definition}

\begin{definition}
  A site $\cC$ which is additive is said to be in addition excellent if it is equipped with a symmetric monoidal structure such that all objects admit duals and such that the functors $-\otimes c$ preserve covers for all $c\in \cC$.
\end{definition}

\begin{lemma}
  The category $\Sp^\fp_E$ is additive and acquires the structure of a site with the covering families given by singletons of $E$-epimorphisms as above. Equipped with the smash product of spectra, it is excellent.
\end{lemma}


\begin{definition}
  A spectrum $E$ is said to be Adams-type if there exists a filtered diagram $X_\alpha$ such that each $X_\alpha$ is in $\Sp^\fp_E$ and such that the natural map $E^*X_\alpha\to \Hom_{E_*}(E_*X_\alpha, E_*)$ is an isomorphism.
\end{definition}

\begin{definition}
  A presheaf $F:\cC^\op \to \cD$ on a category with finite coproducts is said to be spherical if for all $c,c'\in \cC$ the natural map $F(c\amalg c')\to F(c)\times F(c')$ is an equivalence, i.e., if $F$ preserves finite products as a covariant functor on $\cC^\op$.
\end{definition}

\begin{definition}
  The category $\Syn_E$ of synthetic spectra is the category of spherical presheaves of spectra on the excellent site $\Sp^\fp_E$.
\end{definition}



\section{The Bisynthetic Model}

\subsection{Synthetic finite projectives}

\begin{definition}
  Given $F,X\in \Syn_E$ we say that $X$ is $F$-finite projective if it compact as a synthetic spectrum and if $F_{*,*}X:=\pi_{*,*}(F\otimes X)$ is a finitely generated projective module over $F_{*,*}:=\pi_{*,*}F$. We denote the full subcategory of $F$-finite projectives $(\Syn_E)_F^\fp$. A map $X\to Y$ of $F$-finite projectives is said to be a cover if it is an epimorphism after applying taking $F$-homology.
\end{definition}

\begin{lemma}
  The category $\Syn_F^\fp$ is an additive site when equipped with the covering families consisting of single $F_{*,*}$-epimorphisms.
\end{lemma}

\begin{proof}
  The proof is identical to \cite[Lemma 3.22]{piotr}.
\end{proof}

\begin{lemma}
  Equipped with the tensor product of synthetic spectra, $(\Syn_E)_F^\fp$ is excellent.
\end{lemma}

\begin{proof}
  all these proofs look like the one in piotrs paper goes through identically, but I am going to come back to that later.
\end{proof}

\section{Special and Generic fibers over $\lambda$ and $\tau$}

\subsection{The $\lambda$-generic fiber}

\begin{theorem}
  The subcategory of $\lambda$-local objects in $\Bisyn$ is canonically equivalent to $\Syn_E$.
\end{theorem}

\subsection{The $\lambda$-special fiber}

\begin{theorem}
  The category $\Mod(\Bisyn, \bS/\lambda)$ is a full subcategory of $\Stable(\nu F_{*,*}\nu F)$ which is an equivalence if (???). Restricted to the image of $\nu_F$, this equivalence takes an $E$-synthtetic spectrum to its $\nu F$-homology.
\end{theorem}

\subsection{The $\tau$-generic fiber}

\begin{notation}
  We will write $(\Syn_E)^{\tau-\loc}_F$ for the site $(\Syn_E)_{\tau^{-1}F}^\fp$.
\end{notation}

\begin{lemma}
  The functor $\tau^{-1}$ induces a morphism of excellent sites $(\Syn_E)^\fp_{F}\to (\Syn_E)^{\tau-\loc}_F$.
\end{lemma}

\begin{proof}
  Because the category of $\tau$-local synthetic spectra is a smashing localization, inverting $\tau$ preserves compact objects. Then note that there is an equivalence $\tau^{-1}F\otimes \tau^{-1}X\simeq \tau^{-1}(F\otimes X)$, so that we can compute:
  \[
  (\tau^{-1}F)_{*,*}X \cong F_{*,*}X[\tau^{-1}]  
  \]
  and if $F_{*,*}X$ is finitely generated and projective over $F_{*,*}$, then $F_{*,*}X[\tau^{-1}]$ will be finitely generated and projective over $F_{*,*}[\tau^{-1}]\cong (\tau^{-1}F)_{*,*}$ and this process will also preserve epimorphisms. Because the relevant pullbacs in both sites are computed in $\Syn_E$ they are also pushouts and the left adjoint $\tau^{-1}$ will preserve them. The symmetric monoidality of $\tau^{-1}$ shows that this morphism of sites upgrades to one of excellent sites. 
\end{proof}

\begin{lemma}
  In the induced adjunction $F:\Bisyn \to \Sh_{\Sigma}((\Syn_E)^\{\tau-\loc}_F):G$, the right adjoint $G$ is cocontinuous, $G(X)$ is $\tau$-local for all $X$, and the essential image consists of all $\tau$-local bisynthetic spectra.
\end{lemma}

\begin{proof}
  
\end{proof}

\begin{proposition}
  The subcategory of $\tau$-local objects in $\Bisyn$ is equivalent to the category of spherical sheaves on the site $(\Syn_E)^{\tau-\loc}_{\nu F}$.
\end{proposition}

\begin{theorem}
  There is an equivalence of spherical sheaves over $(\Syn_E)^{\tau-\loc}_{\nu F}$ and $\Sp^\fp_F$. As a result, the category of $\tau$-local bisynthetic spectra is equivalent to $\Syn_F$.
\end{theorem}

\subsection{The $\tau$-special fiber}

\maxnote{I have no idea what to do for this at the moment, would love any ideas.}

\end{document}
