\documentclass[10pt]{amsart}
\usepackage[letterpaper,margin=1in,footskip=0.25in]{geometry}


%\usepackage{garamondlibre}
\usepackage{times}
%\usepackage{CormorantGaramond}
%\usepackage{baskervald}
\usepackage{microtype}
\usepackage{eucal}
\usepackage{setspace}
\usepackage{mathrsfs}
\usepackage{tikz-cd}
\usepackage{todonotes}
\usepackage{soul}

\usepackage[
backend=biber,
style=alphabetic,
sorting=nyt, maxbibnames=20, maxalphanames=99
]{biblatex}
%\usepackage{pdfpages}

\addbibresource{refs.bib}


\usepackage{amsmath,amssymb,amsthm}
\usepackage{mathtools}
\usepackage{mathabx}
\makeatletter
  \newcommand{\supsize}{%
    \expandafter\ifx\csname S@\f@size\endcsname\relax
      \calculate@math@sizes
    \fi
    \csname S@\f@size\endcsname
    \fontsize\sf@size\z@\selectfont
  }
  \DeclareRobustCommand{\tsup}[1]{%
    \leavevmode\raise.9ex\hbox{\supsize #1}%
  }
  \DeclareTextSymbolDefault{\textprimechar}{OMS}
  \DeclareTextSymbol{\textprimechar}{OMS}{48}
  \DeclareRobustCommand{\tprime}{\tsup{\textprimechar}}
  \ProvideTextCommandDefault{\cprime}{\tprime}
\makeatother


\usepackage{enumitem}
\setlist{noitemsep}

\usepackage[pdfusetitle,colorlinks]{hyperref}
\hypersetup{allcolors=blue}
\usepackage[capitalise,noabbrev]{cleveref}
\crefformat{equation}{\ensuremath{(#2#1#3)}}
\crefmultiformat{equation}{\ensuremath{(#2#1#3)}}{ and~\ensuremath{(#2#1#3)}}{, \ensuremath{(#2#1#3)}}{, and~\ensuremath{(#2#1#3)}}

\theoremstyle{definition}

\numberwithin{figure}{section}
\numberwithin{equation}{section}

\newtheorem{theorem}[figure]{Theorem}
\newtheorem{lemma}[figure]{Lemma}
\newtheorem{construction}[figure]{Construction}

\newtheorem{corollary}[figure]{Corollary}
\newtheorem{proposition}[figure]{Proposition}
\newtheorem{definition}[figure]{Definition}
\newtheorem{notation}[figure]{Notation}
\newtheorem{exercise}[figure]{Exercise}
\newtheorem{remark}[figure]{Remark}
\newtheorem{example}[figure]{Example}
\newtheorem{conjecture}[figure]{Conjecture}
\newtheorem{convention}[figure]{Convention}

\newtheorem{thm}[figure]{Theorem}
\newtheorem{var}[figure]{Variant}
\newtheorem{lem}[figure]{Lemma}
\newtheorem{cons}[figure]{Construction}

\newtheorem{cor}[figure]{Corollary}
\newtheorem{prop}[figure]{Proposition}
\newtheorem{defn}[figure]{Definition}
\newtheorem{notn}[figure]{Notation}
\newtheorem{rem}[figure]{Remark}

\newcommand{\op}{\mathrm{op}}

\newcommand{\poly}[1]{{#1}[x_1,\ldots,x_n]}
\newcommand{\cA}{\mathcal{A}}
\newcommand{\cB}{\mathcal{B}}
\newcommand{\cC}{\mathcal{C}}
\newcommand{\cD}{\mathcal{D}}
\newcommand{\cE}{\mathcal{E}}
\newcommand{\cF}{\mathcal{F}}
\newcommand{\cG}{\mathcal{G}}
\newcommand{\cH}{\mathcal{H}}
\newcommand{\cI}{\mathcal{I}}
\newcommand{\cJ}{\mathcal{J}}
\newcommand{\cK}{\mathcal{K}}
\newcommand{\cL}{\mathcal{L}}
\newcommand{\cM}{\mathcal{M}}
\newcommand{\cN}{\mathcal{N}}
\newcommand{\cO}{\mathcal{O}}
\newcommand{\cP}{\mathcal{P}}
\newcommand{\cQ}{\mathcal{Q}}
\newcommand{\cR}{\mathcal{R}}
\newcommand{\cS}{\mathcal{S}}
\newcommand{\cT}{\mathcal{T}}
\newcommand{\cU}{\mathcal{U}}
\newcommand{\cV}{\mathcal{V}}
\newcommand{\cW}{\mathcal{W}}
\newcommand{\cX}{\mathcal{X}}
\newcommand{\cY}{\mathcal{Y}}
\newcommand{\cZ}{\mathcal{Z}}
\newcommand{\pp}{\mathbf{p}}
\newcommand{\mm}{\mathbf{m}}
\newcommand{\mbfa}{\mathbf{A}}
\newcommand{\sX}{\mathscr{X}}
\newcommand{\sY}{\mathscr{Y}}
\newcommand{\sch}{\operatorname{Sch}}
\newcommand{\GG}{\mathbf{G}}
\newcommand{\BB}{\mathbf{B}}
\newcommand{\pic}{\operatorname{Pic}}

\newcommand{\MU}{\mathrm{MU}}
\newcommand{\BP}{\mathrm{BP}}
\newcommand{\SU}{\mathrm{SU}}
\newcommand{\BU}{\mathrm{BU}}


\newcommand{\gr}{\mathrm{gr}}
\newcommand{\fil}{\mathrm{fil}}

\newcommand{\BGL}{\mathrm{BGL}}
\newcommand{\Thick}{\mathrm{Thick}}
\newcommand{\Th}{\mathrm{Th}}
\newcommand{\Ext}{\operatorname{Ext}}

\newcommand{\spaces}{\mathcal{S}}
\newcommand{\one}{\mathrm{1}}
\newcommand{\ord}{\mathrm{ord}}
\newcommand{\wt}{\mathrm{wt}}
\newcommand{\unfinished}{\textcolor{red}{INCOMPLETE }}
\newcommand{\done}{\textcolor{green}{DONE }}

\newcommand{\anss}{{}^{\mathrm{an}}\mathrm{E}}
\newcommand{\genanss}{{}^{\mathrm{F}}\mathrm{E}}

\newcommand{\mayss}{{}^{\mathrm{may}}\mathrm{E}}
\newcommand{\vlines}{\mathrm{VL}}
\newcommand{\page}{\mathrm{page}}
\newcommand{\intercept}{\mathrm{incpt}}

\newcommand{\kos}[1]{\mathbf{K}_\bullet(\mathbf{#1})}



\newcommand{\ass}{\operatorname{Ass}}
\newcommand{\spec}{\operatorname{Spec}}
\newtheoremstyle{cited}{.5\baselineskip\@plus.2\baselineskip\@minus.2\baselineskip}{.5\baselineskip\@plus.2\baselineskip\@minus.2\baselineskip}{\itshape}{}{\bfseries}{\bfseries .}{5pt plus 1pt minus 1pt}{\thmname{#1}\thmnumber{ #2}\thmnote{ \normalfont #3}}
\theoremstyle{cited}
\newtheorem{citedthm}[figure]{Theorem}
\newtheorem{citedprop}[figure]{Proposition}
\newtheorem{citedcor}[figure]{Corollary}

%Operators
\DeclareMathOperator{\Aut}{Aut}
\newcommand{\isom}{\operatorname{Isom}}
\newcommand{\sets}{\operatorname{Sets}}
\newcommand{\htensor}{\hat{\otimes}}

%Objects
\newcommand{\tensor}{\otimes}
\newcommand{\into}{\hookrightarrow}
\newcommand{\aff}{\mathbb{A}}
\newcommand{\mf}[1]{\mathbf{#1}}
\newcommand{\ess}{\operatorname{Es}/S}

\newcommand{\bZ}{\mathbb{Z}}
\newcommand{\bN}{\mathbb{N}}
\newcommand{\bS}{\mathbb{S}}
\newcommand{\bD}{\mathbb{D}}
\newcommand{\bE}{\mathbb{E}}
\newcommand{\bF}{\mathbb{F}}

%Maps
\newcommand{\qcoh}[1]{\operatorname{Qcoh}(#1)}
\newcommand{\id}{\mathrm{id}}
\newcommand{\der}[3]{\textrm{Der}_{#1}(#2,#3)}

\newcommand{\affs}{\operatorname{Aff}/S}
\newcommand{\ev}{\operatorname{ev}}
\newcommand{\alg}{\operatorname{Alg}}
\newcommand{\cech}{\operatorname{Cech}}
\newcommand{\tot}{\operatorname{Tot}}
\newcommand{\Fun}{\operatorname{Fun}}
\newcommand{\fun}{\Fun}
\newcommand{\Alg}{\operatorname{Alg}}
\newcommand{\CAlg}{\operatorname{CAlg}}
\newcommand{\PSigma}{\operatorname{P}_{\Sigma}}

\newcommand{\sCAlg}{\operatorname{sCAlg}}
\renewcommand{\poly}{\mathrm{poly}}
\newcommand{\colim}{\operatorname{colim}}
\newcommand{\cof}{\operatorname{cof}}

\newcommand{\HH}{\operatorname{HH}}
\newcommand{\THH}{\operatorname{THH}}

\newcommand{\Fil}{\mathrm{fil}}
%Arrows
\newcommand{\longisoto}{\overset{\sim}{\longrightarrow}}
\newcommand{\spm}{\operatorname{Spm}}

\newcommand{\B}{\mathbf{B}}
\newcommand{\Nm}{\operatorname{Nm}}

\newcommand{\PrL}{\mathrm{Pr}^{\mathrm{L}}}
\newcommand{\Ab}{\mathrm{Ab}}
\newcommand{\Sp}{{\mathcal{S}\mathrm{p}}}
\newcommand{\Mod}{\mathrm{Mod}}
\newcommand{\Set}{{\mathcal{S}\mathrm{et}}}


\newcommand{\HE}{\mathcal{H}\mathrm{Env}}
\newcommand{\HESt}{\HE^{\Sp}}
\newcommand{\Map}{\mathrm{Map}}


\newcommand{\Cond}{\mathrm{Cond}}
\newcommand{\Condbar}{\mathbf{Cond}}
\newcommand{\Cat}{\mathrm{Cat}_\infty}
\newcommand{\proet}{\text{pro-et}}
\newcommand{\pep}{\ast_{\proet}}
\newcommand{\Sh}{\mathrm{Sh}}
\newcommand{\KCond}{\mathrm{K}^{\text{c}}}
\newcommand{\LCA}{\mathrm{LCA}}
\newcommand{\bA}{\mathbb{A}}
\newcommand{\Perf}{\mathrm{Perf}}
\newcommand{\PerfCat}{\mathrm{PerfCat}}
\newcommand{\lc}{\mathrm{lc}}
\newcommand{\LCAbar}{\mathbf{LCA}}

\newcommand{\Condf}[1]{\mathbf{#1}}

\newcommand{\bfC}{\mathbf{C}}
\newcommand{\bfD}{\mathbf{D}}
\newcommand{\CondFun}{\Condf{Fun}}
\newcommand{\CondCat}{\Condf{Cat}}
\newcommand{\cond}{\text{c}}
\newcommand{\CondD}{\Condf{D}}
\newcommand{\CondPerf}{\Condf{Perf}}
\newcommand{\CondLCA}{\Condf{LCA}}
\newcommand{\CondCone}{\Condf{cone}}
\newcommand{\CondK}{\Condf{K}}
\newcommand{\Condpi}{\Condf{\varpi}}

\newcommand{\Gal}{\mathrm{Gal}}
\newcommand{\dKSel}{\mathrm{dK}^{\mathrm{Sel}}}


\newcommand{\maxnote}[1]{\textcolor{blue}{#1}}
\newcommand{\peternote}[1]{\textcolor{orange}{#1}}

\newcommand{\cb}{\mathrm{cb}}

\newcommand{\fp}{\mathrm{fp}}
\newcommand{\Hom}{\mathrm{Hom}}
\newcommand{\Syn}{\mathcal{S}\mathrm{yn}}
\newcommand{\Bisyn}{\mathcal{B}\mathrm{isyn}}
\newcommand{\Stable}{\mathcal{S}\mathrm{table}}
\newcommand{\loc}{\mathrm{loc}}

\renewcommand{\H}{\mathrm{H}}
\newcommand{\G}{\mathrm{G}}

\newcommand{\Comod}{\mathcal{C}\mathrm{omod}}

\newcommand{\Tor}{\mathrm{Tor}}

\renewcommand{\Re}{\mathrm{Re}}

\newcommand{\Coop}{\mathrm{Coop}}

\begin{document}

\title{Bisynthetic Spectra}
\author{Maxwell Johnson and Peter Marek}

\maketitle

(Peter) Here's a list of things we'll need here
\begin{itemize}
    \item Lemma that $\nu_EF_{*,*}(-):(\Syn_E)_{\nu F}^{fp}\to\Comod_{\nu_EF_{*,*}\nu_EF}^{fp}$ is morphism of $\infty$-sites which reflects coverings and admits a common envelope
    \item Other lemmas about other maps of $\infty$-sites? Or wait to do that later
    \item Definition of Adams-type $E$-synthetic spectrum (prolly same as spectra one)
    \item Lemma that $\nu_EF_{*,*}\nu_EF$ is Adams Hopf algebroid a la \cite[Def. 3.1]{Pst22}
    \item Lemma about when $\nu_{\nu F}$ is symmetric monoidal
    \item Lemma about homotopy of $\nu_{\nu F}$-modules being lambda free
\end{itemize}

\section{Categorical Preliminaries and Generalized Synthetic Spectra}

Throughout this section we fix a presentably symmetric monoidal stable $\infty$-category $\cC$. Our goal in this section is to generalize the original construction of synthetic spectra due to \cite{Pst22} for a broader class of homology theories on stable $\infty$-categories. We note that in some sense this has been accomplished by \cite{todo}, however, the categories produced therein are subject to technical limitations, for example, we will need the existence of a nice symmetric monoidal structure on our synthetic categories which is not constructed in loc. cit.

\subsection{Homological Contexts} In order to set up a theory of synthetic spectra, we must first specify what the necessary data is. The original theory takes as input a (nice) ring spectrum $E$, but makes substantial use of the underlying structure present in $\Sp$ and the concomitant properties of the induced homology theory $E_*$. We spell out below what we believe is a suitably general theory of homological contexts, i.e., theories which act sufficiently like functors $X\mapsto \pi_*(X\otimes E)$.


\begin{definition}
  A local grading on a category $\cD$ is an auto-equivalence $-[1]:\cD\to \cD$. A category is said to be locally graded if it has a chosen local grading.
\end{definition}

\begin{example}
  All stable $\infty$-categories are locally graded by by the formal suspension $-[1]:=\Sigma$. 
\end{example}

\begin{example}
  A graded category $\Fun(\bZ,\cD)$ where $\bZ$ is the discrete category on the integers is locally graded by the shift functor induced by $n\mapsto n\pm 1$ on $\bZ$.
\end{example}

Both examples above frequently arise as special cases of the following:

\begin{example}
  If $\cD$ is monoidal and $X$ is a Picard-object, i.e., it is $\otimes$-invertible, then the functor $-\otimes X$ forms a local grading on $\cD$.
\end{example}

\begin{definition}
  Let $\cD$ be a presentably $\bE_n$-monoidal stable $\infty$-category and let $\cA$ be an abelian 1-category equipped with a local grading. A functor $\pi_\star:\cD\to \cA$ is said to be a homotopy groups functor if it commutes with filtered colimits, is lax monoidal and additive, and if in addition it
  \begin{enumerate}
    \item sends cofibers in $\cD$ to exact sequences in $\cA$,
    \item and intertwines the local gradings $\H(\Sigma X)=H(X)[1]$ naturally.
  \end{enumerate}
\end{definition}

\begin{remark}
  An important consequence of the above definition is that any such $\H$ will send a cofiber sequence $X\to Y\to Z$ in $\cD$ to a long exact sequence in $\cA$ as rotating the cofiber in $\cD$ results in a local-grading-shift in $\cA$.
\end{remark}

The above definition is engineered not just to capture the classical examples of homotopy groups of spectra, but also the categories of (genuine) equivariant spectra, cellular motivic spectra, and cellular synthetic spectra. In general, we will want the extra flexibility of considering homotopy groups (and later homology theories) which are multigraded and have long exact sequences with respect to the formal suspension.

\begin{definition}
  \label{def:homcontext}
  A \textit{homological context} is the data of a presentably symmetric monoidal  stable $\infty$-categories $\cC$ and a presentable $\bE_n$-monoidal left adjoint $\H:\cC\to \cD$, and a homotopy groups functor $\pi_\star:\cC\to \cA$. If $\G$ is the right adjoint to $\H$ we write $\H_\star$ for the composite $\pi_\star\circ \G\circ \H$.
\end{definition}

\begin{example}
  All examples we study in this paper will be given by the following data. First we fix $\cC$ as above and assume $\cC$ has a homotopy groups functor $\pi_\star$. We then let $R\in \Alg_{\bE_n}(\cC)$ and put $\H:=-\otimes R$ valued in $\Mod(R)$ where it becomes $\bE_{n-1}$-monoidal (and preserves colimits by assumption). Then $\H_\star$ corresponds to taking the homotopy groups on the underlying $\cC$-objects after extending by $-\otimes R$.
\end{example}

\begin{definition}
  Note that by definition the composite $H_\star$ is lax monoidal, so that the image of the unit $H_\star(\one_{\cC})$, which we will refer to as the coefficient ring for $\H$ and denote by simply $H_\star$, is a ring object in the category $\cA$.
\end{definition}

The essential reason for separating the functors $\H$ and $\pi_\star$ in the definition of a homological context is to be able to make sense of the homological comonad.

\begin{definition}
  The homological comonad for $\H$ is the comonad on $\cD$ (the codomain of $\H$) induced by the adjunction between $\H$ and its right adjoint $\G$. Explicitly, this comonad is given by the functor $\H\G: \cD\to \cD$. 
\end{definition}

\begin{definition}
  The $\cC$-object of $\H$-cooperations is defined to be the object $\Coop(\H):=\H\G\H(\one_{\cC})$. The algebraic cooperations of $\H_\star$ are defined to be $\pi_\star\Coop(H)$ and we denote them by $\H_\star H$.
\end{definition}

\begin{example}
  If $\cC=\Sp$ and $E$ is an $\bE_1$-ring, then the spectral and algebraic cooperations associated to $H=-\otimes E$ are given by the spectrum $E\otimes E$ and $E_*E$ respectively.
\end{example}

Note that the object $\Coop(\H)$ is acquires both a left and right module structure over $\H(\one_{\cC})$ from the structure morphisms arising from the comonad. The lax monoidality of $\pi_\star$ then preserves these structures, so that $\H_\star \H$ acquires a left and right module structure over $\H_\star$.

\begin{definition}[Adams Flat]
  A homological context is said to be Adams flat if $\H_\star \H$ is flat as a right $\H_\star$-module.\footnote{We are interested in various notions which have been called "Adams Type" in the literature. We reserve the later term for Definition {todo}.}
\end{definition}

\begin{proposition}
  If our homological context is Adams flat, then the pair $(\H_\star, \H_\star\H)$ is a Hopf algebroid in $\cA$ and for any $X\in \cC$ we have that $\H_\star(X)$ acquires a canonical comodule structure over $\H_\star \H$.
\end{proposition}

\begin{proof}
  todo
\end{proof}

\subsection{$\H$-finite sites}

In this section we fix a homological context (Definition \ref{def:homcontext})
\[
\cC\xrightarrow{\H} \cD \xrightarrow{\pi_\star} \cA
\]
whose composition is denoted $H_\star$. To such a context we will assocaited a site $\cC_{\H}^\fp$ which will encode the relevant properties of the $\H$-Adams spectral sequence.

\begin{definition}\label{def:projectivesite}
  With notation as above, the $\H$-projective site of $\cC$, denoted $\cC^{\fp}_{\H}$, is the full subcategory of $\cC^\omega$ consisting of objects $X$ such that $\H_{\star}(X)$ is finitely generated and projective in $\cA$. The coverings in $\cC^\fp_{\H}$ are the single maps $f:X\to Y$ such that $\H(f)$ is an epimorphism, which we call $\H$-epimorphisms for short.
\end{definition}

\begin{remark}
  Note that in a symmetric monoidal abelian category $\cA$, the condition of an object $P$ being dualizable is equivalent to be finitely generated, in the sense that there is an epimorphism $\one_{\cA}^{\oplus n}\to P$, and projective.
\end{remark}

\begin{lemma}
  \label{lem:sitepullbacks}
  Let $Q,R,P\in \cC^\omega_\H$. Suppose $f:Q\to P$ is an $\H$-epimorphism and $g:R\to P$ is arbitrary. Let $X=Q\times_P R$ denote the pullback in $\cC$. Then $X$ is again in $\cC^\omega_\H$ and $X\to R$ is an $\H$-epimorphism.
\end{lemma}

\begin{proof}
  First we note that the pullback may equivalently be described via the fiber sequence
  \[
  X\to Q\oplus R \xrightarrow{f-g} P
  \]
  and as a result is compact as the fiber of a map between compact objects. Moreover, because $\H(Q)\to \H(P)$ is an $\H$-epimorphism and because $\H(Q)$ is projective, we get a lift $\tilde g:H(R)\to H(P)$ which then splits the long exact sequence. As a result, the long exact sequence breaks up into short exact sequences, and we use the 2-out-of-3 property to claim that $\H(X)$ is therefore dualizable. That $\H(X)\to \H(R)$ is an epimorphism follows from the additional splitting.
\end{proof}

\begin{definition}
  We say that a homological context $\H_\star$ is projectively monoidal if 
  the symmetric monoidal structure on $\cC$ restricts to the site $\cC^\fp_\H$ if in addition the restriction $\cC^\fp_\H\to \cA$ is a monoidal functor.
\end{definition}

\begin{remark}
  The above condition is resonably common. One way it frequently arises is the existence of a spectral sequence 
  \[
  \Tor_{\cA}(\H(X),\H(Y))\Rightarrow H(X\otimes Y)
  \]
  where the requirement that both terms be projective causes the spectral sequence to collapse at its first page.
\end{remark}

\begin{convention}\label{conv:projectivemonoidality}
  We will assume going forward that the fixed homological context above is in additional projectively monoidal.
\end{convention}

\begin{definition}[\cite{Pst22}]
  A small $\infty$-site is \textit{additive} if the coverings are provided by singletons and the underlying category is additive. 
\end{definition}

\begin{definition}[\cite{Pst22}]
  An additive $\infty$-site is said to be excellent if it is equipped with a symmetric monoidal structure in which every object has a dual and such that the functors $-\otimes P$ preserve coverings for all $P$ in the site.
\end{definition}

\begin{proposition}
  The category $\cC^\fp_\H$ is an excellent site.
\end{proposition}

\begin{proof}
  The symmetric monoidal structure is the one guaranteed by convention \ref{conv:projectivemonoidality}. Because compact objects are dualizable, it is automatic that every object has a dual in $\cC^\fp_\H$. As such, $-\otimes P$ is a right adjoint and preserves all pullbacks for all $P\in \cC^\fp_\H$. It therefore suffices to show that it takes coverings to coverings. But since the functor $H_\star$ is monoidal (again by convention), this is immediate.
\end{proof}

\subsection{$\H$-Synthetic Spectra}

Again we fix a homological context $\cC\to \cD\to \cA$ with notation as in all previous sections. Recall that a presheaf $F:\cC^\op\to \cD$ is said to be spherical if for all $X,Y\in \cC$ the natural map
\[
F(X\amalg Y)\to F(X)\times F(Y)
\]
is an equivalence. A sheaf is said to be spherical if the underlying presheaf is, and the sheafification functor when it exists sends spherical sheaves to spherical presheaves. Spherical presheaves are very well behaved when the category $\cC$ is additive. In particular, in this case we get canonical lifts to grouplike commutative monoids in $\cD$, so long as these make sense. As a result, spherical sheaves of spaces on $\cC$ lift canonically to spherical sheaves of connective spectra.

\begin{definition}\label{def:synsp}
  The category of of Synthetic Spectra with respect to the context above is the category of spherical sheaves of spectra $\Sh^\Sp_\Sigma(\cC^\omega_\H)$ on the $\H$-finite site. We will often drop much of the context data and refer to this as the category $\Syn_\H$ of $\H$-synthetic spectra.
\end{definition}

\begin{lemma}\label{lem:synispsms}
  The category $\Syn_\H$ is presentably symmetric monoidal and stable.
\end{lemma}

\begin{proof}
  Stability and presentability follow from \cite[Corollary 2.13]{Pst22} and presentably symmetric monoidality follows from \cite[Proposition 2.30]{Pst22}.
\end{proof}

Because $\cC_\H^\fp$ is a full subcategory of $\cC$, the yoneda embedding extends to a functor $\cC\to \Sh(\cC_\H^\fp)$ after sheafification, which we will denote by $\hat y$ (reserving the undecorated $y$ for the restriction back to $\cC^\fp_\H$). Note that both $y$ and $\hat y$ automatically land in the subcategory of spherical sheaves by the calculation
\[
  \hat y(c)(d\amalg d')=\Map_{\cC}(d\amalg d', c)\simeq \Map_{\cC}(d, c)\times \Map_{\cC}(d', c).
\]
There is then an adjunction (\cite{Pst22})
\[
\Sigma^\infty_+:\Sh(\cC_\H^\fp) \leftrightarrows \Sh^\Sp(\cC_\H^\fp):\Omega^\infty
\]
which allows us to lift both to functors valued in spherical presheaves. Indeed, because spherical presheaves of spaces canonically lift to groupline $\bE_\infty$-monoids levelwise, the functor $\Sigma_+^\infty$ identifies the source with the full subcategory of the target consisting of those objects which are connective under the induced t-structure (see \ref{todo}), so that spherical sheaves of spaces admit canonical lifts to spherical sheaves of spectra.

\begin{definition}[Synthetic Analog Functor]\label{def:synanalog}
  The synthetic analog functor $\nu:\cC\to \Syn_\H$ is defined to be $\Sigma_+^\infty \hat y$, the canonical lift of $\hat y$ to a spherical sheaf of spectra.
\end{definition}

\begin{proposition}\label{prop:analogprops}
   The category $\Syn_H$ and its synthetic analog functor enjoy the following properties:
   \begin{enumerate}
    \item For all $P\in \cC_\H^\fp$ there is an equivalence $\Map(\nu P, X)\simeq \Omega^\infty X(P)$.
    \item The category $\Syn_H$ is generated under colimits by the compact objects $\Sigma^{k}\nu P$ for $P\in \cC_\H^\fp$.\footnote{In almost all cases of interest, the synthetic category has multi-graded suspensions. Here we are only interested in the formal suspension in the stable $\infty$-category $\Syn_\H$.}
    \item The functor $\nu$ preserves filtered colimits and direct sums.
    \item The functor $\nu$ is lax monoidal.
    \item The restriction $\nu:\cC_\H^\fp\to \Syn_\H$ is symmetric monoidal.
   \end{enumerate}  
\end{proposition}

\begin{proof}
  Claim (1) is proven identically to \cite[Lemma 4.11]{Pst22}:
  \[
  \Map(\nu P, X)\simeq \Map(\Sigma_+^\infty y(P), X)\simeq \Map(y(P), \Omega^\infty X)\simeq \Omega^\infty X(P).
  \]
  After noting that for $P\in \cC_\H^\fp$ the object $\nu P$ is compact in $\Syn_\H$ by \cite[Cor. 4.12]{Pst22}, the result follows from (1) as these objects can detect equivalences of spherical presheaves levelwise. Both (3) and (4) are true for the same reason: we can write the synthetic analog as the composite
  \[
  \cC\xrightarrow{\hat y} \Sh_\Sigma(\cC_\H^\fp) \xrightarrow{\Sigma_+^\infty}\Syn_\H
  \]
  wherein the second functor is a symmetric monoidal left adjoint, so it suffices to show that $\hat y$ is lax monoidal, preserves filtered colimits, and direct sums. Lax monoidality follows from recognizing that the functor $\hat y$ admits a left left adjoint which is the unique colimit preserving extension of $\cC^\fp_\H\hookrightarrow \cC$ and that this left adjoint is symmetric monoidal, so that its right adjoint its automatically lax monoidal. This is already enough for direct sums, which are both finite limits and colimits. For filtered colimits, we let $X_\alpha$ be a filtered diagram in $\cC$ and note that for any $Y\in \cC_\H^\fp$ we have
  \begin{align*}
    \colim_\alpha \hat y(X_\alpha)(P) & \simeq \colim_\alpha \Map(P, X_\alpha).
  \end{align*}
  But then $P$ is compact, so this filtered colimit is computed levelwise. But filtered colimits of sheaves are computed levelwise, so we are done. Finally we note that the restriction in (5) is the composite $\Sigma^\infty_+ \circ y$ so that it suffices to show that $y:\cC_\H^\fp\to \Sh(\cC_\H^\fp)$ itself is symmetric monoidal, but this property characterizes the Day convolution product of sheaves.
\end{proof}

\subsection{The sheaf t-structure} 

Recall (\cite{todo}) that for an arbitrary small $\infty$-site $\cT$ the category of sheaves of spectra on $\cT$ inherits a t-structure from the standard t-structure on spectra. Explicitly, the category of coconnective objects consists of the levelwise coconnective sheaves and the category of connective objects is determined against these. We can take homotopy groups levelwise and sheafify to get functors
\[
  \pi^\heartsuit_n:\Sh^\Sp(\cT)\to \Sh^\Ab(\cT)
\]
and which provide an alternative characterization of the t-structure, the connective objects are those sheaves whose sheaf homotopy groups vanish in negative degrees. 

\begin{proposition}[\cite{Pst22}]
  Let $\cT$ be an additive $\infty$-site. The t-structure on $\Sh^\Sp(\cT)$ described above restricts to a right-complete t-structure on spherical sheaves of spectra which is compatible with filtered colimits. Moreover, the heart of this t-structure is equivalent to the category of spherical sheaves of sets $\Sh_\Sigma^\Set(\cT)$.\footnote{Recall that spherical sheaves acquire abelian group structures levelwise so that the category of sheaves of sets is indeed abelian.}
\end{proposition}

\begin{convention}
  Our synthetic categories of interest will have multiple interesting t-structures. For easy of notation, we will refer to this t-structure as the \textit{sheaf t-structure}.
\end{convention}

\begin{lemma}
  There is an equivalence between the 1-categories of $\H_\star\H$-comodules in $\cA$ and the category $\Sh^{\Set}_\Sigma(\Comod_{\H_\star \H}^\fp)$ where the site $\Comod_{\H_\star \H}^\fp$ is the subcategory of finitely generated and projective comodules with single epimorphisms for covers.
\end{lemma}

\begin{lemma}
  
\end{lemma}

\begin{proposition}
  When $\Syn_\H$ is equipped with the sheaf t-structure, there is a monoidal equivalence $\Syn_\H^\heartsuit\simeq \Comod_{\H_\star \H}$.
\end{proposition}

\begin{proof}
  We claim that the functor
  \[
  H_\star:\cC_\H^\fp \to \Comod_{\H_\star\H}
  \]
  is a morphism of additive $\infty$-sites which induces an equivalence on categories of spherical sheaves of sets. Clearly covers are preserved by definition and finite generation and projectivity are checked on underlying $\H_\star$-modules, so that this is indeed a morphism of sites. In fact, it clearly reflects covers as well. TODO
\end{proof}


\subsection{Thread Structures and $\tau$}

The functor $\nu$ does not preserve (co)fiber sequences in general, although we will prove eventually that it preserves certain $\H$-exact cofibers. In particular, $\nu$ will not commute with formal suspensions. This failure is measured by a canonical comparison map
\[
\tau: \Sigma \circ \nu \to \nu \circ \Sigma
\]
induced by the universal property of the pushout defining $\Sigma$. We will refer to this map as the \textit{deformation parameter} of the deformation $\Syn_\H$ of $\cC$. 


\section{Bisynthetic Spectra}

Having established the necessary categorical preliminaries, the remainder of this article will study our category of Bisynthetic spectra. The following convention will be enforced throughout the remainder of the document unless specifically stated otherwise. 

\begin{convention}
  We will fix two Adams-type ring spectra $E,F$ satisfying the following additional assumptions:
  \begin{enumerate}
      \item The $E$-Adams spectral sequence for $F$ has no nonzero differentials.
      \item Some flatness hypothesis?
  \end{enumerate}
\end{convention}

\begin{definition}
  The homological context for $(E,F)$-bisynthetic spectra is given by the pair
  \[
  \Syn_{E} \xrightarrow{-\otimes \nu_E F} \Mod(\nu_E F) \xrightarrow{\pi_{*,*}} \Fun(\bZ^2, \Ab)
  \]
  and the resulting site of Definition \ref{def:projectivesite} is denoted $(\Syn_E)_{\nu_E F}^\fp$. It can be explicitly described as the full subcategory of $\Syn_E$ containing those compact objects whose $\nu_E F$ homology is finitely generated and projective over $\nu_E F_{**}$.
\end{definition}

\begin{definition}
  The category of Bisynthetic spectra $\Syn_{E,F}$ is the category of synthetic objects (Definition \ref{def:synsp}) with respect to the $(E,F)$-bisynthetic homological context, i.e., it is the category of spherical sheaves on the site $(\Syn_E)_{\nu_E F}^\fp$. We will denote the synthetic analog functor (definition \ref{def:synanalog}) by simply $\nu_F$ for brevity.
\end{definition}

\begin{proposition}
  The category $\Syn_{E,F}$ is a presentably symmetric monoidal stable $\infty$-category. It comes equipped with a synthetic analog functor
  \[
  \nu_{F}: \Syn_{E}\to \Syn_{E,F}
  \]
  which is lax monoidal, fully faithful, and preserves filtered colimits. 
\end{proposition}

\begin{proof}
  todo.
\end{proof}

\begin{notation}
  We will write $\nu^2$ for the functor $\nu_F\circ \nu_E$. Note that this functor again is fully faithful, preserves filtered colimits, and is lax monoidal as a composition of such functors. Note that the unit for $\Syn_{E,F}$ is $\nu^2(\bS)$; we will abuse notation and write $\bS$ for this object.
\end{notation}

\begin{definition}
  We will consider $\Syn_{E,F}$ to be a trigraded category, with the convention
  \[
  \bS^{k,w,v}:=\Sigma^{k-v}\nu_F(\bS^{v,w})=\Sigma^{k-v}\nu_F(\Sigma^{v-w}\nu_E(\bS))
  \]
  and denote by $\Sigma^{k,w,v}:\Syn_{E,F}\to \Syn_{E,F}$ given by $-\otimes \bS^{k,w,v}$.
\end{definition}

\begin{lemma}
  There are equivalences $\bS^{k,w,v}\otimes \bS^{k',w',v'}$.
\end{lemma}

\begin{definition}
  We define the trigraded mapping objects between bisynthetic spectra to be
  \[
  [X,Y]_{k,w,v}=[\Sigma^{k,w,v}X, Y]
  \]
  and trigraded homotopy groups $\pi_{k,w,v}X=[\bS, X]_{k,w,v}$.
\end{definition}

The category $\Syn_{E,F}$ comes naturally equipped with two deformation paramteres. The category $\Syn_E$ naturally has a parameter $\tau$, and we will again denote $\nu_{F}(\tau)$ by $\tau$.\footnote{The reader familiar with synthetic spectra may wonder whether the class $\nu_F(\tau)$ is $\lambda$-divisible. However, our assumptions on the spectra $E,F$ guarantee that $\tau$ will be detected by $\nu_E F$ and therefore it has filtration $0$ and is not divisible by $\lambda$.} The second parameter comes from the deformation along $\nu F$, and we will denote it $\lambda$. A common theme in this paper is that results involving $\lambda$ tend to follow from formal arguments about the synthetic construction, whereas arguments about $\tau$ will require extra work. The main theorem of this section, which we will prove in parts, is the following:

\begin{theorem}
  The structure of the category $\Syn_{E,F}$ with respect to the parameters $\lambda, \tau$ admits the following identifications:

  \begin{enumerate}
    \item The category $\lambda^{-1}\Syn_{E,F}$ of $\lambda$-local objects is equivalent to the category $\Syn_E$. In addition, the map $\bS\to \bS/\lambda$ is $\bE_\infty$ and $\Mod(\bS/\lambda)$ embeds fully faithfully into the category $\Stable(\nu_E F_{**}\nu_E F)$.
    \item The category $\tau^{-1}\Syn_{E,F}$ of $\tau$-local objects is equivalent to the category $\Syn_E$. In addition, the map $\bS\to \bS/\tau$ is $\bE_\infty$ and $\Mod(\bS/\tau)$ embeds fully faithfully into the category $\Stable(\nu_F E_{**}\nu_F E)$.
    \item The category $\tau^{-1}\lambda^{-1}\Syn_{E,F}$ is equivalent to the category $\Sp$ of spectra.
    \item The category $\Mod(\bS/(\tau,\lambda))$ emebds fully faithfully into the category $\Stable(\cP_{**}\cP)$ where $\cP$ is the object $E_*F$ and $\cP_{**}\cP$ is defined to be the homology of $\cP\otimes \cP$ in the category $\Stable(E_*E)$ or $\Stable(F_*F)$.
  \end{enumerate}
  All of the equivalences and functors described are symmetric monoidal.
  
\end{theorem}

\begin{proof}
  Claim (1) is \ref{todo}, claim (2) is \ref{todo}, and claim (4) is \ref{todo}. Claim (3) follows from either (1) or (2) as we already know the effect of inverting the deformation parameter in $\Syn_E$ by \cite{todo}.
\end{proof}

In order to prove these results, we will need the following sites whose categories of spherical sheaves will provide new models for some of the above.

\begin{definition}[The $\tau$-local site]
  Let $\Syn_E^{\fp, \tau=1}$ denote the subcategory of $\tau^{-1}\Syn_E$ consisting of those $\tau$-local synthetic spectra which are compact in $\tau^{-1}\Syn_E$ and whose $\tau^{-1}(\nu_E F)$-homology is finitely generated and projective over $\pi_{*,*}\tau^{-1}(\nu_E F)$. We say that a map in this category is a cover if it is a surjection on $\tau^{-1}(\nu_E F)$-homology.
\end{definition}

\begin{definition}[The mod-$\tau$ site]
  Let $\Syn_E^{\fp, \tau=0}$ denote the subcategory of $\Mod(\Syn_E, \bS/\tau)^\omega$ consisting of objects whose $(\nu_E F/\tau)$-homology is finitely generated and projective over $\pi_{*,*}(\nu_E F/ \tau)$. A map in this category is a cover if it induces a surjection on $\nu_E F/ \tau$-homology.
\end{definition}

\begin{lemma}
  The sites $\Syn_E^{\fp, \tau=1}$ and $\Syn_E^{\fp, \tau=0}$ are both excellent $\infty$-sites with the coverages described above.
\end{lemma}

\begin{proposition}
  There is a symmetric monoidal equivalence  of categories $\tau^{-1}\Syn_{E,F}\simeq \Sh_{\Sigma}^\Sp(\Syn_E^{\fp, \tau=1})$.
\end{proposition}

\begin{theorem}
  There is a symmetric monoidal equivalence $\tau^{-1}\Syn_{E,F}\simeq \Syn_F$.
\end{theorem}

\begin{proof}
  To avoid ambiguity, let us write $\Re$ for the $\tau$-inversion functor when viewed as having codomain $\Sp$ and $\tau^{-1}$ for the mapping telescope internal to $\Syn_E$. With this notation, note that $\Re$ factors through $\tau^{-1}$.
  
  
  It then suffices to show that there is a symmetric monoidal equivalence $\Sh_{\Sigma}^\Sp(\Syn_E^{\fp, \tau=1})\simeq \Syn_F$. We claim that $\Re$ restricts to a functor $\Syn_E^{\fp, \tau=1}\to \Sp^\fp_F$ and that this is an equivalence of sites. 

  First we show that the functor restricts as described. If $X\in \Syn_E^{\fp, \tau=1}$ this amounts to proving that $\Re(X)$ is compact and that it has finitely generated and projective $F$-homology. Note first that the two notions of compactness coincide, due to the equivalence $\tau^{-1}\Syn_E\simeq \Sp$. Then we can directly compare the two homologies as we have
  \begin{align*}
    \pi_{**}(\tau^{-1}\nu_E F\otimes X)\cong F_*\Re(X)[\tau^{\pm}]
  \end{align*}
  and aftering inverting $\tau$, the synthetic weight becomes superfluous as $F_*\Re(X)$ has the same information. As a result the finite generation and projectivity criterion coincide. Finally, the same argument shows that the two notions of coverage coincide.
\end{proof}

\begin{proposition}
  There is a symmetric monoidal equivalence  of categories $\Mod(\Syn_{E,F}, \bS/\tau)\simeq \Sh_{\Sigma}^\Sp(\Syn_E^{\fp, \tau=0})$.
\end{proposition}

\begin{theorem}
  There is a symmetric monoidal equivalence $\Mod(\Syn_{E,F})\simeq \Syn_F$.
\end{theorem}


\subsection{$t$-structures}

In the category $\Syn_E$ of $E$-synthetic spectra developed by \cite{Pst22}, there is a natural $t$-structure which plays an important role in the structure of the category. This $t$-structure is a specialization of a general $t$-structure on spherical sheaves, whose heart can also be identified:

\begin{definition}[\cite{Pst22}]
\label{con_cocon_defin}
Suppose $\cC$ is an additive $\infty$-category and let $Sh_{\Sigma}^{\Sp}(\cC)$ denote the category of spectra-valued spherical sheaves on $\cC$. An object $X\in Sh_{\Sigma}^{\Sp}(\cC)$ is \textit{connective} if the sheafification of the presheaf $\pi_nX$ defined by
$$
c\in\cC\mapsto \pi_nX(c)
$$
satisfies $\pi_nX=0$ for $n<0$. An object $X$ is \textit{coconnective} if $\Omega^{\infty}X$ is a discrete sheaf of spaces.
\end{definition}

\begin{proposition}[\cite{Pst22}]
\label{general_sheaf_tstruct_prop}
The pair $(\Sh_{\Sigma}^{\Sp}(\cC)_{\geq 0},\Sh_{\Sigma}^{\Sp}(\cC)_{\leq 0})$ of full subcategories of connective and coconnective objects determines a right
complete $t$-structure on $\Sh_{\Sigma}^{\Sp}(\cC)$ compatible with filtered colimits. Moreover, there is a canonical equivalence $\Sh_{\Sigma}^{\Sp}(\cC)^\heartsuit\simeq \Sh_{\Sigma}^{\mathrm{Set}}(\cC)$ between the heart of this $t$-structure and the category of
spherical sheaves of sets.
\end{proposition}

When specializing to $\cC=\Sp_E^{fp}$ for an Adams-type spectrum $E$, \cite{Pst22} shows that the functor of additive $\infty$-sites $E_*(-):\Sp_E^{fp}\to\Comod_{E_*E}^{fp}$ induces an equivalence on spherical sheaves of sets. Together with work of Goerss-Hopkins, this gives a nice identification of the heart $\Syn_E^\heartsuit$ of the $t$-structure on $\Syn_E$ in terms of $E_*E$-comodules:

\begin{theorem}[\cite{GH05},\cite{Pst22}]
If $E$ is an Adams-type spectrum, then the functor of additive $\infty$-sites $E_*(-):\Sp_E^{fp}\to\Comod_{E_*E}^{fp}$ induces an equivalence $Sh_{\Sigma}^{\mathrm{Set}}(\Sp_E^{fp})\simeq Sh_{\Sigma}^{\mathrm{Set}}(\Comod_{E_*E}^{fp}).$
In particular, there are equivalences
$$
\Syn_E^{\heartsuit}\simeq Sh_{\Sigma}^{\mathrm{Set}}(\Comod_{E_*E}^{fp})\simeq \Comod_{E_*E}.
$$
\end{theorem}

For $X\in\Syn_E$, \cite{Pst22} also identifies an explicit formula for the homotopy objects $\pi_k^\heartsuit X$ in terms of synthetic $E$-homology:

\begin{theorem}[\cite{Pst22}]
\label{SynE_homology_tstruct}
    At the level of graded abelian groups, there's an isomorphism
$$
(\pi_k^\heartsuit X)_l\cong \nu E_{k+l,l}X
$$
In particular, $X$ is connective if and only if $\nu E_{k,w}X$ is concentrated in non-negative Chow degree $k-w\geq 0$.
\end{theorem}

If $X=\nu Y$ is the synthetic analog of a spectrum $Y$, the calculation
$$
\nu E_{*,*}\nu Y\cong \nu (E\otimes Y)_{*,*}\cong E_*Y[\tau],
$$
where $E_*Y$ is concentrated in bidegree $(k,k)$, shows that $\nu Y$ is always connective in this $t$-structure. In particular, this implies that $\nu Y\otimes C\tau$ lies in $\Syn_E^\heartsuit$. This fact is key in relating $\Mod_{C\tau}(\Syn_E)$ to $\Stable_{E_*E}$ and the $E$-Adams spectral sequence for $Y$ to the $\tau$-Bockstein spectral sequence for $\nu Y$ in $\Syn_E$.

\bigskip

Just as there are two deformation parameters, the category $\Syn_{E,F}$ will have two t-structures corresponding to bisynthetic $E$ and $F$ homology. We first study the t-structure related to $F$ in section \ref{F_tstruct_subsec}. This $t$-structure comes about in the exact same way that the $t$-structure in $\Syn_E$ appears. We also prove several results about this $t$-structure, analogous to results in \cite{Pst22}, which will be useful later for identifying $\Syn_{E,F}[\lambda^{-1}]$ and $\Mod_{C\lambda}(\Syn_{E,F})$ in terms of more familiar categories.

\bigskip

We then study a $t$-structure related to $E$ in \ref{E_tstruct_subsec}. The connective objects of this $t$-structure are controlled by $\nu^2(E)$-homology, analogous to Theorem~\ref{SynE_homology_tstruct}. We also prove several results about this $t$-structure, analogous to results in \cite{Pst22}, which will be useful later for identifying $\Syn_{E,F}[\tau^{-1}]$ and $\Mod_{C\tau}(\Syn_{E,F})$ in terms of more familiar categories.

\subsection{$t$-structure for $F$}
\label{F_tstruct_subsec}

Since $\Syn_{E,F}$ is the category of spherical sheaves on an additive $\infty$-structure, we immediately get a $t$-structure on $\Syn_{E,F}$ via Definition~\ref{con_cocon_defin} and Proposition~\ref{general_sheaf_tstruct_prop}:

\begin{proposition}
\label{F_bisyn_tstruct_prop}
The pair $((\Syn_{E,F})_{\geq 0}^F,(\Syn_{E,F})_{\leq 0}^F)$ of full subcategories of connective and coconnective objects determines a right
complete $t$-structure on $\Syn_{E,F}$ compatible with filtered colimits. Moreover, there is a canonical equivalence $\Syn_{E,F}^{F,\heartsuit}\simeq Sh_{\Sigma}^{\mathrm{Set}}((\Syn_E)_{\nu F}^{fp})$ between the heart of this $t$-structure and the category of
spherical sheaves of sets.  
\end{proposition}

\begin{remark}
We use the superscript $F$ to emphasize that this $t$-structure is related to $F$ and $\lambda$. This will become clearer later in the subsection when we relate the $t$-structure to $\nu^2F$-homology. We will also use the notation $\tau_{\geq n}^F, \tau_{\leq n}^F$ for the associated truncation functors.
\end{remark}

We can identify the heart in a similar manner to \cite{Pst22}:

\begin{theorem}
    The heart $\Syn_{E,F}^{F,\heartsuit}$ is equivalent to $\Comod_{\nu_EF_{*,*}\nu_EF}$. (monoidal conditions should be added too)
\end{theorem}

\begin{proof}
    By Proposition~\ref{F_bisyn_tstruct_prop}, the heart is equivalent to $Sh_{\Sigma}^{\mathrm{Set}}((\Syn_E)_{\nu F}^{fp})$. By (ref. to lemma in Section 2), the morphism of $\infty$-sites $$\nu_EF_{*,*}(-):(\Syn_E)_{\nu F}^{fp}\to\Comod_{\nu_EF_{*,*}\nu_EF}^{fp}$$ is one which reflects coverings and admits a common envelope. By \cite[Rem. 2.50]{Pst22}, this induces an adjoint equivalence $$Sh_{\Sigma}^{\mathrm{Set}}((\Syn_E)_{\nu F}^{fp})\rightleftarrows Sh_{\Sigma}^{\mathrm{Set}}(\Comod_{\nu_EF_{*,*}\nu_EF}^{fp})\,.$$
The bigraded Hopf algebroid $(\nu_EF_{*,*},\nu_EF_{*,*}\nu_EF)$ is Adams, in the sense of \cite[Def. 3.1]{Pst22}, by (Lemma in Section 2 which proves that it's Adams). By a bigraded version of \cite[2.1.12]{GH05}, \cite[Thm. 3.2]{Pst22} there is an equivalence
$$
\Comod_{\nu_EF_{*,*}\nu_EF}\simeq Sh_{\Sigma}^{\mathrm{Set}}(\Comod_{\nu_EF_{*,*}\nu_EF}^{fp}),
$$
and the result follows.
\end{proof}

Now we work towards identifying the homotopy objects $\pi_k^{F,\heartsuit}X$ in terms of $\nu^2F$-homology.

\begin{lemma}
\label{F_dual_tstruct_lemma}
    For $X\in\Syn_{E,F}$, the graded components of the $\nu_EF_{*,*}\nu_EF$-comodule $\pi_k^{F,\heartsuit}X$ are described by
    $$
(\pi_k^{F,\heartsuit}X)_{l,m} \cong \colim_\alpha \pi_kX(\Sigma^{l,m}D\nu_E F_\alpha),
    $$
    where $F\simeq \colim_\alpha F_\alpha$ is a presentation of $F$ as a filtered colimit of $F$-finite projective spectra.
\end{lemma}

\begin{proof}
    This is essentially a bigraded version of \cite[Lemma 4.17]{Pst22} and the proof is similar to the proof of that lemma. By \cite[Thm. 2.58]{Pst22}, the sheaf $\pi_k^{F,\heartsuit}X\in Sh_{\Sigma}^{\mathrm{Set}}((\Syn_E)_{\nu F}^{fp})$ is representable by some comodule $N$; i.e. $$(\pi_k^{F,\heartsuit}X)(-)\simeq \Hom_{\nu_EF_{*,*}\nu_EF}(\nu_EF_{*,*}(-),N).$$
    Now notice that $\nu_EF_{*,*}\nu_EF\simeq \colim_\alpha \nu_EF_{*,*}\nu_EF_\alpha$, since $\nu_E$ commutes with filtered colimits, and $E_*(D\nu_EF_\alpha)\cong\Hom_{\nu_EF_{*,*}\nu_EF}(\nu_EF_{*,*}\nu_EF_\alpha,\nu_EF_{*,*})$. Then by \cite[Lemma 3.3]{Pst22}, as a bigraded abelian group
    $$
N_{l,m}\cong \colim_\alpha \pi_k^{F,\heartsuit}X(\Sigma^{l,m}D\nu_EF_\alpha).
    $$
    By a bigraded version of \cite[Lemma 3.25]{Pst22},
    $$
\colim_{\alpha}\pi_k^{F,\heartsuit}X(\Sigma^{l,m}D\nu_EF_\alpha)\cong\colim_{\alpha} \pi_kX(\Sigma^{l,m}D\nu_EF_\alpha),
    $$
    which completes the proof.
\end{proof}

\begin{theorem}
\label{F_homol_tstruct_theorem}
    For $X\in\Syn_{E,F}$, there is an isomorphism
    $$
(\pi_k^{F,\heartsuit}X)_{l,m}\cong\nu^2F_{k+l,m,l}X,
    $$
    where $\nu^2F_{*,*,*}(-)$ denotes bisynthetic $F$-homology.
\end{theorem}

\begin{proof}
    Again, this is a similar proof to \cite[Thm. 4.18]{Pst22}. We have that
    \begin{equation*}
     \begin{aligned}
      \nu^2F_{k+l,m,l}X&\cong [\bS^{k+l,m,l},\nu^2F\otimes X] \\
      &\cong \colim_\alpha[\Sigma^k\nu_{\nu F}(\bS^{l,m}_E),\nu^2F_\alpha\otimes X] \\
      &\cong \colim_\alpha [\Sigma^k\nu_{\nu F}(\Sigma^{l,m}D\nu_EF_\alpha),X] \\
      &\cong \colim_\alpha \pi_kX(\Sigma^{l,m}D\nu_EF_\alpha) \\
      &\cong (\pi_k^{F,\heartsuit}X)_{l,m}.
    \end{aligned}   
    \end{equation*}
 The first isomorphism is by definition, the second isomorphism follows from (definition from Section 2 about trigraded spheres) and equivalence $\nu^2F\simeq\colim_\alpha \nu^2F_\alpha$, the fourth isomorphism follows from (lemma from Section 2 which shows that $map(\nu_{\nu F}P,X)\simeq \Omega^\infty(X(P))$ for $P\in(\Syn_E)_{\nu F}^{fp}$), and the fifth isomorphism follows from Lemma~\ref{F_dual_tstruct_lemma}.   
\end{proof}

As a corollary, we get the following analog of \cite[Cor. 4.19]{Pst22}:

\begin{corollary}
\label{F_chow_degree_cor}
A bisynthetic spectrum $X\in\Syn_{E,F}$ is in $(\Syn_{E,F})_{\geq 0}^F$ if and only if $\nu^2F_{k,w,v}X =0$ for Chow degree $k-v<0$. 
\end{corollary}

\begin{proof}
    In this $t$-structure, $X\in\Syn_{E,F}$ is in $(\Syn_{E,F})_{\geq 0}^F$ if and only if $\pi_k^{F,\heartsuit}X$ vanishes for $k<0$. By Theorem~\ref{F_homol_tstruct_theorem}, this happens exactly when $k-v<0$.
\end{proof}

This result is what motivates naming this $t$-structure after $F$. As a consequence, we see that the $\nu F$-synthetic analog of an $E$-synthetic spectrum $Y$ is always connective.

\begin{corollary}
    If $Y\in\Syn_E$, then $\nu_{\nu F}Y\in (\Syn_{E,F})_{\geq 0}^F$.
\end{corollary}

\begin{proof}
    Consider the homology calculation
\begin{equation*}
    \begin{aligned}
        \nu^2F_{*,*,*}\nu_{\nu F}Y &\cong \nu_{\nu F}(\nu F\otimes Y)_{*,*,*} \\
        &\cong \nu F_{*,*} Y[\lambda]\, ,
    \end{aligned}
\end{equation*}
where $\nu F_{k,w} Y$ lives in tridegree $(k,w,k)$. The first isomorphism follows from (lemma in Section 2 about when $\nu_{\nu F}$ is symmetric monoidal) and the second isomorphism follows (lemma in Section 2 about homotopy of $\nu F$-module). The result then follows from Corollary~\ref{F_chow_degree_cor}.
\end{proof}

This means that for the $\nu F$-synthetic analog of an $E$-synthetic spectrum $Y$, the tensor product $\nu_{\nu F}Y\otimes C\lambda$ lives in the heart $\Syn_{E,F}^{F,\heartsuit}$.

\begin{corollary}
If $Y\in\Syn_E$, then $\Sigma^{0,0,-1}\nu_{\nu F}Y\simeq \tau_{\geq 1}^F(\nu_{\nu F}Y)$ and $\nu_{\nu F} Y\otimes C\lambda\simeq \tau_{\leq 0}^F(\nu_{\nu F}Y)$. In particular, $\nu_{\nu F}Y\otimes C\lambda\in \Syn_{E,F}^{F,\heartsuit}$.  
\end{corollary}

\begin{proof}
   Again, the proof is similar to the proof of \cite[Lemma 4.29]{Pst22}. Consider the cofiber sequence
   $$
\Sigma^{0,0,-1}\nu_{\nu F}Y\xrightarrow{\lambda}\nu_{\nu F}Y\to \nu_{\nu F}Y\otimes C\lambda
   $$
   By Corollary~\ref{F_chow_degree_cor}, it's clear that $\Sigma^{0,0,-1}\nu_{\nu F}Y$ is 1-connective. By using the definition of $\nu_{\nu F}$ and the colimit-comparison definition of $\lambda$, it follows that $\nu_{\nu F}Y\otimes C\lambda$ lives in $(\Syn_{E,F})_{\leq 0}^F$. The result then follows.
\end{proof}

\begin{remark}
Similar to $\Syn_E$, we see that $\nu_{\nu F}Y\otimes C\lambda$ is lives in an algebraic category; namely the category of $\nu_EF_{*,*}\nu_E F$-comodules. In Section 4, we will show that, in fact, $\nu_{\nu F}Y\otimes C\lambda$ can be identified with the comodule $\nu_{E}F_{*,*}Y$ and there is an embedding $\Mod_{C\lambda}(\Syn_{E,F})\hookrightarrow \Stable_{\nu_EF_{*,*}\nu_EF}$ of $C\lambda$-modules into the stable comodule category associated to the bigraded Hopf algebroid $(\nu_EF_{*,*},\nu_EF_{*,*}\nu_EF)$.    
\end{remark}

\subsection{$t$-structure for $E$}
\label{E_tstruct_subsec}

\section{Specializations by $\tau,\lambda$}

\section{The Categorified Miller Square}

\end{document} 


