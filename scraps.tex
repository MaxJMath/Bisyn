

\begin{itemize}
  \item Like I said in some of the earlier comments, I think it would be a good idea to simplify some of the definitions and make sure we reference people (e.g. Piotr/Irakli) who have already come up with some of these notions.
  \item Some other Piotr/Irakli stuff we could quote from here include thread structure/tau-localization/Ctau-mod
  \item With these notions all set up, it should be straightforward then to start writing down all the categories we need that come out of bisynthetic spectra; e.g. Ctau/Clambda-mod, tau/lambda inverted categories, etc. and the categories these module categories are equivalent to
  \item Question: for this def. of $\Sh_\Sigma(\cC^\omega_\H, \Sp)$, is it still true that Ctau-mod embeds in some "stable comodule" category related to $\H$? Would have to make the stable comodule category precise. This is probably related to the derived categories that Irakli/Piotr deal with
\end{itemize}















\section{Recollections on Synthetic Spectra}

\begin{definition}
  Given $E$ any spectrum and $X$ a finite spectrum, $X$ is said to be $E$-finite projective if $E_*X$ is finitely generated and projective over $E_*$. We denote the full subcategory of spectra spanned by such $\Sp^\fp_E$. A map $X\to Y$ in $\Sp^\fp_E$ is said to be a cover if it is an epimorphism after taking $E$-homology.
\end{definition}

\begin{definition}
  A site $\cC$ which is additive is said to be in addition excellent if it is equipped with a symmetric monoidal structure such that all objects admit duals and such that the functors $-\otimes c$ preserve covers for all $c\in \cC$.
\end{definition}

\begin{lemma}
  The category $\Sp^\fp_E$ is additive and acquires the structure of a site with the covering families given by singletons of $E$-epimorphisms as above. Equipped with the smash product of spectra, it is excellent.
\end{lemma}


\begin{definition}
  A spectrum $E$ is said to be Adams-type if there exists a filtered diagram $X_\alpha$ such that each $X_\alpha$ is in $\Sp^\fp_E$ and such that the natural map $E^*X_\alpha\to \Hom_{E_*}(E_*X_\alpha, E_*)$ is an isomorphism.
\end{definition}



\begin{definition}
  The category $\Syn_E$ of synthetic spectra is the category of spherical presheaves of spectra on the excellent site $\Sp^\fp_E$.
\end{definition}



\section{The Bisynthetic Model}

\subsection{Synthetic finite projectives}

\begin{definition}
  Given $F,X\in \Syn_E$ we say that $X$ is $F$-finite projective if it compact as a synthetic spectrum and if $F_{*,*}X:=\pi_{*,*}(F\otimes X)$ is a finitely generated projective module over $F_{*,*}:=\pi_{*,*}F$. We denote the full subcategory of $F$-finite projectives $(\Syn_E)_F^\fp$. A map $X\to Y$ of $F$-finite projectives is said to be a cover if it is an epimorphism after applying taking $F$-homology.
\end{definition}

\begin{lemma}
  The category $\Syn_F^\fp$ is an additive site when equipped with the covering families consisting of single $F_{*,*}$-epimorphisms.
\end{lemma}

\begin{proof}
  The proof is identical to \cite[Lemma 3.22]{piotr}.
\end{proof}

\begin{lemma}
  Equipped with the tensor product of synthetic spectra, $(\Syn_E)_F^\fp$ is excellent.
\end{lemma}

\begin{proof}
  all these proofs look like the one in piotrs paper goes through identically, but I am going to come back to that later.
\end{proof}

\section{Special and Generic fibers over $\lambda$ and $\tau$}

\subsection{The $\lambda$-generic fiber}

\begin{theorem}
  The subcategory of $\lambda$-local objects in $\Bisyn$ is canonically equivalent to $\Syn_E$.
\end{theorem}

\subsection{The $\lambda$-special fiber}

\begin{theorem}
  The category $\Mod(\Bisyn, \bS/\lambda)$ is a full subcategory of $\Stable(\nu F_{*,*}\nu F)$ which is an equivalence if (???). Restricted to the image of $\nu_F$, this equivalence takes an $E$-synthtetic spectrum to its $\nu F$-homology.
\end{theorem}

\subsection{The $\tau$-generic fiber}

\begin{notation}
  We will write $(\Syn_E)^{\tau-\loc}_F$ for the site $(\Syn_E)_{\tau^{-1}F}^\fp$.
\end{notation}

\begin{lemma}
  The functor $\tau^{-1}$ induces a morphism of excellent sites $(\Syn_E)^\fp_{F}\to (\Syn_E)^{\tau-\loc}_F$.
\end{lemma}

\begin{proof}
  Because the category of $\tau$-local synthetic spectra is a smashing localization, inverting $\tau$ preserves compact objects. Then note that there is an equivalence $\tau^{-1}F\otimes \tau^{-1}X\simeq \tau^{-1}(F\otimes X)$, so that we can compute:
  \[
  (\tau^{-1}F)_{*,*}X \cong F_{*,*}X[\tau^{-1}]  
  \]
  and if $F_{*,*}X$ is finitely generated and projective over $F_{*,*}$, then $F_{*,*}X[\tau^{-1}]$ will be finitely generated and projective over $F_{*,*}[\tau^{-1}]\cong (\tau^{-1}F)_{*,*}$ and this process will also preserve epimorphisms. Because the relevant pullbacs in both sites are computed in $\Syn_E$ they are also pushouts and the left adjoint $\tau^{-1}$ will preserve them. The symmetric monoidality of $\tau^{-1}$ shows that this morphism of sites upgrades to one of excellent sites. 
\end{proof}

\begin{lemma}
  In the induced adjunction $F:\Bisyn \to \Sh_{\Sigma}((\Syn_E)^\{\tau-\loc}_F):G$, the right adjoint $G$ is cocontinuous, $G(X)$ is $\tau$-local for all $X$, and the essential image consists of all $\tau$-local bisynthetic spectra.
\end{lemma}

\begin{proof}
  
\end{proof}

\begin{proposition}
  The subcategory of $\tau$-local objects in $\Bisyn$ is equivalent to the category of spherical sheaves on the site $(\Syn_E)^{\tau-\loc}_{\nu F}$.
\end{proposition}

\begin{theorem}
  There is an equivalence of spherical sheaves over $(\Syn_E)^{\tau-\loc}_{\nu F}$ and $\Sp^\fp_F$. As a result, the category of $\tau$-local bisynthetic spectra is equivalent to $\Syn_F$.
\end{theorem}

\subsection{The $\tau$-special fiber}

Given homological context $\H$ the functor $\nu:\cC\to \Syn_{\H}$ will not in general commute with the suspension functor. This failure is measured by the canonical comparison map
\[
\Sigma \nu(\one_{\cC}) \to 
\]

\maxnote{I have no idea what to do for this at the moment, would love any ideas.}

