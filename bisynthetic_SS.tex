\documentclass[10pt]{amsart}
\usepackage[letterpaper,margin=1in,footskip=0.25in]{geometry}

%\usepackage{garamondlibre}
\usepackage{times}
%\usepackage{CormorantGaramond}
%\usepackage{baskervald}
\usepackage{microtype}
\usepackage{eucal}
\usepackage{setspace}
\usepackage{mathrsfs}
\usepackage{tikz-cd}

\usepackage[
backend=biber,
style=alphabetic,
sorting=nyt, maxbibnames=20, maxalphanames=99
]{biblatex}
%\usepackage{pdfpages}

\addbibresource{refs.bib}

\usepackage{amsmath,amssymb,amsthm}
\usepackage{mathtools}
\usepackage{mathabx}
\makeatletter
  \newcommand{\supsize}{%
    \expandafter\ifx\csname S@\f@size\endcsname\relax
      \calculate@math@sizes
    \fi
    \csname S@\f@size\endcsname
    \fontsize\sf@size\z@\selectfont
  }
  \DeclareRobustCommand{\tsup}[1]{%
    \leavevmode\raise.9ex\hbox{\supsize #1}%
  }
  \DeclareTextSymbolDefault{\textprimechar}{OMS}
  \DeclareTextSymbol{\textprimechar}{OMS}{48}
  \DeclareRobustCommand{\tprime}{\tsup{\textprimechar}}
  \ProvideTextCommandDefault{\cprime}{\tprime}
\makeatother


\usepackage{enumitem}
\setlist{noitemsep}

\usepackage[pdfusetitle,colorlinks]{hyperref}
\hypersetup{allcolors=blue}
\usepackage[capitalise,noabbrev]{cleveref}
\crefformat{equation}{\ensuremath{(#2#1#3)}}
\crefmultiformat{equation}{\ensuremath{(#2#1#3)}}{ and~\ensuremath{(#2#1#3)}}{, \ensuremath{(#2#1#3)}}{, and~\ensuremath{(#2#1#3)}}

\theoremstyle{definition}

\numberwithin{figure}{section}
\numberwithin{equation}{section}

\newtheorem{theorem}[figure]{Theorem}
\newtheorem{lemma}[figure]{Lemma}
\newtheorem{construction}[figure]{Construction}

\newtheorem{corollary}[figure]{Corollary}
\newtheorem{proposition}[figure]{Proposition}
\newtheorem{definition}[figure]{Definition}
\newtheorem{notation}[figure]{Notation}
\newtheorem{exercise}[figure]{Exercise}
\newtheorem{remark}[figure]{Remark}
\newtheorem{example}[figure]{Example}
\newtheorem{conjecture}[figure]{Conjecture}

\newtheorem{thm}[figure]{Theorem}
\newtheorem{var}[figure]{Variant}
\newtheorem{lem}[figure]{Lemma}
\newtheorem{cons}[figure]{Construction}

\newtheorem{cor}[figure]{Corollary}
\newtheorem{prop}[figure]{Proposition}
\newtheorem{defn}[figure]{Definition}
\newtheorem{notn}[figure]{Notation}
\newtheorem{rem}[figure]{Remark}

\newcommand{\op}{\mathrm{op}}

\newcommand{\poly}[1]{{#1}[x_1,\ldots,x_n]}
\newcommand{\cA}{\mathcal{A}}
\newcommand{\cB}{\mathcal{B}}
\newcommand{\cC}{\mathcal{C}}
\newcommand{\cD}{\mathcal{D}}
\newcommand{\cE}{\mathcal{E}}
\newcommand{\cF}{\mathcal{F}}
\newcommand{\cG}{\mathcal{G}}
\newcommand{\cH}{\mathcal{H}}
\newcommand{\cI}{\mathcal{I}}
\newcommand{\cJ}{\mathcal{J}}
\newcommand{\cK}{\mathcal{K}}
\newcommand{\cL}{\mathcal{L}}
\newcommand{\cM}{\mathcal{M}}
\newcommand{\cN}{\mathcal{N}}
\newcommand{\cO}{\mathcal{O}}
\newcommand{\cP}{\mathcal{P}}
\newcommand{\cQ}{\mathcal{Q}}
\newcommand{\cR}{\mathcal{R}}
\newcommand{\cS}{\mathcal{S}}
\newcommand{\cT}{\mathcal{T}}
\newcommand{\cU}{\mathcal{U}}
\newcommand{\cV}{\mathcal{V}}
\newcommand{\cW}{\mathcal{W}}
\newcommand{\cX}{\mathcal{X}}
\newcommand{\cY}{\mathcal{Y}}
\newcommand{\cZ}{\mathcal{Z}}
\newcommand{\pp}{\mathbf{p}}
\newcommand{\mm}{\mathbf{m}}
\newcommand{\mbfa}{\mathbf{A}}
\newcommand{\sX}{\mathscr{X}}
\newcommand{\sY}{\mathscr{Y}}
\newcommand{\sch}{\operatorname{Sch}}
\newcommand{\GG}{\mathbf{G}}
\newcommand{\BB}{\mathbf{B}}
\newcommand{\pic}{\operatorname{Pic}}

\newcommand{\MU}{\mathrm{MU}}
\newcommand{\BP}{\mathrm{BP}}
\newcommand{\SU}{\mathrm{SU}}
\newcommand{\BU}{\mathrm{BU}}


\newcommand{\gr}{\mathrm{gr}}
\newcommand{\fil}{\mathrm{fil}}

\newcommand{\BGL}{\mathrm{BGL}}
\newcommand{\Thick}{\mathrm{Thick}}
\newcommand{\Th}{\mathrm{Th}}
\newcommand{\Ext}{\operatorname{Ext}}

\newcommand{\spaces}{\mathcal{S}}
\newcommand{\one}{\mathrm{1}}
\newcommand{\ord}{\mathrm{ord}}
\newcommand{\wt}{\mathrm{wt}}
\newcommand{\unfinished}{\textcolor{red}{INCOMPLETE }}
\newcommand{\done}{\textcolor{green}{DONE }}

\newcommand{\anss}{{}^{\mathrm{an}}\mathrm{E}}
\newcommand{\genanss}{{}^{\mathrm{F}}\mathrm{E}}

\newcommand{\mayss}{{}^{\mathrm{may}}\mathrm{E}}
\newcommand{\vlines}{\mathrm{VL}}
\newcommand{\page}{\mathrm{page}}
\newcommand{\intercept}{\mathrm{incpt}}

\newcommand{\kos}[1]{\mathbf{K}_\bullet(\mathbf{#1})}



\newcommand{\ass}{\operatorname{Ass}}
\newcommand{\spec}{\operatorname{Spec}}
\newtheoremstyle{cited}{.5\baselineskip\@plus.2\baselineskip\@minus.2\baselineskip}{.5\baselineskip\@plus.2\baselineskip\@minus.2\baselineskip}{\itshape}{}{\bfseries}{\bfseries .}{5pt plus 1pt minus 1pt}{\thmname{#1}\thmnumber{ #2}\thmnote{ \normalfont #3}}
\theoremstyle{cited}
\newtheorem{citedthm}[figure]{Theorem}
\newtheorem{citedprop}[figure]{Proposition}
\newtheorem{citedcor}[figure]{Corollary}

%Operators
\DeclareMathOperator{\Aut}{Aut}
\newcommand{\isom}{\operatorname{Isom}}
\newcommand{\sets}{\operatorname{Sets}}
\newcommand{\htensor}{\hat{\otimes}}

%Objects
\newcommand{\tensor}{\otimes}
\newcommand{\into}{\hookrightarrow}
\newcommand{\aff}{\mathbb{A}}
\newcommand{\mf}[1]{\mathbf{#1}}
\newcommand{\ess}{\operatorname{Es}/S}

\newcommand{\bZ}{\mathbb{Z}}
\newcommand{\bN}{\mathbb{N}}
\newcommand{\bS}{\mathbb{S}}
\newcommand{\bD}{\mathbb{D}}
\newcommand{\bE}{\mathbb{E}}
\newcommand{\bF}{\mathbb{F}}

%Maps
\newcommand{\qcoh}[1]{\operatorname{Qcoh}(#1)}
\newcommand{\id}{\mathrm{id}}
\newcommand{\der}[3]{\textrm{Der}_{#1}(#2,#3)}

\newcommand{\affs}{\operatorname{Aff}/S}
\newcommand{\ev}{\operatorname{ev}}
\newcommand{\alg}{\operatorname{Alg}}
\newcommand{\cech}{\operatorname{Cech}}
\newcommand{\tot}{\operatorname{Tot}}
\newcommand{\Fun}{\operatorname{Fun}}
\newcommand{\fun}{\Fun}

\newcommand{\CAlg}{\operatorname{CAlg}}
\newcommand{\PSigma}{\operatorname{P}_{\Sigma}}

\newcommand{\sCAlg}{\operatorname{sCAlg}}
\renewcommand{\poly}{\mathrm{poly}}
\newcommand{\colim}{\operatorname{colim}}
\newcommand{\cof}{\operatorname{cof}}

\newcommand{\HH}{\operatorname{HH}}
\newcommand{\THH}{\operatorname{THH}}

\newcommand{\Fil}{\mathrm{fil}}
%Arrows
\newcommand{\longisoto}{\overset{\sim}{\longrightarrow}}
\newcommand{\spm}{\operatorname{Spm}}

\newcommand{\B}{\mathbf{B}}
\newcommand{\Nm}{\operatorname{Nm}}

\newcommand{\PrL}{\mathrm{Pr}^{\mathrm{L}}}
\newcommand{\Ab}{\mathrm{Ab}}
\newcommand{\Sp}{{\mathcal{S}\mathrm{p}}}
\newcommand{\Mod}{\mathrm{Mod}}

\newcommand{\HE}{\mathcal{H}\mathrm{Env}}
\newcommand{\HESt}{\HE^{\Sp}}
\newcommand{\Map}{\mathrm{Map}}


\newcommand{\Cond}{\mathrm{Cond}}
\newcommand{\Condbar}{\mathbf{Cond}}
\newcommand{\Cat}{\mathrm{Cat}_\infty}
\newcommand{\proet}{\text{pro-et}}
\newcommand{\pep}{\ast_{\proet}}
\newcommand{\Sh}{\mathrm{Sh}}
\newcommand{\KCond}{\mathrm{K}^{\text{c}}}
\newcommand{\LCA}{\mathrm{LCA}}
\newcommand{\bA}{\mathbb{A}}
\newcommand{\Perf}{\mathrm{Perf}}
\newcommand{\PerfCat}{\mathrm{PerfCat}}
\newcommand{\lc}{\mathrm{lc}}
\newcommand{\K}{\mathrm{K}}
\newcommand{\LCAbar}{\mathbf{LCA}}

\newcommand{\Condf}[1]{\mathbf{#1}}

\newcommand{\bfC}{\mathbf{C}}
\newcommand{\bfD}{\mathbf{D}}
\newcommand{\CondFun}{\Condf{Fun}}
\newcommand{\CondCat}{\Condf{Cat}}
\newcommand{\cond}{\text{c}}
\newcommand{\CondD}{\Condf{D}}
\newcommand{\CondPerf}{\Condf{Perf}}
\newcommand{\CondLCA}{\Condf{LCA}}
\newcommand{\CondCone}{\Condf{cone}}
\newcommand{\CondK}{\Condf{K}}
\newcommand{\Condpi}{\Condf{\varpi}}

\newcommand{\Gal}{\mathrm{Gal}}
\newcommand{\dKSel}{\mathrm{dK}^{\mathrm{Sel}}}


\newcommand{\maxnote}[1]{\textcolor{blue}{#1}}
\newcommand{\peternote}[1]{\textcolor{orange}{#1}}

\newcommand{\cb}{\mathrm{cb}}

\newcommand{\fp}{\mathrm{fp}}
\newcommand{\Hom}{\mathrm{Hom}}
\newcommand{\Syn}{\mathcal{S}\mathrm{yn}}
\newcommand{\Bisyn}{\mathcal{B}\mathrm{isyn}}
\newcommand{\Stable}{\mathcal{S}\mathrm{table}}
\newcommand{\loc}{\mathrm{loc}}

\begin{document}

\section{\texorpdfstring{$\tau$- and $\lambda$-Bockstein spectral sequences}{Tau- and Lambda-Bockstein spectral sequences}}

In this section, we prove results about the $\tau$- and $\lambda$-Bockstein spectral sequences in $\Bisyn$. Let $\nu^2:\Sp\to\Bisyn$ denote the composition of the functors $\Sp\xrightarrow{\nu_E}\Syn_E\xrightarrow{\nu_F}\Bisyn$.

\begin{rem}
For this theorem, I'm using these assumptions:
\begin{itemize}
\item $\nu_F(\mathbb{S}^{k,w}_E) = \mathbb{S}^{k,w,k}$
\item SES in $\nu_EF_{*,*}$-homology induces cofiber sequence in $\nu^2$
\item $\nu^2$ is symmetric monoidal on projective objects
\item homotopy of $\nu^2(F)$-module is $\lambda$-free
\item $\lambda^{-1}$ is symmetric monoidal and satisfies $\lambda^{-1}\circ\nu_F=\mathrm{id}$
\item inclusion or equivalence $\Mod(\Bisyn;C\lambda)\simeq\Stable(\nu_EF_{*,*}\nu_EF)$
\end{itemize}
\end{rem}

\begin{thm}
    For the bisynthetic spectrum $\nu_F(Y)$ with $Y\in\Syn_E$, the $\nu^2(F)$-Adams spectral sequence is isomorphic to the $\lambda$-Bockstein spectral sequence. 
\end{thm}

\begin{proof}
  	Our proof of this follows very closely to the proofs in \cite[App. A]{BHS19}. Consider  the canonical $\nu_E(F)$-Adams tower

\begin{equation*}
	\begin{tikzcd}
\cdots \arrow[r] & Y_2 \arrow[r] \arrow[d] & Y_1 \arrow[r] \arrow[d] & Y_0 \arrow[r, phantom, "{=}" description] \ar[d] & Y \\
                 & Y_2\otimes\nu_E(F)      & Y_1\otimes\nu_E(F)      & Y_0\otimes\nu_E(F) &     
\end{tikzcd}
\end{equation*}

of $Y\in\Syn_E$. This tower is made up of cofiber sequences
\begin{equation*}
Y_{n+1}\to Y_n\to Y_n\otimes\nu_E(F)\to\Sigma^{1,0}Y_{n+1}
\end{equation*}

such that the maps $Y_{n+1}\to Y_n$ induce the zero map on $\nu_E(F)_{*,*}$-homology. By SES prop. of $\nu_F$ (!!!need this fact!!!), these become cofiber sequences
\begin{equation*}
\Sigma^{-1}\nu_F(Y_{n+1})\to \nu_F(Y_n)\to \nu_F(Y_n\otimes\nu_E(F))\to\nu_F(\Sigma^{1,0}Y_{n+1}).
\end{equation*}

in $\Bisyn$. By symmetric monoidality of $\nu_F$ on projective objects (!!!need this fact!!!), we can identify these cofiber sequences as
$$
\Sigma^{0,0,1}\nu_F(Y_{n+1})\to \nu_F(Y_n)\to \nu_F(Y_n)\otimes\nu^2(F)\to\Sigma^{1,0,1}\nu_F(Y_{n+1}).
$$

Hence, the canonical $\nu^2(F)$-Adams tower for $\nu_F(Y)$ can be written as
\begin{equation*}
	\begin{tikzcd}
\cdots \arrow[r] & \Sigma^{0,0,2}\nu_F(Y_2) \arrow[r] \arrow[d] & \Sigma^{0,0,1}\nu_F(Y_1) \arrow[r] \arrow[d] & \nu_F(Y_0) \arrow[r, phantom, "{=}" description] \ar[d] & \nu_F(Y) \\
                 & \Sigma^{0,0,2}\nu_F(Y_2)\otimes\nu^2(F)      & \Sigma^{0,0,1}\nu_F(Y_1)\otimes\nu^2(F)      & \nu_F(Y_0)\otimes\nu^2(F)     
\end{tikzcd} 
\end{equation*}

In the usual way, we get the Adams $E_1$-page
$$
\begin{aligned}
{}_{\nu^2F}E_1^{f,k,w,v}&=\pi_{k,w,v}(\Sigma^{0,0,f}\nu_F(Y_f)\otimes\nu^2(F)) \\
&\cong\pi_{k,w,v-f}(\nu_F(Y_f)\otimes\nu^2(F))
\end{aligned}
$$

Because of the equivalence
$$
\nu_F(Y_n\otimes\nu_E(F))\simeq\nu_F(Y_n)\otimes\nu^2(F),
$$

and the fact that the homotopy of a $\nu^2(F)$-module is $\lambda$-free (!!!need this fact!!!), we can identify the $E_1$-page as
$$
{}_{\nu^2F}E_1^{f,k,w,v}\cong {}_{\nu_EF}E_1^{f,k,w}\otimes \bZ[\lambda],
$$

where $x\in{}_{\nu_EF}E_1^{f,k,w}$ lives in quad-degree $(f,k,w,k+f)$ (I'm pretty sure, but this is just a guess!!! double check when we know about homotopy of $\nu^2F$-modules better).

The differentials in this spectral sequence are of the form
$$
d_r:{}_{\nu^2F}E_r^{f,k,w,v}\to {}_{\nu^2F}E_r^{f+r,k-1,w,v}.
$$

Since $\lambda^{-1}\circ\nu_F=\mathrm{id}$ and $\lambda$ is symmetric monoidal, the $\lambda$-inverted $\nu^2(F)$-Adams tower is the $\nu_E(F)$-Adams tower, and, in particular, the $\lambda$-inverted differentials 
$$
d_{r,\lambda^{-1}}:{}_{\nu_EF}E_r^{f,k,w}\to{}_{\nu_EF}E_r^{f+r,k-1,w}
$$

are exactly the differentials of the $\nu_E(F)$-Adams spectral sequence. Just as in \cite{BHS19}, this implies that if $d_{r,\lambda^{-1}}(x)=y$ is a $\nu_E(F)$-Adams differential, then
$$
d_r(x)=\lambda^{r-1}y.
$$

Using the equivalence (inclusion?)
$$
\Mod(\Bisyn;C\lambda)\simeq\Stable(\nu_EF_{*,*}\nu_EF),
$$

we see that the $\lambda$-Bockstein $E_1$-page
$$
{}_{\lambda}E_1^{f,k,w,v}=\pi_{k,w,v}(\Sigma^{0,0,-f}\nu_F(Y)\otimes C\lambda)
$$

and Adams $E_2$-page are, up to a degree shift, isomorphic. Because they have the same formula for differentials, they must be isomorphic as spectral sequences.
\end{proof}

\begin{rem}
We might actually need to prove a little bit more. \cite{BHS19} cites their Thm. 9.19(1) when they prove their version of this, but I'm not sure how exactly they're using that. Probably need to assume everything is nilpotent/lambda complete too, but don't feel like thinking about this for now. They also prove that the \textit{filtrations} are literally the same, not just that the spectral sequences are isomorphic.
\end{rem}

\begin{rem}
For the next lemmas and theorem, I need these assumptions:
\begin{itemize}
\item $\nu_F(\mathbb{S}^{k,w}_E) = \mathbb{S}^{k,w,k}$
\item $E$ Adams-type and ring map $E\to F$
\item $\nu_EP$, with $P$ finite $E_*$-projective, is "finite" in $\Syn_E$
\item $\nu^2$ is symmetric monoidal on projective objects
\item SES in $\nu_EF_{*,*}$-homology induces cofiber sequence in $\nu^2$
\item Homotopy of $\nu^2E$-module is $\tau$-free
\item Commuting diagram of functors
\begin{equation*}
\begin{tikzcd}
\Bisyn \arrow[r,"\tau^{-1}"]           & \Syn_F        \\
\Syn_E \arrow[u,"\nu_F"] \arrow[r,"\tau^{-1}"'] & \Sp \arrow[u,"\nu_F"']
\end{tikzcd}
\end{equation*}
\end{itemize}
\end{rem}

\begin{lem}
\label{Fprojlemma}
If a spectrum $X\in\Sp$ is $E_*$-projective, then it is $F_*$-projective. In particular for any spectrum $Y\in\Sp$,
$$
\nu_F(Y\otimes E)\simeq\nu_FY\otimes \nu_FE.
$$
\end{lem}

\begin{proof}
    By the assumption that $E$ is Adams-type and $F$ is an $E$-module, by a K\"unneth spectral sequence argument we have a natural isomorphism of functors
    $$
    F_*(-)\xrightarrow{\cong}E_*(-)\otimes_{E_*}F_*.
    $$
    In particular, if $E_*X$ is projective over $E_*$ then $F_*X\cong E_*X\otimes_{E_*}F_*$ is projective over $F_*$.

    The second statement follows from the fact that $E$ is a filtered colimit of finite, $F_*$-projective spectra and \cite[Lem. 4.24]{Pst22}.
\end{proof}

\begin{lem}
\label{nuFsymmonlemma}
The synthetic spectrum $\nu_EE\in\Syn_E$ is a filtered colimit of finite, $\nu_EF$-projectives. In particular, for $Y\in\Syn_E$, $$\nu_F(Y\otimes\nu_EE)\simeq\nu_F(Y)\otimes\nu^2(E).$$ 
\end{lem}

\begin{proof}
Let $E\simeq \colim _{\alpha} E_\alpha$ where each $E_\alpha$ is a finite, $E_*$-projective spectrum. Since $\nu_E(E)\simeq\colim_\alpha \nu_EE_\alpha$, it suffices to show that each $\nu_EE_\alpha$ is a finite, $\nu_EF$-projective synthetic spectrum. Now, $\nu_EE_\alpha$ is automatically finite (!!!make sure we define what this means!!!). For projectivity, we see that
$$
\begin{aligned}
\nu_EE_\alpha\otimes\nu_EF\simeq\nu_E(E_\alpha\otimes F),
\end{aligned}
$$

with homotopy groups $\pi_{*,*}(\nu_EE_\alpha\otimes\nu_EF)\cong F_*E_\alpha[\tau]$. By Lemma~\ref{Fprojlemma}, $F_*E_\alpha$ is projective over $F_*$. Tensoring with $\mathbb{Z}[\tau]$ preserves projectivity, so that
$F_*E_\alpha[\tau]$ is projective over $\nu_EF_{*,*}\cong F_*[\tau]$.  
\end{proof}

\begin{lem}
\label{nuEseslemma}
Suppose
$$
\nu_EX\to\nu_EY\to\nu_EZ
$$
is a cofiber sequence in $\Syn_E$, induced by a cofiber sequence $X\to Y\to Z$ in $\Sp$. If the cofiber sequence in $\Syn_E$ induces a short exact sequence in $\nu_EE_{*,*}$-homology, then it induces a short exact sequence in $\nu_EF_{*,*}$-homology. 
\end{lem}

\begin{proof}
By the assumption that $E$ is Adams-type and $F$ is an $E$-module, there are natural isomorphisms of functors
$$
\begin{aligned}
\nu_EF_{*,*}(\nu_E(-))&\cong F_*(-)[\tau] \\
&\cong (E_*(-)\otimes_{E_*}F_*)[\tau] \\
&\cong \nu_EE_{*,*}(\nu_{E}(-))\otimes_{\nu_EE_{*,*}}\nu_EF_{*,*}\, .
\end{aligned}
$$

Consider a map $Z\to\Sigma X$ induced by the cofiber sequence in $\Sp$. Note that $\nu_E(\Sigma X)\simeq\Sigma^{1,1}\nu_EX$ and the map $\nu_E Z\to\Sigma^{1,0}\nu_EX$ sits in a commutative diagram
\begin{equation*}
\begin{tikzcd}
\nu_EZ \ar[r, "f"] \ar[dr, "g"'] & \Sigma^{1,0}\nu_EX \ar[d,"\tau"] & \\
& \nu_E(\Sigma X) \ar[r, phantom, "{\simeq}" description] & \Sigma^{1,1}\nu_EX
\end{tikzcd}
\end{equation*}
with the horizontal and diagonal maps inducing the zero map in $\nu_EE_{*,*}$-homology. Since everything is $\tau$-free when we apply $\nu_EF_{*,*}(-)$ and the map $\nu_EF_{*,*}(g)$ is zero by the natural isomorphism above, then $\nu_EF_{*,*}(\tau)$ is injective so that $\nu_EF_{*,*}(f)$ is also the zero map.
\end{proof}

\begin{thm}
    For the bisynthetic spectrum $\nu^2(X)$ with $X\in\Sp$, the $\nu^2(E)$-Adams spectral sequence is isomorphic to the $\tau$-Bockstein spectral sequence. 
\end{thm}

\begin{proof}
The proof starts off similarly to the proof for the $\nu^2(F)$-Adams SS and the $\lambda$-Bockstein SS. Consider the canonical $E$-Adams tower

\begin{equation*}
	\begin{tikzcd}
\cdots \arrow[r] & X_2 \arrow[r] \arrow[d] & X_1 \arrow[r] \arrow[d] & X_0 \arrow[r, phantom, "{=}" description] \ar[d] & X \\
                 & X_2\otimes E      & X_1\otimes E      & X_0\otimes E &     
\end{tikzcd}
\end{equation*}

of $X\in\Sp$. This tower is made up of cofiber sequences
\begin{equation*}
X_{n+1}\to X_n\to X_n\otimes E\to\Sigma X_{n+1}
\end{equation*}

such that the maps $X_{n+1}\to X_n$ induce the zero map on $E_*$-homology. Applying $\nu_E$ to these cofiber sequences, as in \cite{BHS19} we get cofiber sequences
$$
\Sigma^{0,1}\nu_E(X_{n+1})\to \nu_E(X_n)\to \nu_E(X_n)\otimes \nu_E(E)\to\Sigma^{1,1} \nu_E(X_{n+1}).
$$
Since $F_*(-)\cong E_*(-)\otimes_{E_*}F_*$ and the maps $\Sigma^{0,1}\nu_E(X_{n+1})\to\nu_E(X_n)$ induce the zero map on $\nu_E(E)_{*,*}$-homology, the maps $\Sigma^{0,1}\nu_E(X_{n+1})\to\nu_E(X_n)$ also induce zero on $\nu_E(F)_{*,*}$-homology. This, together with Lemma~\ref{nuFsymmonlemma}, implies that there are cofiber sequences
$$
\Sigma^{0,1,1}\nu^2(X_{n+1})\to \nu^2(X_n)\to \nu^2(X_n)\otimes \nu^2(E)\to\Sigma^{1,1,1} \nu^2(X_{n+1}).
$$
Regrading by letting $Z_n=\Sigma^{0,0,n}\nu^2(X_n)$, these become cofiber sequences
$$
\Sigma^{0,1,0}Z_{n+1}\to Z_n\to Z_n\otimes \nu^2(E)\to\Sigma^{1,1,0} Z_{n+1}
$$

which build up the canonical $\nu^2(E)$-Adams tower
\begin{equation*}
	\begin{tikzcd}
\cdots \arrow[r] & \Sigma^{0,2,0}Z_2 \arrow[r] \arrow[d] & \Sigma^{0,1,0}Z_1 \arrow[r] \arrow[d] & Z_0 \arrow[r, phantom, "{=}" description] \ar[d] & \nu^2(X) \\
                 & \Sigma^{0,2,0}Z_2\otimes\nu^2(E)      & \Sigma^{0,1,0}Z_1\otimes\nu^2(E)      & Z_0\otimes\nu^2(E)     
\end{tikzcd} 
\end{equation*}
for $\nu^2(X)\in\Bisyn$. The $\nu^2(E)$-Adams $E_1$-page has the form
$$
\begin{aligned}
{}_{\nu^2E}E_1^{f,k,w,v}&=\pi_{k,w,v}(\Sigma^{0,f,0}Z_f\otimes\nu^2(E)) \\
&\cong\pi_{k,w-f,v}(Z_f\otimes\nu^2(E)).
\end{aligned}
$$
Since $Z_f\otimes \nu^2(E)$ is a $\nu^2(E)$-module, the $E_1$-page is $\tau$-free and via the $\tau$-inversion functor $\tau^{-1}$, we have an isomorphism
$$
{}_{\nu^2E}E_1^{f,k,w,v}\cong{}_{\nu_FE}E_1^{f,k,v}\otimes\bZ[\tau]
$$
\end{proof}

where $x\in{}_{\nu_EF}E_1^{f,k,v}$ lives in quad-degree $(f,k,k+f,v)$. The differentials in this spectral sequence are of the form
$$
d_r:{}_{\nu^2E}E_r^{f,k,w,v}\to {}_{\nu^2E}E_r^{f+r,k-1,w,v}.
$$

By Lemma~\ref{Fprojlemma} and the commutative diagram of functors
\begin{equation*}
\begin{tikzcd}
\Bisyn \arrow[r,"\tau^{-1}"]           & \Syn_F        \\
\Syn_E \arrow[u,"\nu_F"] \arrow[r,"\tau^{-1}"'] & \Sp \arrow[u,"\nu_F"']
\end{tikzcd}
\end{equation*}

the symmetric monoidal $\tau$-inversion functor $\tau^{-1}:\Bisyn\to\Syn_F$ sends the $\nu^2(E)$-Adams tower to a $\nu_FE$-Adams tower in $\Syn_F$. In particular, the $\tau$-localized differentials $d_{r,\tau^{-1}}$ are exactly the $\nu_FE$-Adams spectral sequence differentials and, so, if $d_{r,\tau^{-1}}(x)=y$, then
$$
d_r(x)=\tau^{r-1}y.
$$

Using the equivalence (inclusion?)
$$
\Mod(\Bisyn;C\tau)\simeq\Stable(\nu_FE_{*,*}\nu_FE),
$$

we see that the $\tau$-Bockstein $E_1$-page
$$
{}_{\tau}E_1^{f,k,w,v}=\pi_{k,w,v}(\Sigma^{0,-f,0}\nu^2(X)\otimes C\tau)
$$

and Adams $E_2$-page are, up to a degree shift, isomorphic. Because they have the same formula for differentials, they must be isomorphic as spectral sequences.

\printbibliography

\end{document}